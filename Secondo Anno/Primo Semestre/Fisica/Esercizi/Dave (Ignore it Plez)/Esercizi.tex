\documentclass[12pt, a4paper, openany, twoside]{book}
\usepackage[italian]{babel}
\usepackage[T1]{fontenc}
\usepackage[utf8]{inputenc}
\usepackage{amsmath} 
\usepackage{xcolor}
\usepackage[margin=1in]{geometry}
\usepackage{hyperref}
\usepackage{graphicx}
\graphicspath{{./img/}}
\usepackage{tikz}
\hypersetup{
    colorlinks=true,
    linkcolor=blue,
    filecolor=magenta,      
    urlcolor=cyan,
}
%usepackage[latin1]{inputenc}
\begin{document}
\fontfamily{cmss}\selectfont
\author{DaveRhapsody}
\title {Esercizi di Fisica}
\date {15 Ottobre 2019}
\maketitle
\tableofcontents
\chapter{Cinematica ad una dimensione}
\section{Es. 1}
Si considerino i seguenti dati:
\begin{itemize}
	\item $\Delta_{x_{1}} = 5,2km$
	\item $v_{1} ~ costante ~ 43km/h$
	\item $\Delta_{x_{2}} = 1,2km$
	\item $\Delta_{t_{2}} = 27m$
\end{itemize}
La richiesta è:
\begin{enumerate}
	\item $\overrightarrow{V_{media}}$ di tutto il tratto 
%	\item $\overrightarrow{V_{media}}$ di tutto il tratto + il tragitto di ritorno
%	fino alla posizione 1 ($x_{1}$)
\end{enumerate}
Come si risolve? \\
In pratica per ottenere la velocità media si devono avere le rispettive velocità
1 e 2. Come osserviamo $v_{1}$ è già presente. Mentre ciò che ci serve sarà 
qualcosa del tipo $\overrightarrow{v_{media} = \frac{\Delta x_{tot}}{\Delta t_{tot}}}$.
\\ \\
$$\Delta x_{tot} = \Delta x_{1} + \Delta x_{2}$$
ovvero 5,2 + 1,2 = 6,4 km T O T A L I \\
Ora ci servirà anche il tempo totale, perciò:
$$\Delta t_{tot} = \frac{\Delta x_{1}}{v_{1}} + \Delta t_{2} = \frac{5,2 km}
{43 km/h} + \frac{27 min}{60} = 0,571 h$$
\section{Es. 2}
Si considerino i seguenti dati:
\begin{itemize}
	\item $\Delta x = 1100m$
	\item $A = 1,2m/s^{2}$
	\item Da un certo punto in poi otterremo una decelerazione pari a - A
\end{itemize}
E' richiesto sapere:
\begin{itemize}
	\item $\Delta t_{tot}$
	\item $V_{max}$
\end{itemize}
Si ha che $V_{1}^{2} = V_{i}^{2} + 2a \Delta \frac{x}{2}$, e 
$V_{1} = \sqrt{2a \cdot \frac{\Delta X}{2}} = \sqrt{2\cdot1,2m/s^{2}\cdot \frac{1100}{2}} = 36,5 m/s$
Ora ragioniamo sul tempo. \\ \\
$$t_{1} = \frac{v_{1}}{a} = \frac{36,3 m/s}{1,2 m/s^{2}} = 30,3s$$
Sì, mancano tutti gli altri ma li aggiungerò tramite foto a breve
\chapter{Moto circolare}
\section{Es. 1}
Si considerino i seguenti dati
\begin{itemize}
	\item R = 0,1m
	\item $t_{0} = 0$
	\item $V_{0} = 0,05 m/s$
	\item $t_{1} = 1s e S(1) = 8cm$
\end{itemize}
E' richiesto sapere:
\begin{itemize}
	\item L'accelerazione a(t)
\end{itemize}
Sapendo che $a_{c} = \frac{v^{2}}{R}$
L'equazione del moto è la legge oraria del Mrua PERO' considerando la velocità
tangenziale, ovvero.
$$s_{0} = v_{0}\cdot t + \frac{1}{2} a_{tang}\cdot t^{2}$$
Dobbiamo solo fare sostituzioni ora, quindi 
$$a_{t} = \frac{s(t) - v_{0}\cdot t}{\frac{1}{2}a\cdot t^{2}}$$ = 
$$\frac{8\cdot 10^{-2}m - 0,005 m/s \cdot 1s}{\frac{1}{2}\cdot 1s}$$
    
\end{document}