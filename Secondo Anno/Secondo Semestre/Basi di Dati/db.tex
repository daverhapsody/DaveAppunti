\documentclass[12pt, a4paper, openany, twoside]{book}
\usepackage[italian]{babel}
\usepackage[T1]{fontenc}
\usepackage[utf8]{inputenc}
\usepackage{amsmath} 
\usepackage{xcolor}
\usepackage[margin=1in]{geometry}
\usepackage{hyperref}
\usepackage{graphicx}
\graphicspath{{./img/}}
\usepackage{tikz}
\hypersetup{
    colorlinks=true,
    linkcolor=blue,
    filecolor=magenta,      
    urlcolor=cyan,
}
%usepackage[latin1]{inputenc}
\begin{document}
\fontfamily{cmss}\selectfont
\pagestyle{plain}
\author{DaveRhapsody}
\title {Basi di Dati}
\date {4 Marzo 2020}
\maketitle
\tableofcontents
\chapter*{Introduzione}
Un database, o base di dati, o db (che scriverò d'ora in poi) è un insieme di 
dati, tipo deposito, per qualsiasi genere di uso, sia aziendale, che personale.
I suddetti dati sono inseriti, letti, e soprattutto \underline{organizzati} 
secondo certe regole.
\paragraph{Alcuni esempi: }
\begin{itemize}
	\item Agende telefoniche
	\item Studenti di una classe o scuola
	\item Qualsiasi insieme generico in cui ogni elemento differisce da un altro
	secondo delle linee guide (campi) che si decidono alla base.
\end{itemize}
Da Linguaggi di Programmazione sono state viste le Struct, o Record, che sono
delle strutture per i dati statiche, concrete ed eterogenee, aventi più campi
non necessariamente dello stesso tipo. \\ \\
Bene, un DB è un array di record, in cui bisogna garantire integrita, consistenza
e NON ridondanza dei dati.
\section{Interazione tra campi}
Tra loro i campi di questi record possono interagire, nel senso che partendo dal
valore di un determinato campo, si può ricavare il valore di un altro campo.
Esempio? Il \underline{codice fiscale}, che è ricavato da una formula che non 
ricordo mai nella vita forever MA prendendo come dati il nostro nome, cognome, 
etc.
\paragraph{Questo sarà un campo calcolato}
\section{Dove si colloca un DB?}
\begin{itemize}
	\item Intefaccia utente
	\item Applicativi
	\item Software di ambiente e di sistema
	\item DB
	\item Sistema operativo
	\item Hardware
	\item Sistema di comunicazione di rete
\end{itemize}
Impropriamente si può definire nel mezzo
\section{Problemi da NON avere}
\begin{itemize}
	\item Ridondanza dei dati: non devono esserci dati ripetuti, ogni record è
	U N I V O C O. Per renderlo tale credo nel corso che vedremo come si fa, 
	spoiler: chiavi, la nostra futura bacinella di bestemmie.
	\item Rischi di incoerenza: i dati devono essere consistenti, ossia dato un
	valore, se lo si attribuisce ad un simbolo (dato, variabile, campo), quel
	valore dovrà essere SEMPRE quello, per ogni volta che si richiama quel 
	simbolo
\end{itemize}
\section{Condivisione dei dati}
Data un'organizzazione avente più dipendenti, è naturale che la suddetta possa
condividere un determinato insieme di dati, infatti ad ogni settore corrisponde
un sistema informativo (Per chi ha fatto economia, il SIA).\\ \\
Cosa accade quando si condivide una risorsa? Esatto, bisogna fare in modo che
non avvengano accessi concorrenti, quindi sono implementate funzioni e 
procedure di prevenzione di questo genere di problemi. Un DB non protetto da
modifiche NON consentite, oltre a fare schifo tipo fortissimo forever maonna guarda
da bruciarlo, è NON integro. 
\paragraph{Un DB deve essere integro}
\chapter{DBMS}
Il DBMS (Database Management System) è un software in grado di gestire i 
DB.. E grazie al piffero direi, ma perchè si usa? Perchè un DB è una risorsa
condivisa, e siccome potrebbe contenere dati importanti, serve un software che
consenta di tenerla pulita e integra.
\section{Cosa gestiscono}
Le moli di dati su cui operano i DBMS sono tendenzialmente di grandi dimensioni,
persistenti (con periodo di vita indipendenti dalle singole esecuzioni dei 
programmi che ci lavorano) e condivise, nel senso che diverse applicazioni ci 
possono lavorare sopra.
\section{Cosa devono garantire}
Dalle slides si riportano queste 3 qualità:
\begin{itemize}
	\item Affidabilità
	\item Sicurezza
	\item Effiecienza
\end{itemize}
E tra l'altro bastava anche solo specificare consistenza ed integrità, ma il 
ci siamo intesi
\section{Concetti fondamentali}
\subsection{Schema}
Lo schema è la descrizione dei campi di una tabella (banalmente, è come una 
classe in Java)
\subsection{Istanza}
L'istanza è un record allocato a cui assegno un valore per ogni campo (in Java
erano gli oggetti)
\subsection{Modello}
Il modello è l'insieme dei vari costrutti (o regole) utilizzati per organizzare
i dati e descriverne i cambiamenti nel tempo. 
\paragraph{Cosa compone un modello}
Le strutture di rappresentazione dei dati: Nel nostro caso si analizzerà 
la relazione. In base alla struttura cambia il modello, ad esempio il modello
\textbf{relazionale} userà la relazione, mentre il modello a oggetti ad esempio
userà altre strutture.
\subsection{Tipo di modello}
Ce ne sono due:
\paragraph{Logico:} Vengono utilizzati dai programmi e non dipendono
dalle strutture fisiche. Esempio: Relazionale, reticolare, etc.
\paragraph{Concettuale:} Sono ancora più in alto del modello logico,
quindi sono anche indipendenti dal DBMS, e sono delle descrizioni del mondo
reale, usati in fase di progettazione (Quando studieremo il modello E-R 
partiranno le imprecazioni)
\section{Architettura dei DBMS}
Tra un BD e L'udente (o i programmi) ci sono due schemi, schema fisico e schema
logico. Lo schema fisico è vicino al DB mentre il logico all'utente. (Quando 
verrà detto Utente, si intende sia Umano che Programma).
\subsection{Schemi}
Come menzionato prima ci sono schema fisico e logico, il fisico è banalmente
la reppresentazione di files, blocchi di memoria, cache etc. mentre il logico
è quello che abbiam visto prima, quindi schema relazionale etc.
\subsection{Indipendenza dei dati}
Il livello logico è indipendente dal fisico, nel senso che se tu hai un record
avente dentro un ID e altri due campi, ti importa poco se verrà salvata su un
SSD o su un floppy disk di fine 1800, sempre un record dovrai avere.
\paragraph{Osservazione:} Prendiamo l'ipotesi di un record avente id e numero
di telefono. (Non l'ho specificato ma l'ID è un campo numerico intero 
che identifica univocamente il record, fine) Se volessimo in modo sadico dividere
il numero in prefisso + numero? Esatto, cambia lo schema logico, ed il record
passa da 2 a 3 campi.
\paragraph{E lo schema fisico?} Dalle slide non si capisce, quindi azzardo una
risposta: siccome lo schema logico è indipendente dal fisico, non dovrebbe
cambiare.. Ma non ne son sicuro.
\subsection{Le viste}
Se l'utente accedesse di cattiveria direttamente allo schema logico, ad ogni
cambiamento allora dovrebbe cambiare tutto il suo programma. Come si risolve?
Con le viste (O schemi esterni) che sono dei sottoinsiemi dello schema logico,
che quindi contengono quantità di dati limitate (allo scopo di quell'applicativo).
\paragraph{In che senso?} Se l'utente è la portineria, gli si crea una vista
avente dati legati alla portineria.
E sì, schema logico ed esterno sono indipendenti dal fisico, questo significa
che l'applicativo non cambierà se cambio il supporto fisico su cui ho messo il
DB. E addirittura il livello esterno è indipendente dal logico.
\subsection{Linguaggi del DBMS}
Ce ne sono due, uno è un linguaggio descrittivo dei dati, e l'altro è di 
manipolazione (sql). Uno descrive le strutture, l'altro ci scrive dentro e 
legge.
\chapter{Modello Entità-Relazione}
Il modello E-R è uno schema concettuale che consente di rappresentare la 
realtà tramite entità e relazioni tra esse.
\section{Concetto di entità}
E' un'astrazione di un certo insieme di dati (come le classi in Java). Graficamente
un'entità è rappresentata da un rettangolo con all'interno il nome dell'entità.
\paragraph{Il nome dev'essere descrittivo},
ad ogni entità inoltre occorre dare una definizione. Del tipo Automobile, è 
quell'oggetto che se è colorato di rosso ed ha un V8 sotto il cofano ti permette
di avere orgasmi multipli (dai che mi volete bene, lo so <3).
\paragraph{L'istanza di un'entità è il caso specifico:} Automobile = entità, 
Bmw, Ferrari, Porsche è l'istanza. Il nome che identifica un'entità deve essere
singolare ed espressivo, no abbreviazioni, no codici.
\section{Attributo dell'entità}
E' una proprità, o meglio, un campo del record che serve ai fini dell'applicazione
, questo valore dipende solo dall'istanza dell'entità, non dipende da altri
elementi nello schema. Inoltre ogni attributo ha un intervallo di valori finito.
(Il concetto di infinito in informatica non esiste se non per le mie funzioni
ricorsive rottissime di Prolog e Lisp) 
\paragraph{Reppresentazione:}
\begin{center}
\includegraphics[width=0.75\textwidth]{2.png}
\end{center}
Se il campo è chiave, il pallino diventa pieno. (Sto odiando questa slide, è 
tutto così non simmetrico che sclero)
\section{Le Relazioni}
Una relazione (o associazione) è un legame logico tra più entità, ed il numero
di esse determina il grado (numero di entità obv), si rappresenta con un rombo
e dentro ci si scrive cosa lega le entità. Come in questo esempio:
\begin{center}
\includegraphics[width=0.75\textwidth]{3.png}
\end{center}
Siccome c'è differenza tra queste relazioni e le relazioni del modello relazionale,
per comodità le chiamerò associazioni. Quindi associazioni -> modello e-r, relazioni ->
modello relazionale.
\paragraph{Come scrivere le associazioni:} Bisogna usare dei sostantivi al 
singolare. Come da esempio: Se ho due entità studente ed esame ed in mezzo ci
metto "Supera" è sbagliato. Si deve mettere "Esame superato". (Alle superiori
ero abituato che il verbo andasse all'infinito, che ha più senso ed è più 
leggibile, ma icchè vi devo dì, noi s'attacca l'asino indoe vuole i' padrone).
\paragraph{Tra due entità posso insierire più associazioni:} Ad esempio prendendo
uno studente ed un corso, ci puoi applicare due associazioni, una può essere
"frequenZa" e l'altra boh.. "Frequenza passata". E da notare che nelle slides 
ha messo "Frequenta" e "Frequentato in passato".. Niente verbi, ovvio.
\paragraph{Anche un'associazione può avere degli attributi:} E' una proprietà
\textbf{locale} che serve ai fini dell'utente, non è una proprietà della singola
entità ma di tutte coloro che sono in legame.
\paragraph{In termini umani:} E' una funzione che associa ad ogni istanza di 
quella associazione un valore che appartiene ad un dominio dell'attributo.
\paragraph{Esempio:} Studente - \underline{superamento} - Esame, un attributo
di superamento potrebbe essere Voto. (Tra l'altro lui ha messo esameSuperato.. 
boh, diobono ok che superato è un aggettivo, ma è participio passato di superare,
queste slides sono un casino).
\section{Vincoli di cardinalità}
La cardinalità di una realzione specifica le quantità di interazione di istanze
di più entità in un'associazione. si specifica un valore minimo ed uno massimo.
Quindi è un dominio, un insieme finito di valori che è possibile assumere.
\paragraph{Definizione:} Un vincolo di cardinalità è un limite di istanze di
una associazione a cui può partecipre un'istanza di una o più entità, caratterizza
meglio il significato di una associazione.
\paragraph{Esempio banale babbarello:} Studente - \underline{Frequenza} - Corso,
Da 0 a N studenti frequentano N corsi. 1 corso può essere frequentato da N
studenti. 1 studente frequenta N corsi.
\section{Tipi di assoiazione}
Rimanendo nel contesto delle relazioni binarie, ci sono 3 tipi di relazione:
\begin{enumerate}
	\item Relazione uno a uno
	\item Uno a Molti
	\item Molti a Molti
\end{enumerate}
Le quali si spiegano da sole, non c'è molto da specificare.
\section{Generalizzazioni e Relazioni Is-A}
Partendo dall'Is-A
\subsection{Relazione IS-A}
E' una relazione che introduce il fatto che un'istanza possa appartenere a più
entità, esempio Pilota, Pilota di Rally. E' anche definita relazione di sottoinsieme,
e si utilizza per entità in cui una sia sottoinsieme dell'altra, o meglio. Una 
deve essere "padre" dell'altra, che si chiamerà "figlia".
\paragraph{Esempio:} Fuoristrada \underline{Is-A} Automobile \underline{Is-A} Veicolo
\paragraph{Principio di ereditarietà:} Ogni attributo e proprietà dell'entità
padre è anche proprietà della sua sottoentità, va da sè che la figlia può avere
anche le sue proprietà. (Oh è come le classi di Java, funziona tutto così)
\paragraph{Graficamente:}
\begin{center}
\includegraphics[width=0.75\textwidth]{4.png}
\end{center}
\subsection{Generalizzazione tra entità}
Hai un'entità padre ed una serie di figlie, quello che fai è con un arco unirle,
fine. 
\begin{center}
\includegraphics[width=0.75\textwidth]{5.png}
\end{center}
Una generalizzazione può essere di due tipi:
\begin{itemize}
	\item Completa: L'unione delle istanze delle sottoentità è uguale all'insieme
	delle istanza del padre
	\item Non completa: in cui questo non accade
\end{itemize}
\paragraph{Spiegato meglioglioglio:} Hai uomo e donna, e poi essere umano, se prendi
tutti gli uomini + tutte le donne e fai un insieme contenente tutti essi, ottieni
l'essere umano. (Sì, LGBT, abbiate pietà, non è il momento di polemizzare) \\
Ipotizziamo gli animali: hai cane gatto e criceto, e poi animali.. Beh se sommi
tutti i gatti criceti e cani, non ottieni l'insieme degli animali. Mancano i
gamberetti per esempio.
\paragraph{Detto scientificamente accurato:} Se le sotto entità sono partizioni
dell'entità padre, è completa, altrimenti no.
\paragraph{L'entità padre può avere più generalizzazioni}
\section{Attributi composti} Un attributo può esser definito su un dominio di più
campi. In altri termini, un attributo può essere a sua volta un record. Tipo
indirizzo (composto da via numero e cap), è uno di questi. Potrebbe essere anche
un attributo del tipo "dati anagrafici". Quando si ha questa situazione l'attributo
è composto.
\section{Relazioni n-arie}
La relazione BI-naria coinvolge due entità, la ternaria tre, la quaternaria.. 
Via, si è capito, se è n-aria coinvolge n entità. A sua volta anche la relazione
può avere i suoi attributi.
\paragraph{Esistono anche relazioni ricorsive o definite su se stesse:} Il classico
esempio è il papa. papa - \underline{successione} - papa, per tenervelo in testa
pensate a "Morto un papa, se ne fa n'altro".
\section{Identificatori/Chiavi}
E' un insieme di attributi o relazioni che ti permettono di identificare le
istanze di un'entità. (Per chi ha già fatto sta roba sono le chiavi primarie).
Ogni entità ha un certo insieme di identificatori (Da 1 a $\infty$, basta che
ce ne sia uno $\forall$ entità)
\subsection{Tipi di identificatore/chiave}
\begin{itemize}
	\item \textbf{Interno}: Formati solo da attributi dell'entità stessa. può 
	essere solo uno, possono essere anche più di uno. Come in questo esempio:
	\begin{center}
	\includegraphics[width=0.75\textwidth]{6.png}
	\end{center}

	\item \textbf{Esterno}: E' formato da attributi dell'entità e da relazioni
	che la coinvolgono, oppure solo da queste relazioni.
	\paragraph{Esempio grafico:}
	\begin{center}
	\includegraphics[width=0.75\textwidth]{7.png}
	\end{center}
\end{itemize}
\chapter{Livello logico}
Il livello logico è quello in cui figurano le strutture di rappresentazione 
logica dei dati. In altri termini, dove c'è uno schema logico che descrive una
realtà di interesse. E già qua occhio, non è uno schema concettuale come l'E-R,
qua non si parla di generalizzazione della realtà, ma di schema logico dei dati. 
\section{Modello relazionale}
Il concetto di relazione è da considerarsi in 3 modi diversi:
\begin{enumerate}
	\item Relationship (Associazione, classe di fatti)
	\item Relazione matematica (Qualunque sottoinsieme del prodotto cartesiano 
	blablabla)
	\item Relazione secondo il modello relazionale
\end{enumerate}
(Non mi soffermerò sulla spiegazione della relazione matematica, tenete solo
a mente che le n-uple al loro interno sono orninate e che tra le n-uple non c'è
ordinamento. Se volet buttateci dentro anche che ogni n-upla è univoca)
\subsection{Definizioni base}
\begin{itemize}
	\item \textbf{Struttura non posizionale}: Per ogni dominio si associa un attributo
	che ne descriva il ruolo
	\item \textbf{Relazione (o Tabelle)}: Letteralmente la rappresentazione
	matematica di una relazione in cui i valori per ogni colonna sono dello
	stesso tipo, e le righe sono diverse. In termini leggibili? Un cazzo di array 
	di record, con tutte le conseguenze che ne derivano.
	\paragraph{Osservazione:} Si è specificato "valori" per ogni colonna, NON a
	caso, perchè nel relazionale ci si basa su valori, ed i riferimenti tra i 
	dati in relazioni diverse sono rapprentati con valori dei domini che 
	compaiono nelle n-uple.
	\paragraph{La domanda è: Per quale f* motivo?} Le ragioni sono 3:
	\begin{enumerate}
		\item L'utente finale vede gli stessi dati dei programmatori, costituiti
		da tabelle
		\item L'indipendenza della rappresentazione delle strutture fisiche, 
		che cambiano in modo indipendente, quindi potete usare i floppyni del
		Nintendo o un SSD Nvme, I N D I P E N D E N Z A.
		\item i dati sono trasportabili da un sistema ad un altro per la ragione
		del punto 2
	\end{enumerate}
	\item \textbf{Schema di relazione}:
	\[
		nomeRelazione(Attributo_{1}, Attributo_{2}, Attributo_{n})
	\]	
	Facciamo che da ora nomeRelazione $\to$ nR e Attributo $\to$ A
	\item \textbf{Schema di Base di Dati}:
	\[
		superRelazione = \{nR_{1}(A_{1}, nR_{2}(A_{2}, ..., nR_{n}(A_{n}, )\}
	\]
	\item \textbf{N-upla}: Funzione che associa ad un certo insieme X di attributi,
	$\forall$ attributo A $\in$ X uno ed un solo valore del dominio di A
	\item \textbf{Istanza di relazione}: Data nR(A), l'istanza è l'insieme r di
	n-uple su nR
	\item \textbf{Istanza di Base di Dati}: Stesso identico concetto ma applicato
	allo schema della base di dati. Dato lo schema di prima (superRelazione),
	l'istanza è l'insieme delle relazioni r
	\item \textbf{Valori nulli}: Quando un VALORE non c'è in una determinata
	colonna, viene considerao Null 
\end{itemize}
\section{Strutture nidificate}
La realtà è rappresentabile con infiniti schemi equivalenti, nel \textbf{contenuto
informativo}. Ciò vuol dire che schemi equivalenti forniscono lo stesso
insieme di informazioni.
\paragraph{Premessa:} immaginatevi di volere rappresentare tutti gli scontrini di
un ristorante in cui ogni ricevuta ha come campi: \{Numero ricevuta, data emissione
e le varie voci di costo\}
\subparagraph{Le voci di costo:} Per ogni voce c'è una corrispondenza con un 
gruppo di portate, e per ciascun gruppo, il numero di porrtate e il costo del gruppo 
di portate, infine il totale.\\
Con un semplice schema ad elenco si ottiene alla fine la suddivisione in sezioni,
sottosezioni etc., riscrivendolo così vien fuori:
\begin{enumerate}
	\item Numero ricevuta
	\item Data Emissione
	\item Voci di costo
	\begin{enumerate}
		\item Gruppo di portate
		\begin{enumerate}
			\item Numero di portate
			\item Costo del gruppo di portate
		\end{enumerate}
	\end{enumerate}
	\item Totale speso
\end{enumerate}
\section{Vincoli}
Può accadere che dei DB siano sintatticamente corretti MA non rappresentano degli
stati realmente possibili nella realtà, ad esempio il 27 e lode (esempio delle
slides).
\paragraph{Risoluzione:} Per risolvere questo inconveniente occorre aggiungere
un vincolo che dica SE è 30 allora può avere lode, altrimenti non può. (Lode
può essere benissimo un campo booleano).
\section{Vincolo di integrità}
E' una proprietà che hanno i DB che va soddisfatta da tutte le istanze di uno
schema, le quali rappresentano informazioni corrette, consistenti, integre per
l'applicazione.
\paragraph{Banalmente:} E' una funzione booleana (O predicato for those who 
love Prolog) che per ogni istanza r dà:
\begin{itemize}
	\item \textbf{TRUE}: se l'istanza rappresenta la realtà
	\item \textbf{FALSE}: In caso contrario
\end{itemize}
\paragraph{A che servono?} Permettono di rappresentare in modo più preciso la
realtà, e quindi contribuiscono ad avere dati più consistenti. Inoltre sono
utili per progettare perchè lo schema risulta qualitativamente migliore
\subparagraph{Tipi di vincolo}
\paragraph{Vincolo intrarelazionale:} Definito all'interno della relazione.
Sono vincoli applicati agli \textbf{attributi} (si definiscono anche vincoli
di \textbf{dominio}), oppure vincoli di n-upla, o comunque relativi all'insieme
di n-uple di una relazione $\to$ Vincoli di relazione
\paragraph{Interrelazionale:} Definiti tra più relazioni
\section{Vincolo di n-upla}
Esprime condizioni sui valori di ciascuna n-upla in una relazione, ad esempio
non puoi avere due giocatori in una squadra con stesso numero di maglia.
\subparagraph{Sintassi}
\begin{itemize}
		\item Esressioni booleane (AND, OR, NOT)
		\item Atomi da confrontare (>,<, >= etc.)
\end{itemize}	
Per calcolare il valore di verità di un vincolo in una n-upla, occorre sostituire
i valori degli attributi nella n-upla, e poi calcolare il valore vero o falso.
\section{Vincolo di chiave}
C'è bisogno di individuare le informazioni che permettono di rappresentare ogni
oggetto di interesse con una n-upla differente e identificarlo nel caso in
cui ne abbia necessità.
\subsection{Cos'è una chiave} E' un insieme di attributi che identificano in modo
univoco le n-uple di una relazione. 
Un insieme K di attributi è \textbf{superchiave} per una relazione r se r non 
contiene due n-uple distinte $t_{1}$ e $t_{2}$ con $t_{1}[K]$ = $t_{2}[K]$.
\\ Una chiave tipica è il numero di matricola, che identifica uno studente, e
siccome è un solo campo, è anche minimale. \\
Se invece hai [Matricola + cognome] non sono una chiave MA sono superchiave, 
e non è minimale siccome c'è già matricola.
\subparagraph{Vincoli, schemi e istanze}
I vincoli corrispondono a proprietà del mondo reale modellato dal DB. Sono
proprietà dello schema, fanno riferimento ad ogni istanza. Se hai dei vincoli
in uno schema, se la istanza lo soddisfa, è corretto (per ogni vincolo).	
\subsection{Esistenza delle chiavi}	
Una relazione non può contenere n-uple distinte ma uguali. Ogni relazione ha come
superchiave l'insieme degli attributi su cui è definita e ha (almeno) una chiave.
Rappresentiamo per ogni dipartimento di un'università i fornitori di beni ed i 
beni forniti.
\paragraph{Precisazione:} Distinte ma uguali significa che hai un'istanza doppia,
duplicata, ripetuta, ridondante. 
\subparagraph{Importanza delle chiavi}
Identificano e distinguono gli oggetti gli uni dagli altri, e quindi ogni dato
diventa accessibile, inoltre permettono di collegare i dati in realzioni diverse.
\section{Valori nulli}
Un valore nullo indica una generica informazione incompleta, assente, non 
esistente. Un'informazione che a tutti gli effetti non c'è. Occhio a questa cosa,
spesso si fa confusione con il concetto di 0 ed il concetto di nullo: Nullo è 
NON avente valore, mentre 0 è un valore.\\ \\
Esempio su un conto corrente: la casella che indica quanti soldi sono presenti
\[
\begin{cases}
	SE ~ var = null ~ allora ~ NON ~ si ~ ha ~ questo ~ dato\\
	SE ~ var = 0 ~ allora ~ hai ~ 0
\end{cases}
\]
Ci sono 3 principali tipi diversi di valore nullo.
\paragraph{Sconosciuto:} Quando il valore esiste ma non è noto
\paragraph{Inesistente:} Quando invece questo valore non esiste e non è applicabile.
\paragraph{Nessuna informazione:} Nel caso in cui o è sconosciuto o inesistente..
\section{Informazione incompleta}
Come si rappresenta a questo punto l'informazione nulla? Teniamo a mente che i 
dati vanno separati dalle applicazioni, e soprattutto il significato dei dati non
deve essere nascosto nei programmi.
\paragraph{Soluzione:} Il dominio dei valori si estende aggiungendo un valore nuovo
che si chiama NULL, che per l'appunto indica esattamente il valore "inesistente"
di prima. Fine
\subsection{Ammissione di valori nulli}
Ci sono casi in cui non si può accettare un valore nullo, poichè renderebbe il 
record incompleto, non univoco, e quindi ridondanze, inconsistenze etc. 
\paragraph{In presenza di valori nulli:} I valori della chiave NON permettono
di svolgere le due funzioni della chiave (Identificare i record, esprimere i 
riferimenti ad altre tabelle)
\paragraph{Se la chiave primaria fosse nulla?} Verrebbe un agente lì nel punto in
cui c'è il programmatore che ha permesso una cosa simile, e gli sparerebbe in 
fronte.
\section{Vincoli di integrità referenziale}
E' la proprietà per cui le informazioni con stesso significato in relazioni 
diverse sono correlate, rispettando la proprietà per cui: 
\begin{itemize}
	\item Non è possibile che un oggetto dela realtà
    \begin{enumerate}
    	\item Sia rappresentato nella sua relazione con un altro oggetto
    	\item Non sia rappresentato e identificato nella tabella che lo 
    	descrive nativamente
    \end{enumerate}
\end{itemize}
\paragraph{Per definizione:} Un vincolo di integrità referenziale fra un 
insieme X di attributi di una relazione $R_{1}$ ed un'altra relazione $R_{2}$
impone ai valori su X in $R_{1}$ di comparire come valori della chiave primaria
di $R_{2}$
\paragraph{Ok, e se viene violato?} In genere i DBMS per evitare problemi di 
questo tipo agiscono in tre modi principali.
\begin{enumerate}
	\item Rifiutano l'operazione
	\item Applicano una Eliminazione in cascata: Questa è la più cattiva, perchè
	va a pescare ovunque c'è quel valore chiave, e cancella quel record
	\item Introducono dei valori nulli
\end{enumerate}
\paragraph{E se il vincolo è su più attributi?} Come in molti altri ambiti è
necessario che venga specificato un ordine tra gli attributi.
\chapter{Progettazione concettuale}
La progettazione concettuale è la fase che segue la stesura dei requisiti utente,
in quesa fase si raccolgono i requisiti e a partire da questi si crea lo schema
concettuale (Rappresentazione astratta dei requisiti).
\paragraph{Tipici problemi legati alla stesura dei requisiti:} Il cliente non 
è in grado di spiegarsi, vuole una cosa, ma ti racconta una cosa divesa, ha 
poche idee, riesce a confondere pure quelle. (Se non sbaglio c'è su Aps una foto
carina a tema)
\section{Necessità di strategie}
Quando i requisiti sono semplici, costituiti da poche frasi, allora si può generare
uno schema "tutto in una volta", ma se ti becchi requisiti articolati (solito
esempio di una banca), serve metodo.
\paragraph{Ad un problema esistono infinite soluzioni, ed infinite soluzioni
sbagliate.} Pensate quindi cosa vuol dire effettivamente capire se la propria è
la strategia giusta.
\paragraph{Per definizione:} Si definisce Analisi dei requisiti e della progettazione
concettuale, l'insieme di:
\begin{itemize}
	\item Acquisizione dei requisiti
	\item Analisi dei requisiti per rimuovere ambiguità e ridondanze
	\item Costruzione dello schema concettuale
	\item Costruzione di un glossario dei termini utilizzati
\end{itemize}
\section{Definizione di qualità}
Una qualità di uno schema concettuale è una caratteristica che sarebbe auspicabile
fare avere allo schema, e va garantita alla fine della fase di progettazione
concettuale.
\paragraph{Di che qualità si parla?}
\begin{itemize}
	\item Correttezza
	\begin{itemize}
		\item Rispetto ai requisiti: I requisiti rappresentati nello schema
		devono essere rappresentati con il significato dei concetti nell'E-R
		\item Rispetto al modello: I concetti del modello sono usati nello schema
		in accordo al loro significato
	\end{itemize}
	\item Completezza: Tutti i requisiti devono essere rappresentati nello schema
	\item Pertinenza: Nello schema devono esserci solo concetti che compaiono nei 
	requisiti
	\item Minimalità: Non ci devono essere più concetti che rappresentano lo stesso
	requisito.
	\paragraph{Specificazione:} In taluni casi può anche essere accettata la non
	minimalità (o anche ridondanza). L'importante è che sia riconosciuta, in modo 
	tale che a lovello logico sia possibile decidere se mantenere o eliminare la
	ridondanza.
	\item Leggibilità: Lo schema deve rappresentare i requisiti in modo comprensibile 
	\begin{itemize}
		\item Grafica: Leggibilità estetica, nel senso che non ci devono essere
		incroci tra linee, e le line NON devono essere oblique, solo verticali
		od orizzontali
		\item Concettuale: Sarebbe la scelta di strutture del modello che danno
		luogo ad uno schema più compatto, più semplice da capire.
	\end{itemize}
\end{itemize}
\paragraph{Come si raccolgono i requisiti?} L'analista non ha presente come funziona
il dominio d'applicazione, c'è bisogno di individuare diverse fonti informative,
a partire dalle quali raccogliere i requisiti
\subparagraph{Da dove si attingono le informazioni?} 
\begin{itemize}
		\item Dagli utenti tramite interviste o documentazione fornita
		\paragraph{Problemi legati alle interviste:} Utenti diversi possono 
		fornire informazioni diverse e usare termini diversi per indicare la stessa cosa.
		\item Tramite modulistica usata per acquisire i dati
		\paragraph{Ok, e che documentazione esiste?} 
		\begin{itemize}
			\item Normative (Leggi, regolamenti di settore)
			\item Regolamenti interni, procedure aziendali
			\item Ralizzazioni preesistenti (I vecchi Db che l'azienda aveva, 
			per intenderci)
		\end{itemize}
\end{itemize}
\subsection{Regole generali per la documentazione descrittiva}
Per cominciare bisogna scegliere un corretto livello di astrazione, del tipo
è meglio usare "animale" o "cane"? Il concetto è capire quanto salire in alto come
livello di astrazione.
E' utile anche
\begin{itemize}
	\item Standardizzare la struttura dell efrasi
	\item Separare le frasi sui dati da quelle sulle operazioni che aggiornano i 
	dati
\end{itemize}
\subsection{Organizzazione di termini e concetti}
In generale si usa:
\begin{itemize}
	\item Costruire un glossario dei termini
	\item Individuare omonimi e sinonimi e fare un'unificazione dei termini
	\item Rendere esplicito il riferimento tra termini
	\item Riorganizzare le frasi per concetti
\end{itemize}
C'è da fare una distinzione tra strategie semplici (Quelle che danno indicazioni
per scegliere tra attributo, entità, relazione, relazione is\_a e generalizzazione),
oppure complesse (Quando si esprime un'idea complessiva su come percorrere il processo di 
progettazione)
\section{Strategie semplici}
Sono le strategie viste finora per quanto riguardi le lezioni sul modello E-R
\begin{itemize}
	\item Prima le entità, e attributi di entità
	\item Poi le relazioni e attributi delle relazioni
	\item Generalizzazioni
\end{itemize}
E per la scelta dei singoli concetti
\subsection{Strategie di progetto}
Le solite:
\begin{itemize}
	\item Top-down
	\item Bottom-up
	\item Inside-out: Da un concetto "a macchia d'olio" verso tutti gli atri
	\item Mista tra tutte queste tre
\end{itemize}
\section{Strategia Top-down}
Nel top down si possono disciplinare i \underline{raffinamenti}, che sono 
sostituzioni di concetti con altri concetti più dettagliati, usando un numero
limitato di \color{red}\textbf{primitive di raffinamento}\color{black}
\paragraph{Le principali sono:}
\begin{enumerate}
	\item Entità $\to$ Entità + Attributi
	\item Entità $\to$ Relazione tra Entità
	\item Entità $\to$ Generalizzazione tra Entità
\end{enumerate}
\section{Strategia Bottom-up}
Parti da tanti piccoli sottoschemi fatti di concetti, che vai a unire formando
uno schema completo, ci sono casi in cui si può attuare una strategia bottom-up
o una top-down indifferentemente.
\paragraph{Vantaggi e svantaggi dei vari modelli:}
\begin{center}
\includegraphics[width=0.75\textwidth]{8}
\end{center}
\begin{center}
\includegraphics[width=0.75\textwidth]{9}
\end{center}
A questo punto risulta scontato che la soluzione migliore sia la \textbf{mista}
per via del fatto che prende i lati positivi di ognuna delle strategie esistenti.
\section{Schema scheletro}
Lo schema scheletro raccoglie i concetti più importanti, che vngono organizzati
in uno schema concettuale. 
Ci sono due main strategie miste, che si differenziano su come si procede dopo
aver individuato lo schema scheletro:
\begin{enumerate}
	\item La prima è tutta top-down: definisce in maggior dettaglio la strategia
	top-down
	\item La seconda è mista, nel senso che prevede passi ispirati alle altre
	strategie
\end{enumerate}
\subsection{Strategia Top-down completa}
\paragraph{Analisi dei requisiti:}
\begin{enumerate}
	\item Si analizzano i requisiti e si eliminano le ambiguità
	\item Si raggruppano i requisiti in insiemi omogenei
\end{enumerate}
\paragraph{Passo base:}
\begin{enumerate}
	\item Si definisce uno schema scheletro con i concetti più rilevanti
\end{enumerate}
\paragraph{Passo iterativo:} Si ripete finchè non sono stati rappresentati tutti
i requisiti
\begin{enumerate}
	\item Si raffinano i cncetti presenti sulla base dell eloro specifiche
	\item Si aggiungono concetti per descrivere specifiche non descritte
	\item Si verifica la qualità dello schema e eventualmente lo si modifica per 
	raggiunggere le qualità desiderate
\end{enumerate}
\subsection{Metodologia mista}
\paragraph{Analisi dei requisiti:}
\begin{enumerate}
	\item Si crea lo schema scheletro
\end{enumerate}
\paragraph{Decomposizione:}
\begin{enumerate}
	\item Si decompongono i requisiti con riferimenti ai concetti nello schema 
	scheletro
	\item Si creano i diversi sottoschemi
\end{enumerate}
\paragraph{Passo iterativo:} 
\begin{enumerate}
	\item ìSi integrano i vari sottoschemi in uno schema complessivo facendo 
	riferimento allo schema scheletro.
\end{enumerate}
\paragraph{Analisi di quaità}
\begin{center}
\includegraphics[width=0.75\textwidth]{10}
\end{center}
\chapter{Progettazione logica}
La progettazione logica si pone l'obbiettivo di tradurre lo schema concettuale
in uno schema logico che sia in grado di rappresentare lo schema per mezzo
di un \textbf{modello logico}. 
\subsection{Efficienza e correttezza}
Prima di tutto, lo schema nel modello relazionale deve rappresentare la
stessa realtà dello schema E-R, e soprattutto le varie query e transizioni
devono essere eseguite sullo schema con un costo di risorse ridotto al minimo.
\paragraph{Precisazioni sulla correttezza:}
Come già detto, lo schema ER e il relazionale devono rappresentare la stessa
realtà, devono avere un \textbf{contenuto informativo} equivalente. Ci sono però
alcune strutture che non sono direttamente rappresentabili:
\begin{enumerate}
	\item Le generalizzazioni
	\item Le associazioni (in particolari le molti a molti)
	\item Gli attributi multivalore
	\item Le chiavi composte
\end{enumerate}
\paragraph{Precisazioni sull'efficienza:} La riduzione del carico applicativo
comprende \textbf{interrogazioni} (query) e \textbf{transizioni}, banalmente
operazioni che tirano fuori informazioni, e operazioni che ci scrivono roba.
\subparagraph{Fissato il carico applicativo:} dobbiamo scegliere quel particolare 
schema logico che permette di ottimizzare l'esecuzione del carico applicativo.
\section{Fasi della progettazione logica}
\begin{center}
\includegraphics[width=0.75\textwidth]{11}
\end{center}
E qua comincia ciò che maggiormente ho odiato nel mio 5 anno delle superiori:
\subsection{Ristrutturazione dello schema ER}
\paragraph{La domanda è: Perchè?} Perchè bisogna rendere semplice la
traduzione nel modello relazionale, e quindi prendete lo schema ER e ne 
create un altro equivalente (Quindi, che cazzo abbiam creato a fare
il primo in principio? Bella domanda). 
Ah sì poi un altro motivo è l'ottimizzazione del carico applicativo.\\ \\
Bisogna concentrarsi sul tempo di esecuzioni delle query e sullo spazio in
memoria che l'esecuzione di una query occupa. 
\paragraph{Esempio che mi è capitato personalmente:} Prendete due tabellone 
gigantescone con dei datoni gigantesconi dentro..ni. Bene, dobbiamo andare a 
selezionare UN campo di UN record che però si ottieme facendo una join. Ecco, ogni
volta crashava tutto perchè questo caricava in memoria T U T T A la tabella
del prodotto cartesiano. 
\paragraph{Che risorse si hanno a disposizione?} \textbf{Spazio}: Tavola dei volumi, che
per intenderci descrive il numero di istanze delle entità e delle relazioni, e
\textbf{Tempo}: Tavola degli accessi, che descrive per ogni operazione rilevante, 
il numero di istanze di entità e relazioni visitate dalla operazione.
\section{Tavola dei volumi}
La tavola dei volumi serve per capire quanto spazio occuperà una base di dati: 
è una tabella gigante in cui si ha per ogni entità o relazione il numero orientativo
di istanze nel corso della vita del DB. La tabella sarà formata da 3 colonne x N
righe, dove N dipende da numero di entità+numero relazioni.
\begin{center}
\begin{tabular}{ |c|c|c| } 
 \hline
 Nome & Tipo & Volume \\
 \hline
 Entità 1 & E & $\alpha$ \\ 
 Entità 2 & E & $\beta$ \\ 
 Relazione & R & $\gamma$ \\ 
 \hline
\end{tabular}
\end{center}
Prima di tutto si inserisce il numero di istanze delle entità, successivamente,
si va a inserire per ogni relazione il numero, che viene generato dai valori
delle altre entità
\section{Tavole degli accessi}
Se quelle precedenti misurano lo spazio indicativo occupato, in questo caso si 
va a parlare di tempo speso.
\begin{itemize}
	\item In pratica descrive il costo di utilizzo da parte di un'operazione
	(inteso come tempo)
	\item Si misura con il numero di istanze dell'entità e relazioni visitate 
	dall'operazione
	\item Si tratta pur sempre di un'approssimazione, non ragionando su dati
	oggettivi, ma su ipotesi (per quanto plausibili possano essere)
\end{itemize}
\paragraph{Come si fa?}
\begin{enumerate}
	\item Input: Operazione con sua frequenza
		\begin{enumerate}
			\item Si costruisce lo \underline{Schema di navigazione}
			\paragraph{Lo schema di navigazine:} E' un ordine espresso 
			graficamente, in cui l'operazione visita in sequenza entità e relazioni
			dello schema
			\item Costruisci la tavola degli accessi (Basata sull'ipotesi
			che venga seguito lo schema del passo 1)
			\begin{center}
			\includegraphics[width=0.75\textwidth]{12}
			\end{center}
			\item Nota la frequenza dell'operazione, si calcola il \underline{
			costo dell'operazione
			}, che si definisce come accessi per quella frequenza

		\end{enumerate}
\end{enumerate}
\paragraph{Avvertenze:} accanto agli accessi in lettura, le transazioni effettuano
anche accessi in scrittura poichè modificano lo stato del DB.
E negli access iin scrittura si considera per ogni accesso un doppio trasferimento,
uno da memoria secondaria a principale e uno che fa l'inverso.
\section{Le fasi della ristrutturazione}
\begin{enumerate}
	\item Analisi delle ridondanze (Alcune potrebbe aver senso lasciarle)
	\paragraph{Cioè?} Una ridondanza è un'entità o relazione che si può derivare
	da altri. Derivabile si intende che ha dei valori che sono calcolabili 
	(sia da attributi dell'entità stessa, che da altributi di altre entità).
	\paragraph{Nel caso delle relazioni:} Si derivano dalla composizione di 
	altre relazioni in presenza di cicli di entità e relazioni.
	\paragraph{Ok, perchè mantenere le ridondanze?} Si semplificano le query ogni
	volta che compare la struttura ridondante
	\paragraph{Perchè no?} Perchè vai ad aggiungere operazioni di aggiornamento e
	perchè serve più spazio
	\item Eliminazione delle generalizzazioni (Perchè non esistono nel relazionale)
	\item Partizionamento o accorpamento di entità e relazioni per rendere
	efficiente lo schema relazionale finale
	\item Scegliere le chiavi primarie
\end{enumerate}



























\chapter{Lezione WebEx 1 (credo)}
\section{Differenze tra Algebra Relazionale e SQL}
\begin{enumerate}
	\item L'algebra relazionale è un linguaggio formale, l'SQL si usa quando si 
	lavora sui DB
	\item L'algebra relazionale è procedurale, si specifica l'algoritmo, mentre
	SQL invece è parzialmente dichiarativo, specifica solo il risultato da 
	ottenere, senza specifica degli algoritmi
	\item Istruzioni equivalenti differiscono per efficienza, mentre in SQL è 
	solo un questione di leggibilità
	\item Nell'algebra relazionale le relazioni sono matematiche, in SQL sono 
	tabelle
	\item Negli insiemi non ci possono essere elementi uguali, mentre in SQL sì,
	possono esserci righe uguali
\end{enumerate}
\end{document}
