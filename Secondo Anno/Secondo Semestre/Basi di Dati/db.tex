\documentclass[12pt, a4paper, openany, twoside]{book}
\usepackage[italian]{babel}
\usepackage[T1]{fontenc}
\usepackage[utf8]{inputenc}
\usepackage{amsmath} 
\usepackage{xcolor}
\usepackage[margin=1in]{geometry}
\usepackage{hyperref}
\usepackage{graphicx}
\graphicspath{{./img/}}
\usepackage{tikz}
\hypersetup{
    colorlinks=true,
    linkcolor=blue,
    filecolor=magenta,      
    urlcolor=cyan,
}
%usepackage[latin1]{inputenc}
\begin{document}
\fontfamily{cmss}\selectfont
\pagestyle{plain}
\author{DaveRhapsody}
\title Basi di Dati
\date 4 Marzo 2020
\maketitle
\tableofcontents
\chapter*{Introduzione}
Un database, o base di dati, o db (che scriverò d'ora in poi) è un insieme di 
dati, tipo deposito, per qualsiasi genere di uso, sia aziendale, che personale.
I suddetti dati sono inseriti, letti, e soprattutto \underline{organizzati} 
secondo certe regole.
\paragraph{Alcuni esempi: }
\begin{itemize}
	\item Agende telefoniche
	\item Studenti di una classe o scuola
	\item Qualsiasi insieme generico in cui ogni elemento differisce da un altro
	secondo delle linee guide (campi) che si decidono alla base.
\end{itemize}
Da Linguaggi di Programmazione sono state viste le Struct, o Record, che sono
delle strutture per i dati statiche, concrete ed eterogenee, aventi più campi
non necessariamente dello stesso tipo. \\ \\
Bene, un DB è un array di record, in cui bisogna garantire integrita, consistenza
e NON ridondanza dei dati.
\section{Interazione tra campi}
Tra loro i campi di questi record possono interagire, nel senso che partendo dal
valore di un determinato campo, si può ricavare il valore di un altro campo.
Esempio? Il \underline{codice fiscale}, che è ricavato da una formula che non 
ricordo mai nella vita forever MA prendendo come dati il nostro nome, cognome, 
etc.
\paragraph{Questo sarà un campo calcolato}
\section{Dove si colloca un DB?}
\begin{itemize}
	\item Intefaccia utente
	\item Applicativi
	\item Software di ambiente e di sistema
	\item DB
	\item Sistema operativo
	\item Hardware
	\item Sistema di comunicazione di rete
\end{itemize}
Impropriamente si può definire nel mezzo
\section{Problemi da NON avere}
\begin{itemize}
	\item Ridondanza dei dati: non devono esserci dati ripetuti, ogni record è
	U N I V O C O. Per renderlo tale credo nel corso che vedremo come si fa, 
	spoiler: chiavi, la nostra futura bacinella di bestemmie.
	\item Rischi di incoerenza: i dati devono essere consistenti, ossia dato un
	valore, se lo si attribuisce ad un simbolo (dato, variabile, campo), quel
	valore dovrà essere SEMPRE quello, per ogni volta che si richiama quel 
	simbolo
\end{itemize}
\section{Condivisione dei dati}
Data un'organizzazione avente più dipendenti, è naturale che la suddetta possa
condividere un determinato insieme di dati, infatti ad ogni settore corrisponde
un sistema informativo (Per chi ha fatto economia, il SIA).\\ \\
Cosa accade quando si condivide una risorsa? Esatto, bisogna fare in modo che
non avvengano accessi concorrenti, quindi sono implementate funzioni e 
procedure di prevenzione di questo genere di problemi. Un DB non protetto da
modifiche NON consentite, oltre a fare schifo tipo fortissimo forever maonna guarda
da bruciarlo, è NON integro. 
\paragraph{Un DB deve essere integro}
\end{document}