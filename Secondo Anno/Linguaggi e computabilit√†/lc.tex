\documentclass[12pt, a4paper, openany, oneside]{book}
\usepackage[italian]{babel}
\usepackage[T1]{fontenc}
\usepackage[utf8]{inputenc}
\usepackage{amsmath} 
\usepackage{xcolor}
\usepackage[margin=1in]{geometry}
\usepackage{hyperref}
\hypersetup{
colorlinks=true,
linkcolor=blue,
filecolor=magenta,      
urlcolor=cyan,
}
%usepackage[latin1]{inputenc}
\begin{document}
\pagestyle{plain}
\author{DaveRhapsody}
\title{Linguaggi e Computabilità}
\date{2 Ottobre 2019}
\maketitle
\tableofcontents
\chapter{L'esame}
Avremo due compitini, uno a novembre ed uno a Gennaio, in un anno sono 
disponibili 5 appelli, se uno è del terzo anno, può fare i compitini, basta che
ci sia spazio nelle aule, la precedenza va a coloro che sono del secondo anno.
\\ \\
Al secondo appello (Quello di Febbraio) puoi recuperare il voto negativo di uno
dei due compitini. Non presentarsi è esattamente come provarci e non passare, 
quindi rischiate, conviene.
\\ \\ 
L'orale va sostenuto nello stesso appello dello scritto, cioè io faccio lo 
scritto, lo passo, l'orale lo devo fare in quella sessione.
Per chi fa i compitini ed ha consegnato anche gli esercizi di lab. può fare un
orale prima del 5 Febbraio OPPURE si può fare assieme a coloro che hanno fatto
l'esame il 5.
\\ \\
Gli esercizi valgono dal momento che li invii fino a fine anno, quindi ha senso
farli subito tutti
\\
{\color{black} \rule{\linewidth}{0.3mm} }
\\
\chapter{Linguaggi formali}
Nascono per essere in grado di creare i linguaggi di programmazione, o meglio
servono per gestire i protocolli di comunicazione e la possibilità di comunicare
una determinata operazione al calcolatore.
\section{Backus-naur form (Backus Normal Form)}
\paragraph{Definizione}Da 
\href{https://it.wikipedia.org/wiki/Backus-Naur_Form}{Wikipedia}:
è una metasintassi, ovvero un formalismo attraverso cui è possibile descrivere 
la sintassi di linguaggi formali (il prefisso meta ha proprio a che vedere con 
la natura circolare di questa definizione). Si tratta di uno strumento molto 
usato per descrivere in modo preciso e non ambiguo la sintassi dei linguaggi di
programmazione, dei protocolli di rete e così via, benché non manchino in 
letteratura esempi di sue applicazioni a contesti anche non informatici e 
addirittura non tecnologici. La BNF viene usata nella maggior parte dei testi 
sulla teoria dei linguaggi di programmazione e in molti testi introduttivi su
specifici linguaggi.  

\section{Model checking}
Usato per protocolli di comunicazione, per esempio per protocolli di pagamento,
in realtà di qualsiasi tipo, chiaramente per la sicurezza questo è l'ideale, 
perchè si descrive lo stato di sistema, e si specifica se ogni stato è sicuro
(Sicuro sia dal punto di vista dei risultati corretti che sicuri)
\\ \\
E' usato anche per il software, cioè in maniera automatica deduce in base alle
condizioni di ingresso, se son corrette. Ce la fa? Si per programmi piccini, 
ma alla fine, ma ingenerale, non esiste una tecnica che preso un software ti 
dimostra che esso sia corretto in ogni caso. Non esiste nessuna procedura generale,
se esistesse ci sarebbero contraddizioni logiche.
\paragraph{Cos'è una contraddizione logica?} 
E' un paradosso, ma a livello un po' più infame, pensate alla frase "Questa 
frase è vera", se ci scavate a fondo, dopo un po' diventa una contraddizione.
\section{Automi a stati finiti}
Sono insiemi di stati ai quali arrivan dall'esterno dei dati, ed a seconda dello
stato in cui si trovano, e del dato che arriva, allora potrebbero verificarsi le
famose "Transizioni" che consistono nel cambiare stato.
\\ \\ 
La memoria del Latch SR, ad esempio, funziona come un automa, nel senso, varia
a seconda dello stato interno, e del valore di ingresso.
\\ \\
\paragraph{Linguaggio Per} E' uno dei primi linguaggi di scripting, anche se ce 
n'era qualcun altro prima, e contiene istruzioni per gestire espressioni regolari
che possono essere applicate su testi lunghi per fare ricerche.\\ 
In pratia prendevano delle sequenze di DNA (tera di dati), e venivano analizzati
(con espressioni regolari) da questo linguaggio.
\chapter{Alfabeto}
E' un insieme finito e non vuoto di simboli, ad esempio:
\{A, B, C, D, ... , Z\} , \{1, 2, 3, 4, ..., 9\}. \\ 
Per gli alfabeti useremo lettere greche tipo: $\Sigma, \Lambda, \Gamma$, vediamo
alcune definizioni ora:
\paragraph{Stringa} La stringa è una sequenza di simboli, se è 
vuota si definisce vuota, può esistere. \\ Data una stringa w, si indica la sua
lunghezza con |w|. Per esempio: |acdas234| = 8, mentre se ho |$\epsilon$| = 0,
poichè si indica che una stringa è vuota dicendo che essa abbia solo una lettera
greca dentro
\paragraph{Concatenazione tra stringhe}
La concatenazione fa in modo che date due stringhe w, x l'ultimo carattere di x
sarà il successivo dell'ultimo di w. pertanto, w, x $\to$ w $\circ$ x = wx \\
Per esempio se ho una stringa vuota, e la concateno ad una stringa, otterrò la
stringa (3+0 fa 0, no? :)) $\to \epsilon \circ $ w = w \\ \\
Chiaramente si vanno a sommare le lunghezze delle due stringhe in ogni caso. 
\subparagraph{NON Commutatività di una stringa}
Concatenare due stringhe non è sempre possibile, a meno che siano perfettamente
identiche
\paragraph{Potenze di un alfabeto} 
Prendiamo un alfabeto $\Sigma$ e per un k intero >= 0 $\Sigma ^{k} = \Sigma x, 
\Sigma x, \Sigma x, \Sigma x$, ottengo una permutazione di k volte $\Sigma$, tutte
appartenenti a $\Sigma ^{k}$ \\ \\
Come sarà la sua cardinalità? |$\Sigma$| = q $\to$  |$\Sigma ^{k}| - q^{k}$.
\\ \\ 
Per k = 1 avrei $\Sigma ^{1}$ w = qualsiasi elemento di $\Sigma$(un solo elemento) 
\\ \\
Se ho $\Sigma$ = {0, 1} \\
$\Sigma ^{2}$ = Tutte le permutazioni che posso fare con 0, 1 \textbf{i lunghezza 2}
(I valori di $\Sigma$)
\\ \\
Per definizione $\Sigma ^{0}$ = {$\epsilon$}, \\
\subparagraph{Attenzione} Quello che è contenuto in $\Sigma$ è un insieme di STRINGHE
non caratteri o simboli. 
\paragraph{Chiusura di Kleene}
$\Sigma ^{*}$ = U per k $\geq$ 0 $\Sigma ^{*}$ = U per k $\in$ {0 ad $\infty$}
di $\Sigma ^{k}$
\\ \\
$\Sigma ^{+}$ = $\Sigma ^{*} - \Sigma ^{0}$, invece $\Sigma ^{*}$ è considerabile
come $\Sigma ^{+} \cup \Sigma ^{0}$
\paragraph{ATTENZIONE} La L che userò nei prossimi passaggi ($\to L$) è una MACCHINA 
AUTONOMA che verifica la stringa in questione
\paragraph{Linguaggio L su $\Sigma$}
E' un sotto insieme di $\Sigma ^{*}$, o meglio L  $\subseteq \Sigma ^{*}$ 
Ad esempio:
$$\Sigma = {a, b, c} \to L_{1}={aa, cbc}  \subseteq \Sigma ^{*} \\
L_{2} = {w \in {a,b,c}^{*} t.c. ~ W ~ contiene ~ stesso ~ numero~di~a~e~c} $$
In pratica
$$L_{2} = {ac, ca, acb, abc, cab, cba, ...}\\
abc \in L_{2}?? Yes\\
ccbb \in L_{2}?? Nope$$ 

\paragraph{Detto meglio} (Problema di Membership) \\ $W \in \Sigma ^{*} \to L 
\begin{cases}
SI, ~ Se ~ w \in L \\
NO, ~ Se ~ W \in \Sigma ^{*} senza ~ L (Complemento~di~L)
\end{cases}$ \\ \\
Attenzione, il linguaggio è un insieme, contiene quindi ELEMENTI, e di 
conseguenza può contenere anche l'insieme vuoto! \\ \\
\paragraph{Osservazione:} w può essere appartente a $\Sigma ^{*}$ MA non 
all'insieme vuoto. Occhio a non confondersi
\\ \\
In generale un linguaggio formale va studiato secondo due punti di vista almeno.
\\ \\
Avendo un linguaggio L  $\subseteq \Sigma ^{*} posso 
\begin{cases}
generarlo ~(grammatica)\\
riconoscerlo ~(macchina~autonoma)
\end{cases}$
\paragraph{Grammatica} Insieme di regole che specificano come va fatta una stringa
Una grammatica G è una quadrupla $\to$ G = (V, T, P, S) ~in ~cui~
\begin{itemize}
	\item V: variabili
	\item T: Simboli terminali
	\item P: insieme delle produzioni
	\item $S \in V$: simbolo di start
\end{itemize}

\paragraph{I tipi di grammatiche}
Esiste una gerarchia (Noam) Chomsky, che negli anni 50 si poneva domande su 
cosa accade nel cervello umano quando si elabora un linguaggio. \\ \\
La sua ipotesi (smentita) c'è una sorta di grammatica codificata/cablata
per elaborare il linguaggio, e (malgrado smentita) è saltata comunque fuori 
questa gerarchia
\begin{enumerate}
	\item Grammatiche tipo 0: , i 	
	\begin{itemize}
		\item Non hanno restrizioni sulle produzioni
		\item Sono riconosciuti dalle macchine di Turing (Alan Turing)
		\item linguaggi che generano sono i ricorsivamente numerati, li vedremo a 
		computabilità (Sia deterministiche che non)
	\end{itemize}
	\item Grammatiche Tipo 1: La testa ha lunghezza $\geq$ corpo, ne vedremo 
	solo due esempi
	\begin{itemize}
		\item Linguaggi dipendenti dal contesto, riconosciuti da macchine 
		particolari come la macchina di Turing, che lavorano spazio lineare
		Cioè Se n è la lunghezza della prima forma sentenziale da cui parto, 
		tutte le altre forme sentenziali non potranno essere più lunghe, e quindi
		non può crescere il numero di simboli, tenderà sempre a diminuire.
	\end{itemize}
	Le regole di produzione di tipo 1:
	$\alpha _{1} ~ A ~ \alpha _{2} ~ \to ~ \alpha _{1}~ \beta ~ \alpha_{2} ~ con ~ \alpha _{1,2}
	,\beta ~ \in ~ (V ~ \cup ~ T)^{*} ~ , ~ \beta ~ \neq ~ \epsilon, A\in V$
	\paragraph{Problema di decisione} E' un problema la cui risposta possibile è
	sì o no (Cioè alla fine true o false). Risolvere un problema di decisione non 
	pè altro che risolvere un problema di membership.
	\item Grammatiche Tipo 2: 
	Le regole di produzione qui sono del tipo A$\in$B, con A$\in$V e 
	$\beta \in(V \cup T)^{*}$
	Sono linguaggi context free, e vengono riconosciuti da macchine (o automi) a
	pila monoterministica
	\item Grammatiche Tipo 3: Sono le grammatiche regolari e quindi producono e
	generano semplicemente linguaggi regolari, e le produzioni delle grammatiche
	regolari si possono tutte trasformare in modo tale che A$\in$ aB $\wedge$ A
	$\in$ a con A, B$\in$V e a$\in$ T, riconosciuti da automi a stati finiti, 
	deterministici o monodeterministici
\end{enumerate}
Il complemento di un linguaggio può essere sia infinito che finito (Nel senso
posso escludere elementi oppure posso considerare solo quelli!)
\\ \\ 
\section{Linguaggio context-free (CFL) legati a grammatiche Context Free (CFG)}
In questo caso si utilizza una forma ricorsiva per definire questi linguaggi,
\\ \\
Ricordiamoci del fatto che io posso mettere due linguaggi in serie, posso 
includerne uno in un altro MA non posso accavallarli. Nel senso, o tutto di 
entrambi, o niente. Ma torniamo ai \textbf{Context free}
\\ \\
\paragraph{La stringa palindroma}
Sono stringhe la cui lettura è identica in qualsiasi verso si leggano. Supponiamo
$\Sigma$ = {0, 1} es. $L_{pal} \subseteq \Sigma ^{*} \to "0110", "11011", 
\epsilon$, perchè la stringa vuota è considerata palindroma. \\ 
Più in modo formale si può dore che w è \textbf{palindroma} quando $w = w^{R}$
Definizione induttiva:
$$
\begin{cases}
base:~\epsilon, 0, 1 \in L_{pal} \\
passo~induttivo:~ se ~w\in ~L_{pal} ~allora ~OwO, ~1w1 ~\in ~L_{pal}
\end{cases}
$$
\\
$$
S \in \epsilon\\
S ~ \in ~ 0 \\
S ~ \in ~ 1 \\
S ~ \in ~ 0S0 \\
S ~ \in ~ 1S1$$
Con S che è una variabile (categoria sintattica), e \{0, 1\} che sono i simboli
terminali con cui si scrivono le stringhe del linguaggio.\\ \\
Queste si chiamano regole di produzione in cui la testa è occupata in questo
caso dalla freccia, mentre i vari 0 1 0S0 e 1S1 sono il corpo. S PUO' diventare
il corpo \\ \\ 
Detto questo possiamo dire che
$$G_{pal} = (V, T, P, S), ~in ~cui $$\\
\begin{itemize}
	\item V = \{S\}
	\item T = \{0, 1\}
	\item P = $S \in \epsilon, S \in 0, S \in 1, S \in 0S0, S \in 1S1$
\end{itemize}

Più in generale
$$G_{pal} = (\{S\},\{0, 1\}, P, S) dove P = \{S \in \epsilon, ...\}$$
\paragraph{Derivazione}
$S \implies 1S1 \implies 10S01$, dove 1S1 è una forma sentenziale e la S cambia in
funzione delle regole che ho deciso sopra (per generare la stringa ovviamente.)
\\ \\ In modo compatto: \\
$S \in \epsilon | 0 | 1 | 0S0 | 1S1$
\\ \\
C'è una precisazione da fare, se per esempio avessi la regola che le mie stringhe
debbano iniziare per 0, quando andrò a fare 0S0, allora quell'S volendo può
essere sostituita con una \textbf{nuova} variabile che chiamiamo \textbf{X}. \\ \\
\textbf{X} non è altro che una variabile che rappresenta l'insieme di tutte le
palindrome.
Perchè cambiare variabile? Perchè se io voglio ad esempio le palindrome che 
iniziano per 0, devo avere, dato che non posso forzare l'ordine con cui vengono
applicate le mie regolo, devo avere un "permesso" speciale che consenta di 
generare 0 all'inizio alla fine. Cioè, dentro ci può essere un pandemonio, ma 
fuori ci deve essere la regola che stabilisce l'esistenza di 0.
\section{Parentesi bilanciate}
T =\{(,)\}, in cui () $\in$ $L_{pal}$, (()) $\in$ $L_{pal}$, ()() $\in$ $L_{pal}$,
$\epsilon \in L_{pal}$ \\ \\
Se W $\in L_{pal}$, allora (W) $\in L_{pal}$ esattamente come WW$\in L_{pal}$
\paragraph{Ex 5.19 p 182}
$\epsilon$ ()  ()() (w) $\L_{pal}$
$$\begin{cases}
	\underline{base}: \epsilon	\\
	\underline{Passo}: Se~w\in L_{pal} ~ allora ~ ww\in L_{pal} 
	 ~AND~ (w)\in L_{pal} 
\end{cases}$$
Dato G=(V, T, P, B) \\
B $\to$ BB | (B) | $\epsilon$, \\ \\
V = \{B\} T = \{(, )\} e \\
P = \{B $\to$ BB, B $\to$ (B), B$\to \epsilon$\} \\ \\
A questo punto, se avessi ()(()) otterrei: \\
B $\implies _{1}$ BB $\implies _{2}$ B(B) $\implies _{2}$ (B)(B) $\implies _{3}$
()(B) $\implies _{2}$ ()((B)) $\implies _{3}$ ()(()) \\
Quindi questa stringa è possibile generarla.
\section{Produzioni Context - Free}
A $\to$ $\gamma$ dove A $\in$ V e $\gamma \in (V\vee T)^{*}$ \\ \\
Agendo come prima:	\\
B $\implies _{1}$ BB $\implies _{2}$ (B)B $\implies _{3}$ ()B $\implies _{2}$ 
()(B) $\implies _{2}$ ()((B))$\implies _{3}$ ()(())
\section{Derivazione left/right most}
Per evidenziare che sia una letf o right most invece del numeretto, alla freccia
si aggiunge un lm o rm (Left o Right most). Sempre con l'esempio di prima -> \\ \\
B $\implies _{rm}$ BB $\implies _{rm}$ B(B) $\implies _{rm}$ B((B)) 
$\implies _{rm}$ B(()) $\implies _{rm}$ (B)(()) $\implies _{rm}$ ()(())
\paragraph{Ex 5.3, p 162}
Regole di produzione:
E $\in$ I | E + E | E * E | (E) \\ \\
I $\in$ a | b | Ia | Ib | I0 | I1 \\ \\
G = (V, T, P, E)      V = \{E, I\} \\
T = \{+, *, (, ), a, b, 0, 1\} \\ \\ \\
La E diventa identificatore quindi $E \implies I \implies a$ (stessa roba per b)
quindi $E \implies I \implies Ia \implies I0a \implies Ib0a \implies I1b0a 
\implies b1b0a$ 
%\implies I1b0a  \implies b1b0a \implies D10b01a$
\\ \\
Proviamo a generare a * (a+b00) (metodo Left-Most) \\
E $\implies _{lm}$ E * E $\implies _{lm}$ I * E $\implies _{lm}$ a * E
$\implies _{lm}$ A * (E) $\implies _{lm}$ a * (E + E) $\implies _{lm}$ a * 
(I + E) $\implies _{lm}$ a * (a + E) $\implies _{lm}$ a * (a + I) $\implies _{lm}$
a * (a + I0) $\implies _{lm}$ a * (a + I00) $\implies _{lm}$ a *(a + b00) \\ \\
Per parcondicio, ora faremo anche la generazione con il Right Most \\
E $\implies _{rm}$ E * E $\implies _{rm}$ E * (E) $\implies _{rm}$ E * (E + E)
$\implies _{rm}$ E (E + I) $\implies _{rm}$ E * (E + I0) $\implies _{rm}$ 
E * (E + I00) $\implies _{rm}$ E * (E + b00) $\implies _{rm}$ E * (I + b00)
$\implies _{rm}$ E * (a + b00) $\implies _{rm}$ I * (a + b00) $\implies _{rm}$ 
a * (a + b00) \\ \\
Per indicare che "in qualche modo" è possibile ottenere una determinata stringa
si scrive $$E \implies _{rm/lm}^{*}$$.
\\ \\ 
Data $\alpha A \beta, ~con~ \alpha \beta \in (V U T)^{*}, ~con~ A\in V$ \\
$A \in \gamma$ (Regole di produzione) \\
$\alpha A \beta \implies \alpha \gamma \beta$ 
\paragraph{Se A è var. più a sx} $\implies _{lm}$ altrimenti diventa 
$\implies _{rm}$ Nel caso sia più a destra
\\ \\
\section{Definizione di $\implies ^{*}$}
Per induzione: \\ \\
$$\begin{cases}
	Base: ~\forall \alpha \in (V U T)^{*}, \alpha \implies ^{*} \alpha
	Passo: ~Se~ \alpha \implies ^{*} \beta e \beta \implies \gamma \\
			allora~ \alpha \implies ^{*} \gamma ~ dove ~ \alpha, \beta, \gamma 
			\in (V U T)*
\end{cases}$$
Pertanto $\alpha \implies ^{*} \beta$ sse $\exists \gamma _{1}, \gamma _{2}, 
..., \gamma _{n} \in (VUT)^{*}$ con n$\geq$ t.c
$\alpha = \gamma _{1}, \beta = \gamma _{n} e \forall i = 1, 2, ..., n-1$ si ha che
$\gamma _{i} \implies \gamma _{i+1}$ 
\section{Definizione forma sentenziale}
Sia G = (V, T, P, S) una CFG, e $\alpha \in (VUT)^{*}$ t.c. S $\implies ^{*} \alpha$ \\ \\
Ogni volta che io genero nella forma sentenziale uno zero, in realtà se ne 
genera un altro, quindi se ho $S \implies 0S0 \implies 00S00$, imponendo un vincolo
sugli zeri prima e dopo la S, quindi per esempio se avessi: \\ \\
I $\to$ 0 | 1 | $\epsilon$ | I0 | I1 Reg. \\
$I \implies I0 \implies I00 \implies I000 \implies I1000 $
\section{Inferenza Ricorsiva}
L'obbiettivo è dimostrare che dato un obbiettivo si può ricavare in 0 o più passi
una determinata stringa. \\ (E' la stringa dell'esercizio precedente: )
Si agisce ponendo una tabella composta in questo modo 


\begin{center}
\begin{tabular}{ |c|c|c|c|c| } 
 \hline
 / & Stringa ricavata & Variabile & N proof & Stringhe impiegate\\
 \hline
 (1) & a & I & 5 & - \\
 (2) & b & I & 6 & -  \\
 (3) & b0 & I & 9 & (2)\\
 (4) & b00 & I & 9 & (3)\\
 (5) & a & E & 1 & (1)\\
 (6) & b00 & E & 1 & (6)\\
 (7) & a+b00 & E & 2 & (5),(6)\\
 (8) & (a+b00) & E & 4 & (7)\\
 (9) & a*(a+b00) & E & 3 & (5),(8)\\
 \hline
\end{tabular}
\end{center}
\section{Due teoremi importanti}
\subsection{Th. 1}
Sia G = (V, T, P, S) una CFG, e sia $\alpha \in (VUT)^{*} allora 
S \implies ^{*}\alpha ~sse~ \alpha ~ e' ~ ottenibile ~ tramite ~ procedura ~ di  ~
inferenza ~ ricorsiva$
\subsection{Th. 2}
Sia G = (V, T, P, S) una CFG, e $\alpha \in (VUT)^{*} allora S \implies ^{*} \alpha$
se e solo se $S \implies _{rm} ^{*} \alpha$ OPPURE $S \implies _{lm} ^{*} \alpha$



        
  	
\end{document}