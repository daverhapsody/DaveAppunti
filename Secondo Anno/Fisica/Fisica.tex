\documentclass[12pt, a4paper, openany, oneside]{book}
\usepackage[italian]{babel}
\usepackage[T1]{fontenc}
\usepackage[utf8]{inputenc}
\usepackage{amsmath} 
\usepackage{xcolor}
\usepackage{hyperref}
\usepackage[margin=1in]{geometry} 
\usepackage{graphicx}
\graphicspath{{./img/}}
%usepackage[latin1]{inputenc}
\begin{document}
\pagestyle{headings}
\author{\href{https://github.com/daverhapsody}{DaveRhapsody}}
\title{Fisica}
\date{30 Settembre 2019}
\maketitle
\tableofcontents
\chapter{Introduzione al corso}
Non è presente materiale didattico, le lezioni sono architettate in modo che
si segua dalla lavagna, è consigliato dal prof stesso di usare gli appunti od i
libri che (per coloro che han fatto fisica) si usavano alle superiori.
\\ \\
Il programma è \textbf{tutta la fisica} in generale, ma affrontata in modo 
semplice, quasi banale, l'ultimo argomento dovrebbe essere il magnetismo, 
immaginatevi
quanto (non) si farà di quell'argomento. Ci sono 5 appelli in un anno, il primo
sarà a gennaio, poi febbraio, giugno, luglio e settembre, MA Gennaio e 
Febbraio dell'anno dopo sono inclusi
\\ \\
Il che significa che io posso fare i due parziali e poi fare l'orale anche a
Febbraio. Noi possiamo iscriverci solo allo scritto, e verremo spostati 
all'orale SE siamo già sufficienti. 
\section*{Alcune osservazioni}
Lo studio della fisica nasce dall'osservazione di una serie di fenomeni che 
accadono, con lo scopo di misurarli ed infine dimostrare il perchè questi
si verificano, 
\\ \\
Esistono una serie di \textbf{modelli} che sono in grado di descrivere ciò che
noi vediamo, ad esempio quando vedremo il moto, noi diremo "Osserviamo il moto
di un corpo", con corpo inteso come punto. Il punto è un oggetto di dimensioni
infinitesimali, e nel caso del moto ne analizzeremo i dettagli in modo 
specifico.
\\ \\
La nostra teoria parte da un modello semplificato che consente di capire il 
funzionamento di ciò che abbiamo di fronte. Nel caso dei Gas ad esempio ci 
saranno arricchimenti dei modelli (del tipo non esistono solo gas perfetti)
etc. \\ \\
Noi dobbiamo cercare di trovare il modello minimo, più semplice in grado di
\textbf{descrivere} una cosa. In fisica si adotta un atteggiamento \textbf{
Deduttivo}
, infatti non si ragiona generalmente in modo induttivo. Consideriamo che non 
esiste un modello finale che non si possa contradire.
\section{Cosa ci servirà}
Iniziamo definendo alcune quantità che ci interesseranno, ovvero massa, spazio 
e tempo.
\\ \\
C'è bisogno di capire che quantità si stia misurando, quindi si usano le unità 
di misura che cosa oggettivamente stiamo quantificando. Immaginatevi cosa 
significhi quantificare senza unità di misura. (Per dire Galileo usava i 
battiti del cuore.)
\\ \\
\subsection{Dal punto di vista numerico}
In qualsiasi campo si ha un ordine di grandezza, ogni fenomeno ha la
propria scala da usare, ci saranno i coefficienti di riferimento, i prefissi
($\mu,~ mm, ~ n$), c'è un vero e proprio intervallo di grandezze ($10^{n}$).
\section{Notazione scientifica}
Ecco un esempio di numero scritto in notazione scientifica: 
\[
5 \cdot 10^{5} = 50000 \\
8 \cdot 10^{-1} = 0,1 
\]
\\ \\
Ragionando su come sono composti, abbiamo le cifre significative, ovvero cifre
che hanno senso di essere tenute in considerazione. In che senso? Se devo 
misurare un banco di scuola posso dire che è tipo 1034 mm, OPPURE dire che è un
metro e 34 millimetri.. E' la stessa cosa, ok, detta in modi diversi
\\ \\
Se specifico una cifra (tipo anche) lo 0 in un $0,12320$ esso è cifra significativa!
\\ \\
Se ho invece un numero tipo $1,010$ posso scriverlo in due modi
\begin{itemize}
	\item 1,011 +/- (Ok non so come si fa il + e - in \LaTeX) 0,001
	\item 1,01 
\end{itemize}
Nulla di estremamente complesso ma va detto comunque, per dire se ho $1,234567$
posso approssimarlo in $1,23457$. \\ \\
\paragraph{ATTENZIONE} Nel caso della \textbf{NOTAZIONE SCIENTIFICA} si tiene
in considerazione la parte numerica $\neq$ 0. tipo $123.000.000$ ha 3 cifre che 
sono proprio 123
\paragraph{Siccome all'esame c'è una domanda su questa cosa, lo riassumo: } Dato
un qualsiasi numero $\alpha$ che per ipotesi consideriamo 123.000, si considerano
cifre significative tutte quelle che sono diverse da zero, MA potrebbero anche
essere 0 nel caso in cui gli zeri sian racchiusi tra cifre diverse da 0. \\ \\
Sì, faccio un esempio: \\ \\
4.003.000 (quattro milioni, non '4 virgola... cose') avrà quattro cifre, che sono 
4, 0, 0, 3, per cui in notazione scientifica avremmo 4,003 x $10^{6}$.
\chapter{Cinematica}
E' la branca della fisica che si occupa di descrivere la traiettoria di un 
corpo, dovremo predirla, calcolarla, basandosi su un campo di forza, uno spazio
, introdurremo la forza in grado di cambiare il moto di un corpo MA per prima 
cosa
\section{Come definiamo la traiettoria di un corpo}
Definiamo la differenza tra grandezza scalare e vettoriale
\begin{itemize}
	\item Le grandezze vettoriali hanno con sè una direzione, un verso, ed un
	modulo definito anche intensità. L'esempio per eccellenza è lo spostamento 
	e la velocità.
	\item Le grandezze scalari sono valori precisi fissi, dei valori che indicano 
	qualcosa di quantitativo più che qualitativo.
\end{itemize}
\subsection{Esempio di grandezza vettoriale}
Supponiamo di avere due punti $x_{0} ~ e ~ x_{1}$ ponendoli distanti $\lambda$ tra 
loro. $\lambda$ sarà coincidente con $x_{1} ~ - ~ x_{0}$. Per definire il verso
basta osservare chi è il minimo tra $x_{0} ~ e ~ x_{1}$, lo si vede graficamente,
oppure osservando chi dei due è il maggiore. 
\\ \\
Da un lato abbiamo un vettore (ancora monodimensionale), ma abbiamo anche 
dato un piano dimensionale, per esprimere il concetto di vettore relativo alla
posizione del nostro punto.
\\ \\
Il sistema di riferimento è il sistema cartesiano, in questo caso Monoasse 
pertanto ci basta avere solo la $x$. $x_{0} ~ e ~ x_{1}$ sono semplicemente dei 
punti, ma hanno un nome specifico, in questo caso sono delle vere e proprie
posizioni.
\\ \\
Come si diceva prima, per capire il \textbf{Verso} bisogna osservare la differenza
tra $x_{0}  ~~ e  ~~ x_{1}$, se negativa allora va all'indietro, al contrario andrebbe 
avanti molto semplicemente
\subsection{L'esempio di una palla che cade in un piano inclinato}
Il nostro punto materiale è la palla, e per capire lo spostamento bisogna
tracciare un grafico che indica le posizioni lungo le quali la pallina passa,
quindi si semplifica tutto con un grafico a singolo asse.
\\ \\
Chiaro che se ho un modello \textbf{Dinamico} è un problemino diverso perchè
avrei anche forze tipo la grafità etc, ma per ora descriviamo questo moto.
\\ \\
La pallina parte dalla posizione $p_{0}$ e passerà per un $p_{1,2,3,4}$ aventi
una serie di tempi passati dall'istante 0 che si chiameranno $t_{1,2,3,4}$ etc.
\\
Per descrivere questo bisogna trovare una legge che sia in grado di esprimere
per qualsiasi istante quali possano essere le condizioni. 
\[
\begin{cases}
t_{\lambda} = tempo ~ richiesto ~ per ~ arrivare ~ dalla ~ posizione ~ p_{0} ~ a
~ p_{\lambda}  \\
\end{cases}
\]
\subsection{Alcune precisazioni}
\begin{itemize}
	\item Lo spostamento è la distanza in linea d'aria
	\item La distanza percorsa può essere nettamente maggiore, poichè è il 
	percorso specifico che vado ad effettuare
	\begin{itemize}
		\item Per intenderci, da A a B potrei dover passare per un punto C, 
		la distanza diventa minimo la somma di( A + B ) + (C + B), di conseguenza
		a meno che siano allineati, cambia già la distanza
		\item Se da A vado a B e torno indietro, la distanza percorsa è 2AB, 
		mentre lo spostamento vettoriale è 0
	\end{itemize}
\end{itemize}
Lo spostamento è vettoriale, la distanza percorsa è uno scalare
\section{La velocità}
E' la quantità di spazio(s) percorsa da un corpo in un determinato tempo(t),
specificando che ci sia la distanza percorsa e lo spostamento.
\paragraph{Dati due punti} Lo spostamento non è altro che un vettore che parte
dal primo al secondo punto, quindi che va da $p_{0}$ a $p_{1}$.
\\ \\
Attenzione, prima c'è da tenere conto della differenza dei tempi, che chiameremo 
$\Delta t ~ = ~ t_{Finale} ~ - ~ t_{Iniziale}$
\\
Abbiamo più tipi di velocità:
\begin{itemize}
	\item {Velocità media scalare: $v_{media} = \frac{distanza~percorsa}{\Delta
	t}$ che è la distanza percorsa sul tempo passato da quando son partito a 
	quando sono arrivato }
	\item {Velocità media vettoriale: $\vec{v} = \frac{\vec{\Delta x}}
	{\Delta t}$ con $\Delta x$ che è il vettore spostamento tra la posizione 
	$p_{iniziale}$ e $p_{finale}$ }
\end{itemize}
\subsection*{Osservazione:}
Ragionando per formule inverse, se voglio capire quanto ho percorso mi basta 
fare $d = \Delta t \cdot v_{media}$ , ma in realtà non è propriamente corretto. 
\\ \\
Se per esempio avessi qualcosa del tipo
\[
\begin{cases}
SE ~ t_{1} = 1 ~ E ~ x_{1} = 1  \\ 
SE ~ t_{2} = 2 ~ E ~ x_{2} = 4  \\ 
SE ~ t_{3} = 3 ~ E ~ x_{3} = 9  \\
SE ~ t_{4} = 4 ~ E ~ x_{4} = 16 
\end{cases}\]
Posso osservare che lo spazio percorso x(t) corrisponda all'accelerazione A*$t^{2}$
\[x(t) = At^{2}\]
Queste quantità sono vicine alle nostre esigenze quotidiane, oggettivamente lo
spostamento vettoriale non dice nulla, non ci permette di dire assolutamente
nulla durante uno spostamento. Ok, sì, la velocità media, ma in fisica non è che
conti poi così tanto.
\paragraph{Esempio}	
Prendiamo un percorso $\Delta x$ (differenza tra un $x_{0} $ e $x_{1}$ che
decidiamo noi) se io impiego un tempo $\Delta t$ (differenza tra un $t_{0} $ e 
$t_{1}$ che sono istanti di tempo diciamo) avrei: $$\frac{\Delta x}{\Delta t} 
= v_{media}$$ Ora il concetto è che non mi dice nulla di cosa accade nel mezzo
del tragitto. 
\paragraph{Ipotesi} Immaginate di avere istante tra i due che abbiam scelto,
se usassimo la velocità media, in un determinato istante, per via dell'approssimazione
potrebbe risultare che abbiam percorso più o anche meno chilometri, è troppo
impreciso MA
\\ \\
Più sono corte le distanze, o meglio, minore è il valore di $\Delta x$ e minore
sarà l'errore di approssimazione. Basti pensare alla media di un viaggio per
ipotesi da Milano a Roma, magari per un tratto vado a 150, ma in un altro per 
il traffico vado a 3 chilometri al millennio, la media è bassissima MA per via
di questi due picchi
\\ \\
Possiamo ricavare dalla nostra formula con il $\Delta x$ e $\Delta t$ che quindi
la posizione che si assume in un determinato istante sia:
\[x_{1} = x_{0} + v_{media} \Delta t\]
La velocità media però non indica praticamente nulla del moto, se mi servono dati
precisi (es contachilometri) su una determinata velocitá prendo intervalli sempre minori.
Quando il $\Delta t$ tende a 0, notiamo che la funzione non tenderà ad $\infty$ 
perchè c'è corrispondenza negli ordini di infinito.\\ \\
Quindi chiamo questo limite con $t \to 0$ di $\frac{Velocita'~vettoriale}{\Delta t}$ 
\[\lim_{\Delta t \to 0} \frac{\Delta \vec{x}}{\Delta t} = v_{istantanea}\]
velocità istantanea, che non è altro che la velocità in un determinato istante.
%perchè in ognuno degli istanti ho una velocità v(t). \\ \\
Per ogni istante, per definizione di derivata, io calcolo alla fine 
\[v(t) = \frac{dx(t)}{dt}\]
Quindi in pratica otteniamo che la velocità istantanea è letteralmente la 
derivata della posizione, in cui la t è la "discriminante" della velocità istantanea che si 
aveva in un determinato istante (perdonate la ripetizione).
\\ \\ 
Con questa velocità istantanea possiamo (se applichiamo la legge oraria), 
calcolare in modo più preciso la posizione in un determinato istante! Come?
\[x_{t} = x_{0} + v_{t_{0}} dt\]
Cioè siamo arrivati che abbiamo la posizione iniziale e l'istante iniziale, più
la velocità istantanea (che è una derivata), ora ci basta solo applicare la 
formula. 
\\ \\ 
Prendiamo ora in esame un grafico che ha sulle ascisse il tempo e sulle ordinate 
le velocità istantanee registrate. Come nel caso precedente immaginiamo di avere 
un grafico con una funzione monotona crescente. Prendo due punti del grafico e
calcolo la velocità media tra essi 
\[\frac{v_{1} + v_{2}}{2}\]
sappiamo anche che la velocità media è definita come $\frac{\Delta x_{n}}{\Delta t_{n}}$
da cui ricaviamo $\Delta x_{n} = v_{med} \Delta t_{n}$.
Se ripetiamo questo procedimento per tutti i $\Delta t$ avremo:
%Dati due punti $p_{0}$ e $p_{1}$ in una curva, essi avranno quindi i corrispettivi 
%istantanei, quindi otteniamo un $\Delta x$ ed un $\Delta t$, prendendo questi 
%ultimi si avrà che $$\Delta x_{n} = v_{m_{n}} \Delta t_{n}$$ e quindi in 
%pratica per finire avremmo
\[\Delta x = \sum_{k=1}^n \Delta n = \sum_{k=1}^n v_{m_{n}} \cdot \Delta t_{n}\]
La somma dei $\Delta x$ ennesimi, è coincidente con la somma di tutte le aree
$A_{n}$ dove $A_{n} \sim \Delta x$. \\ \\
Se $\Delta t_{n} \to 0$ allora $\Delta x = \int v(t) dt$ \\
Per un punto specifico (sempre facendo tendere $\Delta t ~ a ~ 0$) ad esempio avremmo che 
\[x_{1} = x_{0} + \int_{t_{0}}^{t_{1}} v(t) dt \]
\\
{\color{black} \rule{\linewidth}{0.mm} }
\\
\paragraph{Precisazione di } \href{https://github.com/LiaBell47}{Giulia}:
La velocità va in funzione del tempo, MA gli estremi di integrazione vedendo il
grafico sono anch'essi dei tempi.
\\
{\color{black} \rule{\linewidth}{0.mm} }
\\
quindi per conoscere la posizione è sufficiente passare per un integrale definito della velocità.
Per esempio, se la funzione posizione nel tempo è $x_{t} ~ = ~ at^2 ~ la ~ v_{t}
~ sara' ~ v_{t} ~ =~  2at$
\paragraph{Piccola osservazione} Quando diciamo che $\lim{\Delta t \to 0}$ 
$\frac{\Delta x}{\Delta t} = v_{istantanea}$, stiamo dando per assodato che
diminuendo l'intervallo di tempo, diminuirà il relativo spostamento, pertanto
otterremo $\frac{0}{0}$, ok, ma vedremo che appunto tenderà ad un valore finito.
\section{Come varia la velocità}
Quando ho una velocità che cambia, posso definire la variazione della velocità 
vettoriale nell'unità di tempo, e questa si chiamerà accelerazione vettoriale media
e la indichiamo con $\vec{a_{media}}$, MA come prima servirà trovare l'accelerazione 
istantanea. Come si trova? Come prima si usano i limiti.
\\ \\
Per $\Delta t \to 0$ abbiamo che $\vec{a(t)} = \lim{\Delta _{t \to 0}} 
\frac{\vec{\Delta v}}{\Delta t} = \frac{dv}{dt} $
\\ \\
Data la velocità istantanea v(t) possiamo ricavare che l'accelerazione: 
\[a = a(t) = \frac{d x(t)}{dt} = \frac{d^{2}x}{dt^{2}}\]
Quindi di conseguenza ne segue che 
\[v(t) = v(t_{0}) + \int_{t_{0}}^{t}a(t)~dt ~~~~~ (Con ~ t\geq t_{0}) \] e
di conseguenza: \[x(t) = x(t_{0}) + \int_{t_{0}}^{t}v(t)~dt  \]
Considerando la velodità media si sta ovviamente considerando una velocità costante, 
graficamente, se v = $\frac{Spazio}{Tempo}$, allora v rappresenta il coefficiente
angolare della nostra retta.\\ \\ 
\\
{\color{black} \rule{\linewidth}{0.3mm}}
\paragraph{Illustrazione a cura di Letizia}
\begin{center}
\includegraphics[width=0.50\textwidth]{1}
\end{center}
In questo grafico si osserva che quando si parla di accelerazione si fa riferimento
a quella istantanea \\
{\color{black} \rule{\linewidth}{0.3mm} }
\\ \\ 
Se per ipotesi avessimo l'accelerazione costante invece avremmo non più 
$\frac{\Delta x}{\Delta t}$ MA $\frac{\Delta v}{\Delta t}$, che rappresenta 
l'accelerazione media mantenendo per ovvio che
\begin{itemize}
	\item $\Delta t = t_{1} - t_{0}$
	\item $\Delta v = \Delta t \cdot a_{Media}$ 
	\item $\Delta v = v_{1} - v_{0}$
\end{itemize}
Si riesce a ricavare lo spostamento che a questo punto diventa 
\[\Delta x = v_{0}\Delta t +(\Delta v) \cdot \frac{\Delta t}{2} =
\Delta t \cdot \frac{v_{1}+v_{0}}{2} = \Delta t(v_{media})\]
A questo punto proviamo a ricavare la posizione $x_{1}$:
\[x_{1} = x_{0} + \frac{v_{1}}{v_{0}}{2}\cdot \Delta t\]
Dove in pratica $v_{1} = v_{0} + a_{Media} \cdot \Delta t$, sostituendo vien fuori
\[x_{1} = x_{0} \frac{v_{0}+a_{Media}\cdot\Delta t + v_{0}}{2} \cdot \Delta t\]
Se generalizziamo:
\[x(t) = x_{0} + v_{0}(t-t_{0})+\frac{a_{Media}}{2}(t-t_{0})^{2}\]
Successivamente, per finire, noteremo che:
\[x(t) = x_{0}+\int_{0}^{t}v_{0} dt + \int_{0}^{t} a_{media}t\cdot dt =\]
\[= x_{0} + v_{0}t + \frac{1}{2}at^{2}\]
\\ \\ \\ \\ \\ \\ \\ \\ \\ \\ \\ \\ \\ \\ \\ \\
{\color{black} \rule{\linewidth}{0.3mm} }
\paragraph{Illustrazione a cura di Letizia}
\begin{center}
\includegraphics[width=0.75\textwidth]{vcostante}
\end{center}
In questo grafico si può osservare graficamente cosa accade quando si considera 
la velocità media come costante\\
{\color{black} \rule{\linewidth}{0.3mm} }
\chapter{Moto rettilineo uniformemente accelerato}
Dato qualsiasi piano inclinato, se un corpo parte dall'alto da un punto $x_{0}$
se esso rotola (con attrito volvente che è trascurabile), noteremo che con il
quadrato del tempo ($t^{2}$) la sua velocità si incrementerà
$x = x_{0} + v_{0}t + \frac{1}{2}at^{2}$, in cui possiamo giocare di nuovo a 
modificare le cose belle, tipo la t vi $v_{0}\cdot t$ può diventare 
$\frac{v-v_{0}}{a}$, da cui consegue che $a = \frac{v-v_{0}}{t}$ e quindi 
$x = x_{0} + \frac{1}{2}(v_{0}+v)\cdot t$.\newline
{\color{black} \rule{\linewidth}{0.3mm}}
\paragraph{Grafico dell'accelerazione costante}
\begin{center}
\includegraphics[width=0.75\textwidth]{acostante}
\end{center}
In questo grafico si nota che la variazione della velocità è coincidente con
la variazione del tempo per l'accelerazione media \\
{\color{black} \rule{\linewidth}{0.3mm}}
\section{Accelerazione di gravità}
Escludendo le forze di attrito con l'aria quando un corpo cade in caduta libera
si ottiene che esso (sul nostro pianeta) cada accelerando di 9,8 $\frac{m}{s^{2}}$.
Questi corpi prendono il nome di Gravi, e non importa che massa abbiano SE 
consideriamo l'assenza di aria che faccia attrito.
\\ \\
Poniamo caso di lanciare un sasso verso l'alto, o anche un mattone, MacBook, il
proprio gatto, quel che volete, noterete che questo salirà, e poi dopo un breve
periodo inizierà a tornare giù, questo è per via del fatto che l'accelerazione
gravitazionale è costante, ma all'oggetto che lanciamo cosa accade?
\\ \\
Noi sappiamo che $v = v_{0} + a\cdot t$, il grafico sarà una parabola che va
verso l'alto e poi ad un certo punto smetterà di crescere e inizierà a 
decrescere, e nel momento in cui il corpo scende l'accelerazione noteremo che
sarà $\leq$ 0 (Vettorialmente).
\paragraph{ATTENZIONE} L'accelerazione non si annulla MA sul punto di massimo
locale dove cambia il verso dell'accelerazione si avrà una velocità = 0
\chapter{Sistema cartesiano}
Il grafico di un qualsiasi sistema cartesiano è composto da due assi, uno x ed
uno y, perpendicolari tra essi, ed ogni punto sul grafico è composto da due
coordinate (x, y) tali che $P = (x_{P}, y_{P})$
\section{Coordinate polari} 
Ogni punto ha una distanza dall'origine, quella distanze si chiama R, e l'angolo
che si forma tra l'asse x e R si chiama $\phi$, di conseguenza ricaviamo che:
\begin{itemize}
	\item $\frac{y}{R} = \sin(\phi)$
	\item $\frac{x}{R} = \cos(\phi)$
	\item $y = R \cos(\phi)$
	\item $x = R \cos(\phi)$
	\item $\frac{y}{x} = \tan(\phi)$
	\item $\phi = \arctan(\frac{y}{x})$
	\item $R = \sqrt{x^{2}+y^{2}}$
\end{itemize}
Tutto ciò rimanendo comunque nel secondo quadrante, nel primo chiaramente c'è
da dire che rimane tutto cambiato di segno e viene indicato con le stesse lettere
del primo ma aggiungendoci un \' dopo.
\section{Somma tra vettori}
Dati due vettori è possibile individuarne la somma tramote la regola del 
parallelogramma che consiste nel tracciare due semirette parallele ai due 
vettori che abbiamo, e vanno aggiunte al termine del segmento del vettore
già presente, l'incrocio tra queste due semirette sarà il punto da cui puoi
calcolarti la somma (Non si capisce, ok, senza grafici è impossibile ma 
come già detto, li aggiungerò.) \\ \\ \\ \\
\\
{\color{black} \rule{\linewidth}{0.3mm} }
\paragraph{Letizia ci salva un'altra volta con un grafico}
\begin{center}
\includegraphics[width=0.75\textwidth]{sommaVettoriale}
\end{center}
%DESCRIZIONE DELL'IMMAGINE
Graficamente si verifica quello che ho scritto qui sopra. Si noti che tra l'altro
la velocità media
$\overrightarrow{v_{m}} è parallela (||) a \Delta \overrightarrow{r}$\\
e $\overrightarrow{v}=\frac{ds}{dt} \widehat{u}t $ (con u che sarebbe un 
versiore, ed s lo spazio), mentre per quanto concerne l'accelerazione 
$\overrightarrow{v}=\frac{dv}{dt} \widehat{u}t $ (Non c'è più s ma v che sarà
la velocità)
{\color{black} \rule{\linewidth}{0.3mm} }
\\
Se ragioniamo nelle due dimensioni la questione del nostro moto si complica, 
perchè abbiamo:
\[\overrightarrow{r} = \widehat{i}\cdot x + \widehat{j} y\] e quindi
\[\frac{d \overrightarrow{v} (t)}{dt} = \widehat{i} \cdot \frac{dx}{dt}+ 
\widehat{j} \cdot \frac{dy}{dt}\]
Scritto più easy: $\widehat{i} v_{x} + \widehat{y} = \overrightarrow{v}$
\[\overrightarrow{v_{Media}}= \frac{\Delta \overrightarrow{x}}{\Delta t} = 
\frac{\overrightarrow{r}}{\Delta t} \] dove $\Delta r = r_{1} - r_{0}$ e sarebbe
la distanza vettoriale.  \\ 
$\overrightarrow{v}(t) = \frac{\Delta \overrightarrow{r}}{\Delta t}$ con 
$\Delta t \to 0$
\\ \\ 
L'obbiettivo è quello di passare dal livello 1D al 2D, e la trigonometria entra
in gioco per via del fatto che vengono a formarsi dei triangoli
\\ \\
\\
{\color{black} \rule{\linewidth}{0.3mm} }
\paragraph{Illustrazione a cura di Simona}
\begin{center}
\includegraphics[width=0.75\textwidth]{mruvettoriale}
\end{center}
Da questo grafico è possibile notare in che modo interagiscono graficamente 
(considerando il punto di vista vettoriale) le velocità. 
\\
{\color{black} \rule{\linewidth}{0.3mm} }
\\
\section{Moto Circolare}
Consideriamo una circonferenza di raggio r, ed un punto materiale che si muove
lungo la suddetta circonferenza, che con l'asse x forma un angolo $\theta$. \\
\\
Supponiamo che il moto della velocità istantanea lungo la circonferenza sia 
costante, o meglio, diciamolo in modo figo: $\frac{dv}{dt} ~ costante$, se calcolo
l'accelerazione vettoriale, anche se il modulo è costante ma la direzione cambia
allora l'accelerazione non potrò mai esser nulla. 
\[\overrightarrow{a} = \frac{\overrightarrow{v}}{dt} \neq 0\]
\subsection{I vantaggi delle coordinate polari}
Introduciamo il concetto di radiante, ovvero date l, $\theta$ ed r,
\begin{itemize}
	\item Con r Raggio
	\item $\theta$ che è l'amgolo
	\item l che è la distanza NON vettoriale, cioè l'arco insomma tra due punti
\end{itemize}
Con $\theta = \frac{l}{r} = \frac{2 \pi r}{r} = 2 \pi$, pertanto si ottiene che
la velocità angolare $\omega$ sia = a $\omega = \frac{d \theta}{dt}$, e ricordiamo
che prima si era detto di essere in velocità costante, pertanto se moltiplichiamo
la velocità angolare per il raggio cosa otteniamo? Esatto, la velocità a cui
ci stiamo spostando, ovvero:
\[\omega R = \frac{dl}{dt} = v\]
Inoltre definiamo il concetto di PERIODO (t) che è il tempo impiegato per fare
tutto un giro della circonferenza: $T = \frac{2 \pi}{\omega}$, analizziamo
ora le leggi orarie
\[s(t) = s_{0} + v\cdot t\] Oppure
\[\theta(t) = \theta _{0} + \omega t\] a questo punto ragioniamo su quella
che è l'accelerazione vettoriale:\\ 
Sappiamo che $\overrightarrow{v_{m}} = \frac{\Delta \overrightarrow{r}}{\Delta t}$,
quindi ne deriva che $\overrightarrow{a_{m}} =
\frac{\Delta \overrightarrow{v}}{\Delta t}$ oppure in modo anche più preciso
\[\overrightarrow{a} = \lim \limits_{\Delta t \to 0} 
\frac{\Delta \overrightarrow{v}}{\Delta t} = a_{c} \widehat{u}_{n}\] in cui la
n specifica che è perpendicolare e non tangente, altrimenti sarebbe $\widehat{u}_{t}$, 
quindi si può concludere che l'accelerazione centripeta \[\overrightarrow{a_{c}}
= \frac{v^{2}}{r} \cdot \widehat{u}_{n}\] in cui ricordiamo che n indica che il 
versore sia PERPENDICOLARE, infatti la n sta per versore NORMALE.
Tutto questo se la velocità è costante.
\section{Moto circolare non uniforme}
Oltre all'accelerazione centripeta servirà l'accelerazione tangente o tangenziale
che è l'accelerazione lungo l'arco PERO' tangente alla circonferenza.
\[\overrightarrow{a_{t}} = \frac{d^{2}l}{dt^{2}}\cdot \widehat{u}_{t}\] E
si ricava che quindi $\overrightarrow{a} = \overrightarrow{a_{t}} \vee 
\overrightarrow{a_{c}}$, nel senso che è la stessa accelerazione ma considerata
da due sistemi di riferimento diversi. \\
Di conseguenza se è uniformemente accelerato, possiamo
anche stabilirne la legge oraria, che è simile a prima: 
\[l(t) = l_{0} + v_{0}t + \frac{1}{2} \cdot a_{t}t^{2}\], che è di base, mentre
nel caso andiamo a considerare le velocità angolari.
\[\theta(t) = \theta _{0} + \omega _{0}t + \frac{1}{2} a_{\omega}t^{2}\]
Se prima si è considerata la velocità normale ora consideremo quella tangenziale, 
ma alla fine, in un caso vedavamo quanto ci si era spostati, nel secondo caso
si vede quanto si è allargata o ristretta la circonferenza. Perchè? 
Perchè \textbf{la circonferenza è l'insieme di tutti i punti equidistanti da un punto
detto centro}.
\section{Circonferenza e Moto Armonico}
Prendendo una qualsiasi circonferenza ed un punto P su di essa, è possibile 
tracciare sull'asse x ed y i punti relativi a P, e ne consegue che
\[x = R\cos \theta = R\cos(\omega t) e \\
y = R\sin \theta = R\sin(\omega t) \] 
E quindi 
\[\frac{dx}{dt} = \omega R \cdot \sin(\omega t) \] e
\[\frac{dy}{dt} = \omega R \cdot \cos(\omega t) \]
Inoltre \[v = \sqrt{\frac{dx^{2}}{dt} + \frac{dy^{2}}{dt}} \]
\\
{\color{black} \rule{\linewidth}{0.3mm} }
\\
\paragraph{Illustrazione a cura di Simona}
\begin{center}
\includegraphics[width=0.75\textwidth]{motoarmonico}
\end{center}
In quest'illustrazione si vede riassunto graficamente quello che ho menzionato
nel paragrafo superiore \\
{\color{black} \rule{\linewidth}{0.3mm} }
\\
Finalmente si può rispondere alla domanda che ci si è fatti all'inizio: ovvero,
come posso calcolare la traiettoria di un corpo BASANDOMI sulle forze che 
agiscono su di esso?\\ \\ 
Introduciamo pertanto il concetto di trasformazione Galileiana
\section{Trasformazione Galileiana} 
Il primo passo è sempre avere un sistema di riferimento a coordinate, ma di 
questi ne esistono infiniti, pertanto sarà importante rimanere nello stesso 
sistema di riferimento, nel senso che tutto varia in base a quello. \\ \\
Ad esempio, supponiamo di avere 3 posizioni di 3 corpi diversi. 
\begin{itemize}
	\item A: Un'automobile
	\item P: Un pedone
	\item B: Una bicicletta
\end{itemize}
E di conseguenza le distanze tra essi $\overrightarrow{X_{BA}}, 
\overrightarrow{X_{PA}}, \overrightarrow{X_{PB}}$, noi possiamo dedurre che:
$v_{PB} = \frac{dx_{PB}}{dt} = v_{AP} + v_{AB}$. \\ \\
Se queste sono velocità da considerarsi vettoriali ($\overrightarrow{v_{PA}}$), 
siccome i punti sono in serie sullo stesso piano, allora basta fare semplicemente
la somma. 
\paragraph{Detto in modo più umano: } Siete in macchina, andate a 70$\frac{km}{h}$,
vi supera una macchina che va a 130$\frac{km}{h}$, ecco, dal punto di vista
della velocità abbiamo appena detto che è come essere fermi, e una macchina va
a 60$\frac{km}{h}$, dipende dal vostro sistema di riferimento.
\\ \\
\chapter{La Forza}
Finora abbiam descritto le accelerazioni per esempio di un corpo in caduta 
libera, abbiam descritto COME un oggetto cade, ma nessuno ha risposto al perchè
l'oggetto cade. Prima ci si chiedeva perchè c'è movimento? \\ \\
Perchè alcuni oggetti si muovono senza una causa? Esempio lampante: La Luna,
si muove (da un pochino di tempo anche) da sola, senza nessuno che spinge.
\paragraph{Consiglio: } Guardatevi questo spettacolo a teatro di Paolini che 
parla di Galileo, io l'ho adorato, anche se mi rendo conto che in sessione non
avrete per nulla voglia di guardare qualcosa di questo genere ->
\href{https://www.youtube.com/watch?v=8alJ9eFl634}{This}
\\
{\color{black} \rule{\linewidth}{0.3mm} }
\\
Ora, da una parte abbiamo le forza che sono descrivibili con delle formule, 
e dall'altro abbiamo la meccanica che vuole sapere cosa fanno le nostre forze
al punto materiale che le subisce. Ammesso che capisco quanto valga una forza, 
cos'è una forza?
\paragraph{Definizione: }Da \href{https://it.wikipedia.org/wiki/Forza}{Wikipedia}: 
Una forza è una grandezza fisica 
vettoriale che si manifesta nell'interazione reciproca di due o più corpi, sia 
a livello macroscopico, sia a livello delle particelle elementari. Quantifica il 
fenomeno di induzione di una variazione dello stato di quiete o di moto dei corpi 
stessi; in presenza di più forze, è la risultante della loro composizione vettoriale
a determinare la variazione del moto. La forza è descritta classicamente dalla 
seconda legge di Newton come derivata temporale della quantità di moto di un
corpo rispetto al tempo \\ \\
All'atto pratico una forza è quella che causa il movivmento, di fatto se si 
applica una forza, o comunque ci sono delle forze che agiscono su di esso, allora
quest'ultimo si muoverà, o per esser più preciso Accelera. Essendo una grandezza
vettoriale, se la somma algebrica delle forza applicate su un corpo è = 0 allora
il corpo avrà accelerazione 0. \\ \\ \\
\footnote{Lamentela personale}Credo di volermi rifiutare di fare menzione della 
definizione dataci a lezione con l'esempio della molla. Ditemi voi se è ammissibile 
una definzione del tipo "Una forza è quella esercitata da una molla quando la tiri 
o comprimi". 
\\ \\
Detto in modo più matematico, primo principio della dinamica (\textbf{principio di inerzia}):
\[
Se~~\overrightarrow{F} = \sum_{\kappa=1}^n \overrightarrow{F_{\kappa}} = 0
~~ allora ~~ a = 0
\]
Se su un corpo non agisce alcuna forza viene mantenuta costante la propria 
velocità, pertanto essa può essere anche $\geq 0$, ma rimarrebbe costante per
via dell'assenza di forze che si oppongono al movimento.
\\ \\
Esatto, perchè potrebbe accadere che esistano forza che non agiscono direttamente
come quella centripeta, quella di gravità, quella centrifuga, è pieno di forze
che agiscono indirettamente. Sono forze che compaiono nel sistema di riferimento
non inerziale.
\\ \\
\section{Secondo principio della dinamica}
Dato un corpo fermo, in base al suo peso dovremo applicare una forza maggiore o
minore, per ottenere una determinata velocità. Cioè se devo spingere una penna, 
e farla viaggiare a una velocità $\lambda$, è ben diverso far raggiungere la
stessa velocità $\lambda$ ad un camion. (Ponendo $\lambda > 0$) \\ \\
Da questo consegue che la massa influenza sia l'accelerazione che la forza.
Infatti il secondo principio della dinamica dice che:
\[
F = m\cdot A
\]
Provocare dell'accelerazione in un corpo richiederà più o meno forza in base
alla massa del corpo da spingere. Ne consegue che:
\[
m_{x} = m_{0} \cdot \frac{a_{0}}{a_{x}}
\]	
Chiaramente c'è da dire che un qualsiasi corpo, se è fermo necessiterà di 
applicare più forza per produrre un determinato movimento, se spingiamo un corpo
che già si muove ci vorrà meno forza.
\paragraph{La Massa: } Anche se la si dà per scontata, non l'abbiamo definita in
modo specifico, è considerabile al momento come quantità di materia, pertanto
mi raccomando da ora non confondiamo più massa e peso.
\paragraph{Il Peso: } Il peso è una forza, che è esercitata dal campo gravitazionale
terrestre ed è di circa 9,8 N (dove N sarebbe Newton). Il peso è una forza, è una
grandezza vettoriale, ha una direzione, un verso, ed un modulo o intensità.
\paragraph{Precisazione: } Nella meccanica classica tendenzialmente una forza è
realizzabile/applicabile/esistente SE son presenti due corpi, pertanto ci si può
anche esprimere dicendo che: Presenza di forze $\to$ presenza di più corpi. \\ \\
In assenza di forza non si ha accelerazione, pertanto lo stato di moto di un
punto materiale è o a 0 o velocità costante, di fatto nel mondo in cui viviamo,
in quello reale ci sono sempre delle forze che agiscono, e appare che i corpi
non abbiano velocità costante. \\ \\
Questo è dovuto per esempio alle forze di attrito che ci sono sempre, cose
di questo tipo. \\ 
Per tastare con mano quello che è un moto perfettamente (o quasi) costante, si 
può osservare il movimento della luna, che ok, non è perfettamente nel vuoto 
anche perchè il vuoto assoluto non esiste, però non ha degli attriti così consistenti.
\\ \\
Il principio (detto principio di inerzia) si basa sul concetto del "In assenza
di forte il corpo conserva il suo moto", o meglio, se non ci sono corpi vicini
che influenzino un corpo allora la sua accelerazione sarà = 0 .
\section{Quantificare una forza}
L'unità di misura di una forza è il Newton (N), \\
Se il corpo pesa 1kg si ha che: 1 N = $1m/s^{2}\cdot kg$ pertanto $F(t) \alpha a(t)$,
cioè la forza applicata in un tempo è proporzionale all'accelerazione in quel
determinato tempo. 
\begin{itemize}
	\item [m] = kg
	\item [a] = $\frac{m}{s^{2}}$
	\item [F] = kg $\frac{m}{s^{2}}$ = N 
\end{itemize}
Ovviamente (come specificato prima) anche $\sum_{i=1}^n
\overrightarrow{F_{i}} = 0 \to \overrightarrow{a} = 0$, e quindi ad esempio:
\[
\begin{cases}
m_{0} = 1kg \to a_{0}\\
m_{1} = \lambda kg \to a_{1}
\end{cases}
\]
$\frac{m_{0}}{m_{1}} = \frac{a_{1}}{a_{0}}$, (Siccome sono inversamente proporzionali)
e quindi questo significa che 
ragionando per formule inverse $m_{1} = m_{0}\cdot \frac{a_{1}}{a_{0}} \forall F$
\\ \\
Non si è specificato, MA la somma di due masse è fattibile, cioè se prendo 
una mela e una pera e le incollo, la loro massa sarà $mela + pera + colla$, 
pertanto l'accelerazione seguira questa somma, nel senso, se ho calcolato
l'accelerazione prima che avrebbe la mela, poi la pera, la loro somma avrà un 
valore approssimativativamente proporzionale alla somma delle masse (circa perchè
c'è la colla ma è diciamo trascurabile.)
\\ \\
\paragraph{Riprendendo il secondo principio della dinamica: }
Ricordiamoci la difficilissima formula $F = m \cdot a $, e quindi 
$\overrightarrow{F} = m \cdot \overrightarrow{a}$. \\ \\
Il problema è che io data una forza posso concludere di avere una massa per un'accelerazione MA
non posso concludere data una massa ed accelerazione che ci sia quella forza.
Non mi sta definendo la forza, è una sua espressione. Ti serve una teoria sulla
forza per lavorarci su. In termini più semplici: Da sinistra verso destra l'equazione 
$\overrightarrow{F} = m \cdot \overrightarrow{a}$ va bene MA da destra verso
sinistra no. Un esempio di alternativa a questo? 
\paragraph{Legge di Gravitazione Universale}
(Approfondito da pagina \pageref{LeggeGrav} di questi appunti)
\[\overrightarrow{F_{a}} = G \cdot \frac{M\cdot m}{d^{2}}\] in cui la G è la
costante di gravitazione universale ed è $g = \frac{G\cdot M}{R_{Terra}^{2}}$, 
g invece è la forza g, quella che si vede di solito in F1 in curva o frenata, in
cui letteralmente si hanno decelerazioni e accelerazioni (non necessariamente 
rettilinee, generalmente più in curva) fino ai 5g. \\ \\
Inoltre, siccome questa g dipende dal raggio della $terra^{2}$, quindi può 
cambiare anche sulla terra stessa.
\section{La Molla}
La molla è un tipo di corpo su cui si può applicare sia una compressione che 
un'espansione, ma attenzione, non si chiamerà più espansione ma allungamento, 
perciò essendo un corpo, su di esso si può applicare una forza. \\ \\
Sulla molla si può applicare una forza sia tirandola che spingendola, e questa
forza è:
\[
\overrightarrow{F_{k}} = -k \cdot (\overrightarrow{x} - \overrightarrow{x_{0}})
\]
In cui $\overrightarrow{x} - \overrightarrow{x_{0}}$ sui libri di fisica vien
chiamata $\Delta l$ e sarebbe l'allungamento, o quanto s'è deformata la molla.
\\ \\
\section{Il peso}
Siccome misurare il rapporto tra masse è uguale a misurare il rapporto tra forze,
allora si conclude che si ha la stessa gravità, che sulla terra è 9,8, e quindi
si può anche dire che per misurare una massa, bisogna tener conto della forza peso 
e confrontarla con la gravità che si sta considerando.
\[\overrightarrow{P} = m \overrightarrow{a} = m \overrightarrow{g} = - m\cdot 
g \cdot \widehat \gamma\]. 
Se una bilancia misura una forza, dato g, si ottiene che $\frac{forza}{g} = massa$,
per cui se è vero che il peso cambia in base a dove ci troviamo, non è vero
che lo faccia anche la massa, e infatti la massa quella è e quella rimane.
\section{Terza legge della dinamica}
E' anche detto principio di azione/reazione, ovvero:
\paragraph{Definizione: } Dati due corpi, per qualsiasi forza che viene applicata 
da un corpo all'altro (per ogni azione), corrisponderà una forza (reazione) che
sia uguale e contraria $\to azione(corpo_{a}, corpo_{b}) = 
reazione(corpo_{a}, corpo_{b}) = azione(corpo_{b}, corpo_{a})$ \\ \\
Questo principio è diretto derivante dall'interazione tra due oggetti, due corpi,
e con questa consapevolezza si deduce che (oltre al discorso della presenza di
più corpi) si avranno anche delle ulteriori forze dovute ad attriti etc. \\ \\
L'azione e reazione sono chiaramente esercitate dipendentemente da chi applica
forza su chi. Nel senso, se io pesto un pugno sul tavolo, io agisco, il tavolo
reagisce, io perisco la reazione del tavolo e sento dolore.\\ \\
La somma di tutte le forze che agiscono su un corpo è $$\overrightarrow{F_{A}}
= \sum_{i=1}^{n} \overrightarrow{F_{A_{i}}} = m \cdot \frac{d^{2}
\overrightarrow{x_{A}} (t)}{dt^{2}} = m\cdot \overrightarrow{a_{A}}$$ In cui, 
per non confonderci specifico che in questa formula 
\begin{itemize}
	\item A = nome del corpo
	\item a = accelerazione
	\item $d^{2}$ = derivata alla seconda
	\item $t^{2}$ = letteralmente t al quadrato
\end{itemize}
\paragraph{Forza normale}
Supponiamo di avere un tavolo T ed un corpo di massa M, un cubo, per intenderci.
E ricordiamo che il tavolo sta al di sopra della terra, che chiameremo TE. 
(Mantenendo per ovvio che in questo caso la gravità si possa considerare 
costante $\to F_{G} = G \cdot \frac{M_{Terra \cdot M}}{R_{Terra}^{2}}$)
\\
{\color{black} \rule{\linewidth}{0.3mm} }
\begin{center}
\includegraphics[width=0.75\textwidth]{forzanormale}
\end{center}
In breve, sulla massa agiscono ben due vettori, e pure sul tavolo agiscono più
vettori, in quanti agiscono sia l'azione della forza peso che la reazione del 
cubo. \\ \\
In italiano: Sul tavolo c'è il peso del cubo, il tavolo esercita peso sulla terra,
ed esercita azione su cubo. \\ \\
La domanda è: Perchè questi corpi rimangono fermi? La somma di tutte le forze
è uguale a zero, quindi l'accelerazione è 0, quindi stanno fermi.
\paragraph{Precisazione:} La massa non esercità forza peso, ma è la forza peso
che viene esercitata sulla massa.
\\
{\color{black} \rule{\linewidth}{0.3mm} }
\\
Il concetto di azione e reazione funziona perfettamente, ma bisogna sempre
ricordarsi del fatto che le forze vengano applicate da due corpi distinti.
\section{Il lavoro di una forza}
Il lavoro è il legame tra lo spostamento di un corpo per la quantità di energia
spesa per muoverlo. Attenzione, non è in generale quanto è lo sforzo, MA è quanto
movimento viene prodotto grazie ad una determinata forza. \\ \\
Più semplicemente, se tendo di spostare un pallone da calcio, applico una forza,
si muove, ergo compio del \textbf{lavoro}. \\ \\
Se tento di spostare l'Empire State Building, oltre a far la figura del deficiente,
ottengo che quest'ultimo malgrado applico l'incredibile forza dei miei muscoli,
non si muove, sta fermo, quindi il lavoro sarà 0.
\\  \\
Più precisamente:
\[L = F \cdot \Delta x\]
Forza x spostamento, molto semplicemente, ma occhio a non confondersi, poichè
spostamento e forza sono grandezze vettoriali ($\overrightarrow{F}, 
\overrightarrow{\Delta x}$)
\\ \\
L'unità di misura del lavoro è il Joule (J), che coincide con $N\cdot M$, e come
è possibile notare questo è uno \textbf{scalare} nel senso che (Raga, in algebra
io 'sta roba non l'avevo ancora capita.) il prodotto di due vettori è uno scalare.
\\
{\color{black} \rule{\linewidth}{0.3mm}}
\begin{center}
\includegraphics[width=0.75\textwidth]{graficoLavoro}
\end{center}
{\color{black} \rule{\linewidth}{0.3mm}}
\\
Avendo definito il lavoro, con questa definizione abbiamo una quantità che 
consente di definire il quantitativo di energia trasferito in un sistema
dall'esterno verso l'interno di un sistema. \\ \\
E se invece fossero più forze a generare uno spostamento? Cioè molto semplicemente,
se spostiamo un tavolo in due invece che da solo? Beh cambia molto lo 
spostamento, e infatti \[L = \sum_{i=1}^{n} \overrightarrow{F_{i}} \cdot \overrightarrow{S}\]
Ah ovviamente $L_{i} = \sum_{i=1}^{n} \overrightarrow{F_{i}} \cdot \overrightarrow{S}$,
da questa prospettiva si conclude anche che $L_{tot} = \sum_{i=1}^{n} L_{i}$.
\\ 
Infine, altro modo di esprimere il lavoro:
\[L_{\kappa} = \int_{a}^{b} F_{\kappa}dx  \]
Quindi chiaramente è possibile dire che:
\[L = \int_{a}^{b} F(x) dx\]
Ossia, nel caso prima semplicemente abbiamo specificato un punto preciso MA,
se ragionassimo con $\overrightarrow{F}$ per ottenere il lavoro totale manca un
piccolo passaggio:
\[L = \int F(x) dx + \int F(y) dy + \int F(z) dz = \int \overrightarrow{F}\cdot 
d \overrightarrow{s}\]
{\color{black} \rule{\linewidth}{0.3mm} }
\begin{center}
\includegraphics[width=0.75\textwidth]{velocitalavoro}
\end{center}
In questo caso cosa si osserva? La velocità al quadrato che avremo sarà coincidente
con la velocità iniziale alquadrato, che si somma con la velocità generata
dall'accelerazione provocata dalla applicazione della forza in questione.\\ \\
Quel K si chiama \textbf{ENERGIA CINETICA}, ossia l'energia associata al moto. Mi raccomando,
contate sempre che ci sono altre energie in questione che agiscono su un corpo
in movimento, allora tutto prende senso. (in seguito sarà approfondita nello
specifico)
\\
{\color{black} \rule{\linewidth}{0.3mm} }
\chapter{Principio di conservazione dell'energia}
E' il classico esempio del modo di pensare dei fisici, cioè osservi un fenomeno,
noti qualcosa di anomalo, lo testi, trai delle conclusioni, ma in questo caso
addirittura vengono inventate delle unità di misura.
\\ \\
Il concetto di Energia è risultante da calcoli che effettuiamo MA NON ESISTE, 
l'energia NON è fisica, non è qualcosa di tangibile, è CALCOLABILE, e rimane
nel tempo sempre uguale a se stessa in determinate situazioni. \\ \\
Se nel caso della meccanica e dinamica si aveva qualcosa di misurabile, tangibile,
per quanto riguarda la conservazione dell'energia si dovrà capire il concetto di
sistema.
\paragraph{Cos'è un sistema: } E' un insieme di punti materiali (può essere 
pure uno solo), può essere contenuto in un sistema, ovviamente.
\\ \\
\subparagraph{Precisazione: }
Siccome non so quanto fidarmi effettivamente delle nozioni riportate a lezione, 
vi includo anche la definizione "Un insieme di due o più corpi viene chiamato sistema"	
riportata a pagina 84 del manuale di Halliday-Resnick (Si ringrazia RC per la 
precisazione),
\paragraph{Alla fine} si ha una quantità di Energia, che viene associata al sistema, è
una serie di formule che vedremo, ma prima di applicarle devo capire bene quale
è il mio sistema di riferimento.\\ \\
Neanche a dirlo ci sono diversi tipi di energia, diciamo infinite forme da questo
punto di vista ad esempio:
\begin{itemize}
	\item Calore
	\item Cinetica
	\item Nucleare
	\item Chimica
\end{itemize}
Quello di cui ci occuperemo nello specifico per ora è l'energia cinetica.
\section{Energia Cinetica}
L'energia cinetica è (come detto sopra) l'energia che si associa al moto, o 
meglio l'energia che ha un corpo in movimento. Diciamoci subito le formule utili:
\begin{itemize}
	\item $L = \Delta \kappa$
	\item $\kappa = \frac{1}{2}m\cdot v^{2}$
	\item $L = \sum_{i=1}^{n} L_{i}$
	\item $L = \sum_{i=1}^{n} \overrightarrow{F_{i}} \cdot \Delta
	\overrightarrow{x} (Spostamento)$
\end{itemize}
Lavoro ed energia cinetica sono strettamente legati per via del fatto che 
appunto il lavoro è il prodotto di una forza per uno spostamento. Per cui \\
\[
L = \int_{x_{iniziale}}^{x_{finale}} F(x) dx = 
\int_{x_{iniziale}}^{x_{finale}} m\cdot a ~ dx
\]
A questo punto osserviamo che l'accelerazione è: $a = \frac{dv}{dt}$, quindi 
la formula è riscrivibile come \[L = \int_{x_{iniziale}}^{x_{finale}} m\cdot v 
\cdot \frac{dv}{dt} dx\] 
Ora le cose iniziano un po' a complicarsi, perchè se per ipotesi noi considerassimo
soltanto la velocità si avrebbe: $L = \int_{x_{iniziale}}^{x_{finale}} m\cdot v 
= \frac{1}{2}m\cdot v^{2}$\\
Tutto sto casino è per arrivare a dire che: 
\[\frac{1}{2}m\cdot v_{finale}^{2} - \frac{1}{2}m\cdot v_{iniziale}^{2} = 
\Delta\kappa\]
Ecco, questo $\Delta\kappa$ è la variazione dell'energia che c'è stata da un 
punto iniziale ad un punto finale. \\ \\
Ora la domanda è: se ho compiuto un lavoro, perchè $\Delta\kappa$ è = 0? Perchè
non stiamo considerando tutte le altre forze! \\
\\Se io prendo il mio telefono e lo
striscio sul tavolo da un punto A ad un punto B, questo si ferma, MA per via
della forza di attrito. 
\\
{\color{black} \rule{\linewidth}{0.3mm}}
\\
Da un punto di vista "Telemetrico" l'azione delle forze sul telefono che ho 
strisciato sul tavolo è di questo tipo:
\begin{center}
\includegraphics[width=0.75\textwidth]{telefonoStriscia}
\end{center}
Il primo grafico indica l'energia cinetica del telefono (che poraccio lo stiamo
massacrando a furia di sballottarlo in giro), mentre invece il grafico di sotto
indica come si comporta l'attrito che lo frenerà.\\ \\
(Per capire meglio cosa succede in un contesto del vuoto, ai videogiocatori 
consiglio di giocare \href{https://store.steampowered.com/app/244850/Space_Engineers/}
{Space Engineers}. E' bello, fidatevi.)
\section{La forza d'attrito}
\subsection{Cos'è}
Dal punto di vista pratico non è altro che la forza in grado di frenare un corpo
in movimento (Perchè? Perchè essendo una forza, essa si applica da un corpo
ad un altro, per questo è certo che si opponga al moto).\\ \\
E' \underline{sempre} \textbf{Opposta} alla \textbf{Direzione} dello spostamento,
o detto meglio: \textbf{Compie sempre lavoro \underline{negativo}} 
\subsection{Come si esprime?}
La chiameremo $F_{a}$, e si esprime in:
\[
F_{a} = \mu_{a}\cdot N
\]
Mentre invece per quanto riguarda l'effettivo suo \textbf{lavoro}:
\[
L_{a} = -\mu_{a}\cdot N \cdot \Delta x
\]
Date queste due nuove formule, possiamo concludere che un corpo si sposta SE
il lavoro dell'attrito è \textbf{minore} del lavoro che ha generato il moto:
\[
L = L_{\kappa_{i}} + L_{a} < 0 
\]
SE si verifica questa condizione ALLORA esiste il moto, in alternativa il corpo
sta fermo. 
\paragraph{Precisazione: } Come vedete c'è un `+` in mezzo, e non solo. C'è 
anche segnato una i sul nostro lavoro del corpo che consideriamo, cosa significa?
\paragraph{Risposta: } La forza d'attrito rimane la stessa (\textbf{A parità
di materiale/corpo/superficie}) MA progressivamente potrebbe cambiare nel tempo
il lavoro del corpo in questione. Come? Se smettete di dare gas, la macchina
diminuirà la propria velocità, e più andavate forte, più scenderà di botto 
all'inizio (La richiesta energetica rimane sempre proporzionale a $v^{2}$),
\section{L'energia potenziale}
\paragraph{Da} \href{https://www.wikiwand.com/it/Energia_potenziale}{Wikipedia}:
~In fisica, l'energia potenziale di un oggetto è l'energia che esso possiede a 
causa della sua posizione o del suo orientamento rispetto a un campo di forze.
Nel caso si tratti di un sistema, l'energia potenziale può dipendere dalla 
disposizione degli elementi che lo compongono.\\ \\ Si può vedere l'energia 
potenziale anche come la capacità di un oggetto (o sistema) di trasformare la 
propria energia in un'altra forma di energia, come ad esempio l'energia cinetica. 
\\ \\
\paragraph{Ragioniamo con un esempio: }
Prendiamo un corpo ed una molla, sappiamo che se il corpo si muove allora:
\[
L = \Delta \kappa
\]
Ecco, una volta detto questo, il lavoro che invece farà la molla per fermare 
il corpo che le arriva contro sarà:
\[
-L = \Delta U
\]
Quella U è l'energia potenziale, che per come è definita ($\Delta U = -L$) sarà
possibile definirla anche $\Delta U = \frac{1}{2}\kappa \cdot \Delta x^{2}$,
e quindi
\[
\Delta U = f(\Delta x)
\]
Data una forza $\mathbf{F}$, il lavoro W lungo una curva C è dato in generale
dalla relazione: 
\[
W = \int_{C} \mathbf{F}\cdot dx
\]
\paragraph{Precisazione sulle forze conservative: } A noi ci importa poco 
in questo caso di cosa succeda in mezzo tra un punto A ed un punto B, per capire
il lavoro fa testo solo il punto iniziale ed il punto finale, \textbf{INVECE},
nel caso dell'attrito si ragiona in modo diverso, perchè fa testo quello che
succede nel tragitto che separa A da B ($\Delta x$). Detto più scientifico:
\[L = \Delta\kappa = -\Delta U ~ con ~ A\to B \]
\[L = L(A,B) \]
\[L = L(x) \]
\[\Delta U = -L(x) = - \int_{A}^{B}  F(x)\cdot dx\]
Un esempio di forza NON conservativa è il magnetismo, ma lo vedremo più avanti.
Tornano all'energia potenziale, bisogna specificare che \textbf{Non esiste 
una energia potenziale assoluta}, dipende dal sistema di riferimento. In che senso?
Se calcolo l'energia potenziale di una bottiglia su un tavolo, rispetto al tavolo
è 0, MA rispetto al pavimento? Quindi attenzione, bisogna capire quale \textbf{punto
di riferimento} si usa per fare i confronti
\section{Energia potenziale in un determinato tempo}
Definiamo ora U(x), che ha diversi modi per essere definito, ovvero:
\begin{enumerate}
	\item \[
	U(x) = -L(x) = - \int_{x_{0}}^{x}  F(x) \cdot dx
	\]
	\item \[
	U(x) = - \int_{x_{0}}^{x}  F(x) \cdot dx + U(x_{0})
	\]
\end{enumerate}
Se ho un potenziale, allora esiste una forza conservativa, che è coincidente
con $- \frac{d(U(x))}{dx} = F(x)$, e qui c'è una doppia implicazione perchè SE
ho una forza conservativa allora si ha anche dell'energia potenziale.
\paragraph{Una forza conservativa } è una forza che dipende SOLO da istante 
iniziale e finale.
\subsection{Variazione dell'energia potenziale della forza peso}
Incominciamo dal ridefinire la forza peso:
\[
F_{p} = -m\cdot g
\]
Ragionando sulle definizioni menzionate qui sopra (Non ho ragionato mai nella
vita forever, sto copiando brutalmente dalla lavagna perchè i miei livelli
di comprensione son lievemente... Circa calati, circa..) si ha che:
\[
\Delta U = -L_{p} = m\cdot g \cdot \Delta y
\]
\chapter{Legge della Gravitazione Universale}
\label{LeggeGrav}
Riprendiamo un concetto importante, ovvero il secondo principio della dinamica:
\[
\overrightarrow{F} = m \cdot \overrightarrow{a}
\]
E quindi ragionando per formula inverse, Newton definì:
\[
\overrightarrow{a} = \frac{\overrightarrow{F}}{m}
\]
Successivamente occorrerà menzionare le leggi di Keplero.
\paragraph{Storicamente: }Ciò che fece fu semplicemente analizzare una serie di
valori numerici, stabilì un algoritmo che potesse generarli tutti, o che potesse
creare una relazione tra tutti quanti. 
\section{$1^{a}$ legge di Keplero}
Si rende conto che in generale ci sia la tendenza ad orbitare con traiettorie
elittiche e non perfettamente a cerchio. Per 'Orbita' si intende il fatto che 
un corpo si muova attorno ad un altro.
\section{$2^{a}$ legge di Keplero} 
Da YouMath:
\begin{center}
\includegraphics[width=0.75\textwidth]{ym}
\end{center}
~
\begin{center}
\includegraphics[width=0.75\textwidth]{2kep}
\end{center}
\section{$3^{a}$ legge di Keplero}
Qui si fa riferimento a diversi corpi che orbitano attorno al sole (o comunque
ad un altro corpo). Si dimostra che il rapporto tra il $periodo^{2}$ ossia il
tempo necessario per compiere un intero giro, ed il $raggio^{3}$ rimane costante.
\[ 
\frac{T^{2}}{R^{3}} = costante
\]
E se rimane costante significa che $T^{2} = k\cdot r^{3}$, e teniamoci a mente
questo $k = \frac{T^{2}}{r^{2}}$, servirà tra poco.
Con T che è il periodo e R che sarebbe il raggio dell'orbita
\paragraph{Queste leggi non spiegavano nulla, } Erano solo delle osservazioni, 
ci dicevano più che altro COME accadeva qualcosa più che PERCHE'. Si è sempre
lì. Le opinioni sul perchè erano sempre varie.
\paragraph{Galileo Galilei }con il suo principio di inerzia trova che non vi è
nulla a spingere i pianeti per farli orbitare, e ne consegue quindi che esista
una forza centripeta (che appunto attrae i corpi a sè verso un centro).
\paragraph{Si dimostra quindi che: }
\[
\frac{dA}{dt} = \frac{|\overrightarrow{r}|\cdot |d \overrightarrow{r}|}{2\cdot dt}
\]	
Con  $d \overrightarrow{r} = \overrightarrow{v}\cdot dt$ = $r\cdot d \theta$
\[
\frac{dA}{dt} = \frac{r^{2}}{2} \cdot \frac{dv}{dt} = \frac{r^{2}}{2}\cdot \omega
\]
E $\frac{r^{2}}{2}\cdot \omega$ rimane costante, ed $\omega$ sarebbe la velocità
angolare, calcolabile facendo $\frac{2\pi}{T}$ con T che è il periodo che ci
si impiega a compiere un giro.
\[
a_{c} = \omega^{2} \cdot r
\]	
Con $a_{c}$ che sarebbe l'accelerasione centripeta, essendoci una accelerazione
ora si può definire la FORZA centripeta:
\[
F_{c} = m\cdot a_{c} = m\cdot \omega^{2}\cdot r
\]	
\[
F_{c} = 4 \cdot \pi^{2} \cdot \frac{m_{T}}{k_{T}} \cdot \frac{1}{r^{2}} = F_{TS}
\]	
La T indica letteralmente la Terra, mentre invece
$F_{TS}$ sarebbe la forza esercitata dalla Terra sul Sole, mentre nel caso della 
forza esercitata dal Sole sulla terra:
\[
F_{ST} = 4\cdot \pi^{2} \frac{m_{S}}{k_{S}} \cdot \frac{1}{r^{2}}
\]	
Quindi è possibile dire che la forza centripeta sia proporzionale ad 
$\frac{1}{r^2}$, e si ha una dimostrazione del PERCHE' ciò che ha detto Keplero
funziona. E non è tutto. \\ \\ 
Si nota che $F_{TS} = F_{ST}$, che quindi significa: 
\[
4 \cdot \pi^{2} \cdot \frac{m_{T}}{k_{T}} \cdot \frac{1}{r^{2}} = 
4\cdot \pi^{2} \frac{m_{S}}{k_{S}} \cdot \frac{1}{r^{2}}
\]
Definiamo inoltre un valore $\gamma = \frac{4\pi^2}{m_{T}\cdot k_{S}} = 
\frac{4\pi^2}{m_{S}\cdot k_{T}}$ E da qui ragionando su formule inverse di 
questa uguaglianza si può esprimere che:
\[
F_{ST} = \gamma\cdot \frac{m_{S}m_{t}}{r^{2}} = F_{TS}
\]	
Si definisce $g = \gamma \frac{M_{Terra}}{R_{Terra}^{2}}$, e da qui consegue che
l'accelereazione lunare sia $a_{Luna} = \gamma \frac{M_{Terra}}{R_{Luna}^{2}}$
con M che indica la massa mentre R il raggio del corpo (in questo caso celeste
di riferimento).
\paragraph{Riporto il grafico di come funziona la velocità angolare rispetto
alla Terra: }
\begin{center}
\includegraphics[width=0.75\textwidth]{3}
\end{center}

\paragraph{Calcolo della velocità di un proiettile per non toccare mai terra:}
In un secondo è dimostrato che un proiettile scenda di 5 metri poichè 
l'accelerazione $\Delta h = \frac{1}{2}g\cdot t^{2}$. Ne consegue che
$\Delta L = 7900 \frac{m}{s}$
\paragraph{Considerando invece l'ISS: } che si trova  a 400km da terra, si 
osserva che l'accelerazione del corpo in caduta libera è di $g^{'} = 8,6 
\frac{m}{s^{2}}$
\paragraph{Spiegato in termini pratici: } Il corpo è in caduta libera MA, nel 
momento in cui si sposta si sposta con sè anche il pianeta stesso, e quindi
non cade mai direttamente, o meglio. Quando il corpo arriva nel punto in cui 
avrebbe dovuto esserci la collisione, il pianeta si è spostato, perciò continua
ad orbitare
\subparagraph{In pratica: } la velocità a cui viaggiano i satelliti permette
di fare in modo che rimanga costante anche la distanza dalla Terra	
\section*{To conclude}
Si è dimostrato che esista una forza che dipende dalla massa ed è in grado
di attirare a sè i corpi. Matematicamente si esprime $\overrightarrow{F_{1,2}} 
= -g_{N}\cdot \frac{m_{1},m_{2}}{R_{1,2}^{2}} \cdot \widehat{r_{1,2}}$, e
per il principio di azione-reazione $\overrightarrow{F_{2,1}}= - 
\overrightarrow{F_{1,2}}$
Mentre invece, da dove ciccia fuori questa $G_{N}$? 
\[
G_{N} = 6,7 \cdot 10^{-11} N \frac{m^{2}}{kg^{2}}
\]	
E la massa della terra è calcolabile partendo da $mg =
G_{N}\cdot \frac{M_{Terra}m}{R^{2}_{Terra}}$, eliminando m che compare da una 
parte e dall'altra si ottiene la massa terrestre
\section{Densità}
La densità indica la quantità di materia distribuita in un volume, o meglio.
quanta materia c'è in un determinato volume, e si calcola con:
\[
\rho = \frac{M}{\frac{4}{3}\pi \cdot R^{3}}
\]	
In cui $\frac{4}{3}\pi\cdot R^{3}$ indica il volume della sfera per poter calcolarne la
densità. (Siccome stiamo considerando una sfera) ALTRIMENTI in generale lì 
si indica il volume, riscriviamola
\[
\rho = \frac{M}{Volume}
\]	
Letteralmente $\frac{Massa}{Volume}$
Se la densità della terra è a simmetria sferica (cioè dipende dalla posizione
della sfera), allora è corretto dire $g = 9,8 \frac{m}{s^{2}}$. Ma sulla terra
esistono mari e montagne, chiaramente la forza g può cambiare se uno si trova
in un determinato posto.
\paragraph{Inoltre }La terra non è una sfera perfetta, è un geoide (Una sfera 
schiacciata, un pallone dopo che ti ci sei seduto sopra)
\paragraph{Esempio: }
Passando dall'equatore al polo, quant'è la differenza di accelerazione?
Sappiamo che
\begin{itemize}
	\item R = 6400km
	\item $\delta R = 21km$
	\item $\frac{\delta g}{g} \sim -2 \frac{2\delta R}{R} = -2\cdot \frac{21km}{6400km}
	\to \delta g \sim - 0.06 \frac{m}{s^{2}}$
\end{itemize}
\paragraph{Altro esempio: }
\begin{center}
\includegraphics[width=0.75\textwidth]{4}
\end{center}
Se un punto materiale si muove lungo una circonferenza, deve per forza esserci 
una forza che lo muove. Un'accelerazione centripeta ($\frac{V^{2k}}{R}$). 
Prendo una massa m, la attacco ad una molla, quando è in equilibrio deduco che
la forza che applica la molla è uguale alla mossa che attrae la massa verso il 
basso. La forza che si oppone alla molla la interpreto come forza peso, e la
considero come $F_{P} = mg_{P} = \kappa x$. Se questo corpo sta fermo si deduce
che $\Sigma F = 0 \to a = 0$, supponendo$ F_{G} = mg_{P}$ allora $F_{G} > F_{P}$,
per cui $F_{apparente} - mg_{P} = m\cdot a_{Centripeta}$. Ragioniamola più in
termini pratici: 
\[
m\cdot \omega^{2} R_{T} = G_{N}\cdot \frac{M_{T}\cdot m}{R_{T}^{2}}-mg_{P} 
\]	
In cui:
\[
w^{2}R_{T} = g - g_{P} ~~e~~ g_{P} = g - \omega^{2}R_{T} 
\]
\paragraph{Cosa si deduce? }Il nostro punto materiale di massa m, siccome si 
trova su una circonferenza e si muove con una velocità non trascurabile, allora
è sottoposto ad una forza centripeta
\subparagraph{Spoiler: }Io francamente non ci ho capito molto, nel senso, non ho
trovato una vera utilità per tutto sto casino, forse è solo un ripasso di formule,
boh.	
\chapter{Le forze conservative}
Sono le forze per cui il lavoro risulti indipendente dal percorso. Tutto ciò
che conta è solo punto iniziale e finale
\[
\Delta U_{AB} = -L_{AB}
\]
Ad esempio la forza peso è conservativa, conpie un lavoro che dipende dalla 
quota iniziale e finale $L = \int_{y_{1}}^{y_{2}} m\cdot g ~ dy  $, pertanto
$ \Delta U = -L = mgh $ e per piccoli spostamenti sulla Terra possiamo anche
dire che g sia costante. 
Come avevamo già visto, $L = \int_{A}^{B} \overrightarrow{F}(distanza)\cdot d 
\overrightarrow{s}$
\paragraph{In pratica: }Se hai un lavoro nullo prodotto da una forza su un corpo
lungo un cammino chiuso, allora quella forza si dice \textbf{Conservativa}.
\subparagraph{Mi spiego peggio: }Prendi una palla da tennis, vai in mezzo ad un 
prato e lanciala verso l'alto. Bene, la palla parte da un punto A e arriva fino 
ad un punto B, beeene. (Ora, fate finta di avere una precisione buona)La palla 
ad un certo punto raggiunge il suo apice, (B), poi ripercorre a ritroso
la strada e torna in posizione A. \textbf{Il lavoro} che produciamo noi per 
farla andare da A a B, ed il lavoro della forza peso per far tornare la palla da
B ad A sono identici. 
\subparagraph{E quindi? }Sei in un percorso chiuso in cui posizione iniziale e 
finale coincidono, ergo lavoro = 0. Perciò la forza conservativa è quella che se
il lavoro compiuto da essa su un corpo che percorre un cammino lungo, è uguale a 0. 		
\paragraph{Perchè ci serve? }Perchè in pratica permette di classificare le forze
dal punto di vista dell'energia. Ci serve più nello specifico per capire meglio
l'energia potenziale
\section{Cammino percorso}
Siccome si considera una forza conservativa, e si sa che il lavoro sia nullo, 
allora: 
\[
L_{AB} + L_{BA} = 0
\]	
A questo punto riprendendo la definizione di lavoro generale, si può
dire che: 
\[
\overbrace{\int_{A}^{B} \overrightarrow{F} \cdot d \overrightarrow{s}}^{Percorso 1}
= - \overbrace{\int_{B}^{A} \overrightarrow{F} \cdot d \overrightarrow{s}}^{Percorso 2}
\]    
Perciò il lavoro di una forza conservativa da A a B è lo stesso a prescindere dal 
percorso che si sceglie per andare da A a B. Quindi si hanno queste due 
importanti proprietà: 
\begin{itemize}
	\item Lavoro nulla su un percorso chiuso: 
	\[
	\oint_{\gamma} \overrightarrow{F} \cdot d\overrightarrow{s} = 0
	\]
	\item Indipendenza del lavoro dal cammino percorso: 
	\[
	\overbrace{\int_{A}^{B} \overrightarrow{F} \cdot d \overrightarrow{s}}^{Percorso 1}
	= - \overbrace{\int_{B}^{A} \overrightarrow{F} \cdot d \overrightarrow{s}}^{Percorso 2}
	\]    
\end{itemize}
\chapter{Preparazione per il primo parziale}
In questo capitolo vi riporto delle nozioni utili per il primo parziale:
\section{Costante di Hook}
Le unità di misura della costante elastica della molla \textbf{k} (Costante di 
Hook) sono $\frac{kg}{s^{2}}$ 
\section{Il lavoro della forza peso}
E' una precisazione futile, ma va fatta, date due sfere di massa \textbf{m} ed 
\textbf{M} (Sì, \textbf{M} > \textbf{m}, obv), se queste ultime vengono allontanate
l'una dall'altra, ciò che è il lavoro compiuto dalla forza peso, siccome si 
oppone all'allontanamento, allora è NEGATIVO.
\section{Le componenti di una forza}
Data una qualsiasi forza, o più semplicemente un vettore, quando indichiamo le 
sue componenti, stiamo dando due coordinate sul sistema cartesiano, che rispecchiano
un punto A, che sarebbe il vettore. 
\paragraph{Vediamo un esempio:}
\subparagraph{Da YouMath:}	
\begin{center}
\includegraphics[width=0.75\textwidth]{componenti}
\end{center}
\chapter{I fluidi (SECONDO PARZIALE)}
Finora si è ragionato sulla meccanica classica, e si è studiato come descrivere
il moto di un punto materiale (un oggetto rappresentato da un punto senza
dimensioni), si è descritto il moto etc. Insomma, se siete qui probabilmente 
avrete letto anche tutto il resto. 
\section{Cos'è un fluido}
Un fluido è un insieme di "particelle" distinte, più precisamente liquidi e gas.
Questi ultimi non hanno una forma precisa, non sono corpi rigidi, perciò saranno
soggetti a delle regole particolari. 
\paragraph{Più in generale: }
Hai diversi stati di materia (Solido, liquido, gas, plasma), cioè la materia è 
la stessa ma con stati diversi, e questo stato è derivante da cosa accade a 
livello \textbf{microscopico}.
\paragraph{La Fluidodinamica: }
E' la branca della fisica che si occupa di studiare tutto quello che accade per
quanto riguardi i fluidi MA fermandosi prima del concetto di atomo, quindi non 
andrà a scavare troppo nel piccolo. Prendendo ad esempio un solido su un tavolo
se volessimo strisciarlo sopra NON otterremmo un attrito. Il liquido per dire
scorre su se stesso, l'unica resistenza che c'è è la viscosità (Lo scorrimento
degli strati uno sull'altro è più veloce o lento per via di una specie di attrito
la viscosità). Noi ci occuperemo di liquindi che hanno una viscosità trascurabile.
\section{Le unità di misura}
Per quanto concerne i fluidi non ha senso parlare di massa, nè di volume, in 
questo caso si ragionerà su quantità \textbf{intensive}, prese "per unità di".
Quali sono? 
\begin{itemize}
	\item Densità
	\item Temperatura
	\item Pressione
\end{itemize}
Hanno un'intensità NON DIPENDENTE dalle dimensioni dell'oggetto di cui si parla
Supponiamo di voler ragionare su un determinato volume avremo che $\Delta V \to 
\Delta M$, cioè c'è una proporzionalità tra massa e volume tale che: 
\[
\rho = \frac{\Delta M}{\Delta V}
\]
Con $\rho$ che sarebbe la densità.
\paragraph{I liquidi non possono esser compressi: }Pertanto la densità sarà 
uguale e costante in ogni punto. Però la definizione corretta di densità è 
differenziale OVVERO c'è da prendere il limite tale per cui 
\[
\rho = \frac{d \Delta M}{d \Delta V} ~ con ~ \rho(x, y, z)
\]    
L'unità di misura è: $[\rho] = \frac{kg}{m^{3}}$
\section{La pressione}
Per definizione, la pressione è la forza perpendicolare applicata alle pareti 
fratto la superficie. 
\[
P = \frac{\Delta F_{\perp}}{\Delta S}
\]
PERO' solo se è costante, ma siccome la forza non è necessariamente costante
sulla superficie, allora
\[
P = \frac{dF_{\perp}}{dS}
\]
\[
[P] = \frac{N}{m^{2}} = Pa (Pascal)	
\]	
\[
1 ~ atm ~ (atmosfera) = 1.01 \cdot 10^{5} Pa
\]
\[
1 bar = 10^{5} Pa \approx 1 atm
\]
Più semplicemente la pressione è una forza applicata ad una superficie. ($
Con F_{H} = -k\Delta x $)La pressione su una superficie è la componente \textbf{
perpendicolare
} di una forza. \textbf{ATTENZIONE} Non è un vettore, si considera solo il 
modulo della componente verticale. (La forza è un vettore)
\section{Legge di Stevino}
Prendiamo un contenitore con una sua superficie, un asse di riferimento che
chiamiamo asse y, fissano $y_{0}$ sul livello del fluido. Tutto ciò che vien 
sotto è negativo. 
Consideriamo ora un solido e posizioniamolo immerso nel liquido di riferimento.
Il fluido è un fluido in quiete, è fermo. Noteremo che il solito avrà un'altezza
$y_{1} $ ed un'altra $y_{2} $ ne ricaviamo abbastanza easy che h = $y_{1} - y_{2}$.
Ora, questo volume avrà una pressione sulle sue superifici ideali, pertanto ci 
saranno due forza che picchiano su tutte le superfici.
\begin{center}
\includegraphics[width=0.75\textwidth]{5}
\end{center}
A questo punto iniziamo a ragionare in termini \textbf{ESTENSIVI}, cioè andiamo
a ragionare su qualcosa di diverso dalle dimensioni delle oggetti. 
\begin{center}
\includegraphics[width=0.75\textwidth]{6}
\end{center}
\paragraph{La legge di Stevino} dice appunto che la pressione dipenda davvero 
dalla profondità 
\[
P_{2} = P_{1} + \rho \cdot g(y_{1} - y_{2})
\]
E quindi 
\[
\Delta P = \rho \cdot g \cdot h 
\]
Riprendendo l'altezza si ha che $\Delta y = y_{2} - y_{1}$ e quindi
\[
\Delta P = -\rho \cdot g \cdot \Delta y
\]
In via invece infinitesimale si può dire che:
\[
\frac{dP}{dy} = -\rho \cdot g
\]
In superficie la pressione che "sentiamo" è quella atmosferica, mentre per 
esempio quando ci immergiamo in una piscina, più si scende più va sommato un 
termine lineare con la profondità OVVERO $P = P_{0} + \rho \cdot g \cdot h$
\paragraph{Contenuti speciali:}
\subparagraph{Esistono liquidi ideali? }No, è come chiedersi se esistono i punti
materiali
\subparagraph{Come si misura la pressione? }Con il Manometro
\section{Misura della pressione atmosferica}
E' il primo dei casi in cui possiamo applicare la legge di Stevino. Torricelli 
prese un tubo e lo riempì di mercurio (Che è un metallo liquido a temperatura 
ambiente). 
Il mercurio (Hg) è infatti il primo liquido usato per misurare la pressione. 
\[
\rho_{Hg} = 13.6 \frac{g}{cm^{3}} ~ e ~ \rho_{H_{2}O} = 1 \frac{g}{cm^{3}}
\]
\begin{center}
\includegraphics[width=0.75\textwidth]{7}
\end{center}
Ne consegue che 1 Tor = 1 mm di Mercurio E 1 Bar sono 760 Tor.
Ora, a questo punto le stesse formule applicate in questo caso su dei liquidi
sono anche applicabili a corpi aeriforme. 
\section{La pressa idraulica}
Avete presente quando siete a letto, che state prendendo sonno, ma c'è il telefono
lì di fianco a voi sul comodino, con sopra YouTube (Va bene pure Instagram, e sì,
pure Reddit MA NON 4Chan), 
ecco. Bene. Iniziate
a guardare un video di qualcosa che vi interessa, 5 video dopo vi ritrovate a 
guardare video di presse idrauliche. Ecco quelle.
\begin{center}
\includegraphics[width=0.75\textwidth]{8}
\end{center}
\section{Principio di Archimede}
\paragraph{Enunciato: }Il principio di Archimede afferma che «ogni corpo immerso 
parzialmente o completamente in un fluido riceve una spinta verticale dal basso
verso l'alto, uguale per intensità al peso del fluido spostato».
\paragraph{Spiegato peggio: }
Ho una vaschetta d'acqua, ci butto dentro un oggetto avente un peso p=mg. La 
forza applicata al corpo che dovrebbe portarlo verso l'alto si chiama $F_{B}$,
(Bob, rimanere in superficie)
\[
F_{B} = \rho _{Liquido} \cdot V \cdot g = P_{Liquido}
\]
In cui la V è il volume, g.. è g, e P sarebbe la forza peso del liquido
\section{Il fluido ideale }
\begin{enumerate}
	\item Ha una densità costante $\rho = costante$
	\item Incomprimibile
	\item Il fluido è stazionario, cioè prendendo un punto in un fluido, la 
	velocità è costante nel tempo (in modulo e verso)($\overrightarrow{v}(x,y,z)
	= costante$)
	\begin{itemize}
		\item Cioè alla fine la velocità è costante e NON dipende dal tempo. 
		\item Questa cosa in realtà va di pari passo con il concetto di 
		\textbf{Laminare}, cioè NON ha turbolenze 
		\item Le traiettorie non si incrociano, cosa invece che nei fluidi 
		turbolenti accade insomma.
	\end{itemize}
	\item NON è \textbf{viscoso}, nel senso che gli attriti interni all'interno
	del fluido sono trascurabili, non vengon considerati
	\item Irrotazionale, cioè opposto di rotazionale, non c'è una rotazione 
	insomma
\end{enumerate}
Il moto di un fluido è descritto in maniera utile per la comprensione con le 
\textbf{linee di flusso}, o anche \textbf{linee di corrente}. 
Essendo che il flusso è stazionario, ogni volta che un elemento passa per questo
flusso si avranno una serie di velocità vettoriali. Se prendo un elemento di 
flusso che passa da un punto e lo seguo, l'insieme dei punti per cui passa è la
linea di flusso (La velocità è tangente alla linea). 
\paragraph{Quando abbiamo un flusso stazionario: }Dato un punto, di lì passerà
una sola linea (perchè non sono presenti intersezioni). 
\section{Tubo di flusso}
E' la conseguenza del fatto che non vi siano intercezioni, pertanto vien fuori
un grafico di questo tipo:
\begin{center}
\includegraphics[width=0.75\textwidth]{9}
\end{center}
\section{Equazione di continuità}
E' la conseguenza dell'incomprimibilità del nostro fluido (Prendete l'esempio
della siringa)
\paragraph{Cosa ci dice? }Ciò che c'è dentro in qualche maniera deve uscire 
(Prendiamo una siringa non bucata), perciò quindi noi premiamo sulla siringa (
siccome mi impressiono ipotizzio che non ci sia un ago e che ci sia acqua.. deh)
insomma, io premo con una velocità, ma l'acqua esce ad una velocità molto maggiore.
\subparagraph{La canna dell'alqua? }Ricordate quando per lavare gli amici andavamo
a tappare l'uscita dell'acqua? Accadeva per questo magico effetto
\subparagraph{Lo champagne sui podi del motorsport: }Stessa roba

Bene ora ragioniamo in termini però più pratici: 
\begin{center}
\includegraphics[width=0.75\textwidth]{10}
\end{center}		 
Ora quello che accade è che in un tempo $\Delta t$ un tot d'acqua si sposta, 
con una determinata velocità, e si ha un tot di volume dentro il liquido, bene
dall'altra parte lo stesso liquido dovrà uscire, k? Sì, sempre proporzionalmente
alla velocità con cui vien spinta dall'altro lato. 
\paragraph{Il fluido non si può comprimere} Perciò nel nostro tubo accade che il 
fluido compreso tra entrata e uscita NON CAMBIA perchè non aumenta nè diminusice
la densità del liquido. In definitiva la quantità che entra deve essere la 
quantità che esce.
\begin{center}
\includegraphics[width=0.75\textwidth]{11}
\end{center}
Si conclude dicendo che $v_{1} = v_{2}$, no? perciò $A_{1}v_{1} = A_{2}v_{2}$.
Per formule inverse accade quindi che $\frac{v_1}{v_2} = \frac{A_1}{A_2}$
Si ottiene la \textbf{Portata Volumetrica}:
\[
R_{V} = Av	
\]
L'unità di misura è $[R_{v}] = \frac{m^{3}}{s}$
\[
R_{m} = \rho R_{v} = \rho Av
\]
In questo caso invece l'unità di misura è:$[R_{m}] = \frac{kg}{s}$
\paragraph{Si osserva: }Esiste una vera e propria proporzionalità tra velocità 
e dimensione del buco perciò:
\[
v ~ \alpha ~ \frac{1}{A} ~~~~~~~~~ E ~~~~~~~~~ n_{L} ~ \alpha ~ \frac{1}{A}
\] 
\section{Equazione di Bernoulli}
Prendo un esempio fatto non male dal webbe: 
\begin{center}
\includegraphics[width=0.75\textwidth]{12}
\end{center}
Allora
\begin{itemize}
	\item h1 e h2 sono precisamente le linee \textbf{centrali}, cioè sono 
	precise in mezzo al fluido
	\item P1 e P2 sono due pressione
	\begin{itemize}
		\item Anche se non specificato va da sè che la pressione è da intendersi
		come $\frac{F}{A}$
	\end{itemize}
	\item A1 e A2 indicano le aree che rappresentano la dimensione dei buchi
	\item Ricaviamo F1 ed F2, contando (dalla pressione) che F = P $\cdot$ A,
	siccome ne abbiam due diverse vi scrivo la formula generale 
	$F_{i} = P_{i}$ $\cdot$ $A_{i}$ 
	\item La densità? $\rho$ ma rimane costante, solo che mi piace esteticamente
	$\rho$ quindi mi va di scriverla a caso. E siccome mi piace $\lambda$ scriverò
	un $\lambda$ a caso, perchè mi va :C
\end{itemize}
Ora passiamo a calcolare il lavoro partendo da questi dati:
\[L_{i} = \overrightarrow{F_{i}}\cdot\Delta \overrightarrow{l_{i}} = P_{i} 
\cdot A_{i} \cdot \Delta l_{i} \]

Quindi ne consegue che dato che abbiamo due lavori da trovarci, verrà furi che:
\begin{enumerate}
	\item $L_{1} = \overrightarrow{F_{1}}\cdot\Delta \overrightarrow{l_{1}} = 
	P_{1} \cdot A_{1} \cdot \Delta l_{1} $
	\item $L_{2} = \overrightarrow{F_{2}}\cdot\Delta \overrightarrow{l_{2}} = 
	P_{2} \cdot A_{2} \cdot \Delta l_{2} $
\end{enumerate}
Ora, noi sappiamo che $L_{G} = -mg(y_2 - y_1)$ (Ricordando che $m = \Delta V \rho$)
perciò ricaveremo che $L_{1} + L_{2} + L_{G} = \Delta K$ (Sì, L g è il lavoro 
della gravità, siccome so che me ne dimenticherò pure io, lo ripeto)
\paragraph{A cosa è uguale K? } $K_{i} = \frac{1}{2}mv_{i}^{2}$
Anche qui, stesso discorso di prima per la i, perciò $\Delta K = K_{2} - K_{1}$
\paragraph{Ora arriva il casino vero: } Perchè ora andiamo ad unificare tutto 
sto casino in una formula unica:
\[
-\rho \Delta v\cdot g(y_{2} - y_{1}) + (P_{1} - P_{2})\Delta v  = 
\frac{1}{2}\rho \Delta v (v_{2}^{2} - v_{1}^{2})	
\] 
Si semplifica il $\Delta v$
\[
\rho g \cdot (y_1 - y_2) + (P_{1} - P_{2}) + \frac{1}{2}(v_{1}^{2} - 
v_{2}^{2}) = 0
\]	

Insomma ne emerge in qualche modo che 
$P_{1} + \frac{1}{2} \rho v_{1}^{2} + \rho g \cdot y_{1} = P_{2} + \frac{2}{2} \rho
v_{2}^{2} + \rho g \cdot y_{2}$
Ne segue che
\[
P + \frac{1}{2} \rho v^{2} + \rho \cdot g \cdot y = COSTANTE
\]	
\chapter{Termodinamica}
Si occupa di trasferimenti di energia sotto forma di lavoro e \textbf{calore}
che è una quantità che studieremo in questo capitolo. Va ricordato come 
funziona il meccanismo di conservazione dell'energia. 
\paragraph{Dal punto di vista microscopico} è pressochè impossibile capire in 
che modo interagiscono gli atomi e le molecole di due corpi che per ipotesi
collidono. Ci son però due possibilità per trattare questi micro punti materiali,
(Micro perchè son punti materiali che fan riferimento a quantità microscopiche),
una è dal punto di vista statistico (che descrivono quantità medie) collegate a
delle quantità che sono osservabili macroscopicamente (pressione, temperatura 
etc).
\section{Dal punto di vista macroscopico}
La prima cosa da fare è definire cos'è un sistema (in questo caso termodinamico)
, ossia una porzione di "universo" che in qualche modo isoliamo dal resto, e 
prendiamo in esame/considerazione. Un sistema scambia energia e lavoro con il 
mondo esterno, avrà delle trasformazioni, ovviamente, e la termodinamica studia
queste trasformazioni.
\subparagraph{}Un sistema può pure essere contenente solidi, liquidi e gas tutti
assieme, è un sottoinsieme dell'universo alla fine. 
\paragraph{Un sistema è inserito in un ambiente} e quindi ci sarà una "interfaccia"
che collega il nostro sistema all'ambiente esterno.
\paragraph{Sistema aperto: }E' un sistema che permette entrata ed uscita di 
materia. Si ok, in che senso?
\subparagraph{La pentola a pressione} è un sistema chiuso, nel senso che la 
materia da dentro la pentola non esce (ok dopo un po' potrebbe esplodere). 
La pentola aperta, invece questa è un sistema aperto (avete presente quando 
per via dei 200 chili di sale so forma la schiumetta bianca? Ecco)
\paragraph{Sistema isolato: }Isolato è letteralmente quello che appare, cioè 
un sistema che non può aver alcun legame con l'esterno, isolato per l'appunto. 
\textbf{OCCHIO A NON CONFONDERE}, all'interno può pure succedere il finimondo, 
ma da dentro a fuori e viceversa
no. (Avete presente la \textbf{Subaru Baracca} isolata da dentro per fuori e non
dentro
per dentro? Ecco stesso discorso). 
\paragraph{Concentrandosi sui Gas}: \\
\textbf{Definiamo le variabili che useremo d'ora in poi:}
\begin{itemize}
	\item P: Pressione
	\item V: Volume
	\item T: Temperatura
	\item n:
\end{itemize}
\section{Equazione di stato}
Ogni volta che si ha una terna di P, V e T si ha un'equazione di stato di un 
sistema. f(P, V, T), visto che sto studiando pure LISP la scrivo così: (f P V T).
E' la \textbf{Relazione} che lega le variabili all'equilibrio (Sì, varibili 
termodinamiche per sistema termodinamico)
\textbf{A che serve?} Serve per definire gli stati di equilibrio
\paragraph{Cos'è?} Un sistema è in equilibrio se tutte le variabili che lo 
descrivono sono \textbf{COSTANTI} nel tempo. 
\paragraph{L'equazione di stato} è applicabile però su un sistema in particolare,
quindi non è universale, dipende come al solito dal sistema, tra l'altro non 
ragioneremo nemmeno dal punto di vista chimico
\section{Temperatura}
E' un casino, perchè la temperatura è un concetto molto relativo, nel senso
quello che definiamo noi come "\color{red}caldo\color{black}" o "\color{blue} 
freddo\color{black}
" varia in base allo scambio termico con l'oggetto che consideriamo. Tutto va in
funzione della rapidità con cui avviene il trasferimento di calore, perciò noi
percepiamo più caldo con il crescere della velocità del trasferimento. 
\paragraph{Esempio: }
Prendiamo 3 sistemi: \{A, B, C\} aventi le loro 3 temperature: \{$T_{A}, T_{B}
, T_{C}$\}, supponendo che A e C entrano in contatto, allora a un certo punto
si raggiungerà l'equilibrio (Ossia, $T_{A} = T_{C}$, il più caldo si raffredda
e il più freddo si scalda), ora attacchiamo C a B, 
cosa notiamo? C andrà in equilibrio con B, quindi c'è una relazione \textbf{
Transitiva
} tale che SE $
\begin{cases}
SE ~ A ~ e' ~ in ~ Relazione ~ con ~ B, \\B ~ e' ~ in ~ Relazione ~ con ~
C, \\ALLORA ~ A ~ e' ~ in ~ Relazione ~ con ~ C 
\end{cases}$ 
\subparagraph{L'unità di misura: }E' il Kelvin [T] = K ed è di valore $\in (0, 
+\infty)$, in pratica misura il livello di agitazione delle molecole, oppure 
anche solo il livello di "movimento", perciò sì, 0 sta per fermo, e meno fermo
che fermo non può essere un corpo quindi easy. 
Ora, noi ragioniamo in Celsius, ed abbiamo come centro gli $0^{\circ}C$ 
che sarebbero 
coincidenti a 273,15 K (Perchè? E' la temperatura a cui ghiaccia l'acqua). 
\paragraph{Definiamoci $T_{0} e T_{1}$} tali che:
$
\begin{cases}
T_{0} = 0^{\circ}C	
T_{1} = 100^{\circ}C
\end{cases}
$
Sapendo che T = aX + b allora possiamo dire che:
\[
\begin{cases}
0^{\circ}C	= a X_{0} + b
100^{\circ}C = a X_{1} + b
\end{cases}
\]
Questa X da dove ciccia fuori? Che cosa sarebbe? Sarebbe la pressione, infatti
c'è una relazione tra temperatura e pressione tale che X = X(T) e T = T(X),
perciò è possibile stabilire che T = T(X) = a X (con a che sarebbe un coeffieciente)
\section{Equazione di stato dei Gas PErfetti}
\[
PV = nR\cdot T
\]
Cos'è un gas perfetto? E' un gas ideale fatto di corpi che interagiscono impattando
gli uni sugli altri (Niente vibrazioni, niente reazioni o azioni particolari).
Non solo, nel senso ogni particella è una sfera perfetta, perciò alcune molecole
tipo quella dell'acqua vanno un po' approssimate.
\textbf{Supponiamo un gas perfetto}, e vogliamo trovare un legame tra le sue 
variabili che possano descriverlo (Che saranno le equazioni di stato), riprendendo
l'esempio della pentola a pressione, hai una sorgente costante di  calore 
che proviene da sotto, e 3 pareti isolanti (sopra e lateralmente). Supponiamo 
ora che nella pentola hai una massa M di un gas (l'elio che è un gas quasi perfetto)
, dopo un po' il sistema va in equilibrio, nel senso che la temperatura che 
c'è nella parete inferiore della pentola poi diventa la temperatura del gas.
Si ottiene che
\[
PV = costante = K, \\
K = K(m, T)
\]
\section{Legge di Boyle}
In termodinamica e chimica fisica, una equazione di stato è una legge 
costitutiva che descrive lo stato della materia sotto un dato insieme di 
condizioni fisiche. 
\paragraph{}Fornisce una relazione matematica tra due o più variabili
di stato associate alla materia, come temperatura, pressione, volume o energia
interna. Le equazioni di stato sono utili nella descrizione delle proprietà 
dei fluidi, dei solidi e persino per descrivere l'interno delle stelle.
\begin{center}
\includegraphics[width=0.75\textwidth]{13}
\end{center}
P = $\frac{K}{V}$, \\
K = am = a(T)m \\
A questo punto ci si accorge che non è la massa la quantità che ci interessa MA
è il numero di atomi componenti il gas
\section{Legge di Avogadro}
Date V, T e P, c'è un valore n costante che è il numero di atomi che compongono
il gas in questione (Atomi, ma potrebbero essere molecole, ioni, siamo in fisica,
non chimica). 

Ripetendo \[PV = a(T) \cdot N\]
Questa N, citata anche sopra è letteralmente il numero di molecole (Facciamo 
generiche particelle se no mi confondo pure io), e una particella come è 
costituita? \[N = \frac{Massa_{Totale}}{Particelle}\]
\section{Concetto di mole}
Misura la quantità di "sostanza" ma NON come la massa, ma comenumero di atomi, 
è un po' diverso diciamo. La massa è quantità di materia, la mole è quantità 
di elementi che compongono la materia. \\
Una mole contiene lo stesso numero di particelle del numero di atomi esistenti
in 12 grammi di $^{12}C$ (Carbonio 12) ed è pari al numero di Avogadro ($6,023 
\cdot 10^{23}$ atomi).
Da qui consegue che K = n$\cdot$ A(T), e quindi dipende dal numero di moli, che
sarebbe $n = \frac{N}{N_{Avogadro}}$ 
\paragraph{Quando è la massa di una mole? }Dipende dalla molecola che si 
considera, nel senso, dipende da cosa compone la molecola etc. Generalmente
si dice che \[1 amu = \frac{1}{12} ^{12}C\], ossia un dodicesimo della massa 
del carbonio.
\paragraph{Digressione: }A causa del fatto che nel carbonio 12 ci siano 6 
protoni e 6 neutroni, dovrebbe essere 12 MA in realtà è un po' meno, poichè in
pratica c'è una legge: $E = mc^{2}$ che dice che c'è dell'energia che vien 
spesa dalle molecole per attrarsi, e quest'energia fa sì che diminuisca la massa.
Ma a noi indovinate quanto ci importa... Esatto, una bellissima *\textit{Censura}*
\paragraph{Riassumendo} $M(^{12}C = \frac{12}{N_{Avogadro}}$, La massa molare in amu è
$M_{A} \cdot N_{A}$  mentre in grammi si ha $\frac{1}{12} \cdot \frac{12}
{N_{Avogadro}}$ = $1_{N_{A}} grammi$
Perciò è possibile dire che la massa di $^{12}C = 12 amu$, e da qua ne consegue
che è possibile calcolarsi le masse di altri atomi:
\begin{itemize}
	\item L'ossigeno (O) ha massa 15.9994 amu
	\item L'idrogeno (H) ha massa 1.00794 amu
	\begin{itemize}
		\item Sì, la massa della molecola d'acqua è $\sim$ 18 amu
		\item Bisogna letteralmente fare la somma di due H + 1 O 
	\end{itemize}
\end{itemize}            
Quando prendiamo la tavola periodica degli elementi infatti si trova in basso 
a destra la massa atomica, mentre invece in alto a destra si ha il numero di
protoni e neutroni, la riporto qui in basso per dare un che di informativo:
\begin{center}
\includegraphics[width=0.75\textwidth]{Nome immagine}
\end{center}









































\end{document}




