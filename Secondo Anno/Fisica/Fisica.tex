\documentclass[12pt, a4paper, openany, oneside]{book}
\usepackage[italian]{babel}
\usepackage[T1]{fontenc}
\usepackage[utf8]{inputenc}
\usepackage{amsmath} 
\usepackage{xcolor}
\usepackage{hyperref}
\usepackage[margin=1in]{geometry} 
%usepackage[latin1]{inputenc}
\begin{document}
\pagestyle{headings}
\author{DaveRhapsody}
\title{Fisica}
\date{30 Settembre 2019}
\maketitle
\tableofcontents
\chapter{Introduzione al corso}
Non è presente materiale didattico, le lezioni sono architettate in modo che
si segua dalla lavagna, è consigliato dal prof stesso di usare gli appunti od i
libri che (per coloro che han fatto fisica) si usavano alle superiori.
\\ \\
Il programma è \textbf{tutta la fisica} in generale, ma affrontata in modo 
semplice, quasi banale, l'ultimo argomento dovrebbe essere il magnetismo, 
immaginatevi
quanto (non) si farà di quell'argomento. Ci sono 5 appelli in un anno, il primo
sarà a gennaio, poi febbraio, giugno, luglio e settembre, MA Gennaio e 
Febbraio dell'anno dopo sono inclusi
\\ \\
Il che significa che io posso fare i due parziali e poi fare l'orale anche a
Febbraio. Noi possiamo iscriverci solo allo scritto, e verremo spostati 
all'orale SE siamo già sufficienti. 
\section*{Alcune osservazioni}
Lo studio della fisica nasce dall'osservazione di una serie di fenomeni che 
accadono, con lo scopo di misurarli ed infine dimostrare il perchè questi
si verificano, 
\\ \\
Esistono una serie di \textbf{modelli} che sono in grado di descrivere ciò che
noi vediamo, ad esempio quando vedremo il moto, noi diremo "Osserviamo il moto
di un corpo", con corpo inteso come punto. Il punto è un oggetto di dimensioni
infinitesimali, e nel caso del moto ne analizzeremo i dettagli in modo 
specifico.
\\ \\
La nostra teoria parte da un modello semplificato che consente di capire il 
funzionamento di ciò che abbiamo di fronte. Nel caso dei Gas ad esempio ci 
saranno arricchimenti dei modelli (del tipo non esistono solo gas perfetti)
etc. \\ \\
Noi dobbiamo cercare di trovare il modello minimo, più semplice in grado di
\textbf{descrivere} una cosa. In fisica si adotta un atteggiamento \textbf{
Deduttivo}
, infatti non si ragiona generalmente in modo induttivo. Consideriamo che non 
esiste un modello finale che non si possa contradire.
\section{Cosa ci servirà}
Iniziamo definendo alcune quantità che ci interesseranno, ovvero massa, spazio 
e tempo.
\\ \\
C'è bisogno di capire che quantità si stia misurando, quindi si usano le unità 
di misura che cosa oggettivamente stiamo quantificando. Immaginatevi cosa 
significhi quantificare senza unità di misura. (Per dire Galileo usava i 
battiti del cuore.)
\\ \\
\subsection{Dal punto di vista numerico}
Si ha che in qualsiasi campo si ha un ordine di grandezza, ogni fenomeno ha la
propria scala da usare, ci saranno i coefficienti di riferimento, i prefissi
($\mu,~ mm, ~ n$), c'è un vero e proprio intervallo di grandezze ($10^{n}$).
\section{Notazione scientifica}
Ecco un esempio di numero scritto in notazione scientifica: 
\[
5 \cdot 10^{5} = 50000 \\
8 \cdot 10^{-1} = 0,1 
\]
\\ \\
Ragionando su come sono composti, abbiamo le cifre significative, ovvero cifre
che hanno senso di essere tenute in considerazione. In che senso? Se devo 
misurare un banco di scuola posso dire che è tipo 1034 mm, OPPURE dire che è un
metro e 34 millimetri.. E' la stessa cosa, ok, detta in modi diversi
\\ \\
Se specifico una cifra (tipo anche) lo 0 in un $0,12320$ esso è cifra significativa!
\\ \\
Se ho invece un numero tipo $1,010$ posso scriverlo in due modi
\begin{itemize}
	\item 1,011 +/- (Ok non so come si fa il + e - in \LaTeX) 0,001
	\item 1,01 
\end{itemize}
Nulla di estremamente complesso ma va detto comunque, per dire se ho $1,234567$
posso approssimarlo in $1,23457$. \\ \\
\paragraph{ATTENZIONE} Nel caso della \textbf{NOTAZIONE SCIENTIFICA} si tiene
in considerazione la parte numerica $\neq$ 0. tipo $123.000.000$ ha 3 cifre che 
sono proprio 123
\chapter{Cinematica}
E' la branca della fisica che si occupa di descrivere la traiettoria di un 
corpo, dovremo predirla, calcolarla, basandosi su un campo di forza, uno spazio
, introdurremo la forza in grado di cambiare il moto di un corpo MA per prima 
cosa
\section{Come definiamo la traiettoria di un corpo}
Definiamo la differenza tra grandezza scalare e vettoriale
\begin{itemize}
	\item Le grandezze vettoriali hanno con sè una direzione, un verso, ed un
	modulo definito anche intensità. L'esempio per eccellenza è lo spostamento 
	e la velocità.
	\item Le grandezze scalari sono valori precisi fissi, dei valori che indicano 
	qualcosa di quantitativo più che qualitativo.
\end{itemize}
\subsection{Esempio di grandezza vettoriale}
Supponiamo di avere due punti $x_{0} ~ e ~ x_{1}$ ponendoli distanti $\lambda$ tra 
loro. $\lambda$ sarà coincidente con $x_{0} ~ - ~ x_{1}$. Per definire il verso
basta osservare chi è il minimo tra $x_{0} ~ e ~ x_{1}$, lo si vede graficamente,
oppure osservando chi dei due è il maggiore. 
\\ \\
Da un lato abbiamo un vettore (ancora monodimensionale), ma abbiamo anche 
dato un piano dimensionale, per esprimere il concetto di vettore relativo alla
posizione del nostro punto.
\\ \\
Il sistema di riferimento è il sistema cartesiano, in questo caso Monoasse 
pertanto ci basta avere solo la $x$. $x_{0} ~ e ~ x_{1}$ sono semplicemente dei 
punti, ma hanno un nome specifico, in questo caso sono delle vere e proprie
posizioni.
\\ \\
Come si diceva prima, per capire il \textbf{Verso} bisogna osservare la differenza
tra $x_{0}  ~~ e  ~~ x_{1}$, se negativa allora va all'indietro, al contrario andrebbe 
avanti molto semplicemente
\subsection{L'esempio di una palla che cade in un piano inclinato}
Il nostro punto materiale è la palla, e per capire lo spostamento bisogna
tracciare un grafico che indica le posizioni lungo le quali la pallina passa,
quindi si semplifica tutto con un grafico a singolo asse.
\\ \\
Chiaro che se ho un modello \textbf{Dinamico} è un problemino diverso perchè
avrei anche forze tipo la grafità etc, ma per ora descriviamo questo moto.
\\ \\
La pallina parte dalla posizione $p_{0}$ e passerà per un $p_{1,2,3,4}$ aventi
una serie di tempi passati dall'istante 0 che si chiameranno $t_{1,2,3,4}$ etc.
\\
Per descrivere questo bisogna trovare una legge che sia in grado di esprimere
per qualsiasi istante quali possano essere le condizioni. 
$$
\begin{cases}
t_{\lambda} = tempo ~ richiesto ~ per ~ arrivare ~ dalla ~ posizione ~ p_{0} ~ a
~ p_{\lambda}  \\
\end{cases}
$$ 
\subsection{Alcune precisazioni}
\begin{itemize}
	\item Lo spostamento è la distanza in linea d'aria
	\item La distanza percorsa può essere nettamente maggiore, poichè è il 
	percorso specifico che vado ad effettuare
	\begin{itemize}
		\item Per intenderci, da A a B potrei dover passare per un punto C, 
		la distanza diventa minimo la somma di( A + B ) + (C + B), di conseguenza
		a meno che siano allineati, cambia già la distanza
		\item Se da A vado a B e torno indietro, la distanza percorsa è 2AB, 
		mentre lo spostamento vettoriale è 0
	\end{itemize}
\end{itemize}
Lo spostamento è vettoriale, la distanza percorsa è uno scalare
\section{La velocità}
E' la quantità di spazio(s) percorsa da un corpo in un determinato tempo(t),
specificando che ci sia la distanza percorsa e lo spostamento.
\paragraph{Dati due punti} Lo spostamento non è altro che un vettore che parte
dal primo al secondo punto, quindi che va da $p_{0}$ a $p_{1}$.
\\ \\
Attenzione, prima c'è da tenere conto della differenza dei tempi, che chiameremo 
$\Delta t ~ = ~ t_{Finale} ~ - ~ t_{Iniziale}$
\\
Abbiamo 3 velocità:
\begin{itemize}
	\item {Velocità media scalare: $v_{media} = \frac{distanza~percorsa}{\Delta
	t}$ che è la distanza percorsa sul tempo passato da quando son partito a 
	quando sono arrivato }
	\item {Velocità media vettoriale: $\vec{v} = \frac{\vec{\Delta x}}
	{\Delta t}$ con $\Delta x$ che è il vettore spostamento tra la posizione 
	$p_{iniziale}$ e $p_{finale}$ }
\end{itemize}
\subsection*{Osservazione:}
Ragionando per formule inverse, se voglio capire quanto ho percorso mi basta 
fare $d = \Delta t \cdot v_{media}$ , ma in realtà non è propriamente corretto. 
\\ \\
Se per esempio avessi qualcosa del tipo
$$
\begin{cases}
SE ~ t_{1} = 1 ~ E ~ x_{1} = 1  \\ 
SE ~ t_{2} = 2 ~ E ~ x_{2} = 4  \\ 
SE ~ t_{3} = 3 ~ E ~ x_{3} = 9  \\
SE ~ t_{4} = 4 ~ E ~ x_{4} = 16 
\end{cases}$$
Posso osservare che lo spazio percorso x(t) corrisponda all'accelerazione A*$t^{2}$
$$x(t) = At^{2}$$
Queste quantità sono vicine alle nostre esigenze quotidiane, oggettivamente lo
spostamento vettoriale non dice nulla, non ci permette di dire assolutamente
nulla durante uno spostamento. Ok, sì, la velocità media, ma in fisica non è che
conti poi così tanto.
\paragraph{Esempio}	
Prendiamo un percorso $\Delta x$ (differenza tra un $x_{0} $ e $x_{1}$ che
decidiamo noi) se io impiego un tempo $\Delta t$ (differenza tra un $t_{0} $ e 
$t_{1}$ che sono istanti di tempo diciamo) avrei: $$\frac{\Delta x}{\Delta t} 
= v_{media}$$ Ora il concetto è che non mi dice nulla di cosa accade nel mezzo
del tragitto. 
\paragraph{Ipotesi} Immaginate di avere istante tra i due che abbiam scelto,
se usassimo la velocità media, in un determinato istante, per via dell'approssimazione
potrebbe risultare che abbiam percorso più o anche meno chilometri, è troppo
impreciso MA
\\ \\
Più sono corte le distanze, o meglio, minore è il valore di $\Delta x$ e minore
sarà l'errore di approssimazione. Basti pensare alla media di un viaggio per
ipotesi da Milano a Roma, magari per un tratto vado a 150, ma in un altro per 
il traffico vado a 3 chilometri al millennio, la media è bassissima MA per via
di questi due picchi
\\ \\
Possiamo ricavare dalla nostra formula con il $\Delta x$ e $\Delta t$ che quindi
la posizione che si assume in un determinato istante sia:
$$x_{1} = x_{0} + v_{media} \Delta t$$
La velocità media però non indica praticamente nulla del moto, se mi servono dati
precisi (es contachilometri) su una determinata velocitá prendo intervalli sempre minori.
Quando il $\Delta t$ tende a 0, notiamo che la funzione non tenderà ad $\infty$ 
perchè c'è corrispondenza negli ordini di infinito.
Quindi chiamo questo limite con t che tende a 0 della velocità vettoriale fratto 
delta t = v $\lim_{\Delta t \to 0} \frac{\Delta \vec{x}}{\Delta t} = v_{istantanea}$
velocità istantanea, che non è altro che la velocità in un determinato istante.
%perchè in ognuno degli istanti ho una velocità v(t). \\ \\
Per ogni istante, per definizione di derivata, io calcolo alla fine 
$$v(t) = \frac{dx(t)}{dt}$$
Quindi in pratica otteniamo che la velocità istantanea è letteralmente la 
derivata della posizione, in cui la t è la "discriminante" della velocità istantanea che si 
aveva in un determinato istante (perdonate la ripetizione).
\\ \\ 
Con questa velocità istantanea possiamo (se applichiamo la legge oraria), 
calcolare in modo più preciso la posizione in un determinato istante! Come?
$$x_{t} = x_{0} + v_{t_{0}} dt$$
Cioè siamo arrivati che abbiamo la posizione iniziale e l'istante iniziale, più
la velocità istantanea (che è una derivata), ora ci basta solo applicare la 
formula. 
\\ \\ 
Prendiamo ora in esame un grafico che ha sulle ascisse il tempo e sulle ordinate 
le velocità istantanee registrate. Come nel caso precedente immaginiamo di avere 
un grafico con una funzione monotona crescente. Prendo due punti del grafico e
calcolo la velocità media tra essi 
$$\frac{v_{1} + v_{2}}{2}$$
sappiamo anche che la velocità media è definita come $\frac{\Delta x_{n}}{\Delta t_{n}}$
da cui ricaviamo $\Delta x_{n} = v_{med} \Delta t_{n}$.
Se ripetiamo questo procedimento per tutti i $\Delta$ t avremo:
%Dati due punti $p_{0}$ e $p_{1}$ in una curva, essi avranno quindi i corrispettivi 
%istantanei, quindi otteniamo un $\Delta x$ ed un $\Delta t$, prendendo questi 
%ultimi si avrà che $$\Delta x_{n} = v_{m_{n}} \Delta t_{n}$$ e quindi in 
%pratica per finire avremmo
$$\Delta x = \sum_{k=1}^n \Delta n = \sum_{k=1}^n v_{m_{n}} \cdot \Delta t_{n}$$
La somma dei $\Delta x$ ennesimi, è coincidente con la somma di tutte le aree
$A_{n}$ dove $A_{n} \sim \Delta x$. \\ \\
Se $\Delta t_{n} \to 0$ allora $\Delta x = \int v(t) dt$ \\
Per un punto specifico (sempre facendo tendere $\Delta t a 0$) ad esempio avremmo che 
$$x_{1} = x_{0} + \int_{t_{0}}^{t_{1}} v(t) dt $$
\\
{\color{black} \rule{\linewidth}{0.mm} }
\\
\paragraph{Precisazione di } \href{https://github.com/LiaBell47}{Giulia}:
La velocità va in funzione del tempo, MA gli estremi di integrazione vedendo il
grafico sono anch'essi dei tempi.
\\
{\color{black} \rule{\linewidth}{0.mm} }
\\
quindi per conoscere la posizione è sufficiente passare per un integrale definito della velocità.
Per esempio, se la funzione posizione nel tempo è $x_{t} ~ = ~ at^2 ~ la ~ v_{t}
~ sara' ~ v_{t} ~ =~  2at$
\paragraph{Piccola osservazione} Quando diciamo che $\lim{\Delta t \to 0}$ 
$\frac{\Delta x}{\Delta t} = v_{istantanea}$, stiamo dando per assodato che
diminuendo l'intervallo di tempo, diminuirà il relativo spostamento, pertanto
otterremo $\frac{0}{0}$, ok, ma vedremo che appunto tenderà ad un valore finito.
\section{Come varia la velocità}
Quando ho una velocità che cambia, posso definire la variazione della velocità 
vettoriale nell'unità di tempo, e questa si chiamerà accelerazione vettoriale media
e la indichiamo con $\vec{a_{media}}$, MA come prima servirà trovare l'accelerazione 
istantanea. Come si trova? Come prima si usano i limiti.
\\ \\
Per $\Delta t \to 0$ abbiamo che $\vec{a(t)} = \lim{\Delta t \to 0} 
\frac{\vec{\Delta v}}{\Delta t} = \frac{dv}{dt} $
\\ \\
Data la velocità istantanea v(t) possiamo ricavare che l'accelerazione: 
$$a = a(t) = \frac{d x(t)}{dt} = \frac{d^{2}x}{dt^{2}}$$
Quindi di conseguenza ne segue che 
$$v(t) = v(t_{0}) + \int_{t_{0}}^{t}a(t)~dt ~~~~~ (Con ~ t\geq t_{0}) $$ e
di conseguenza: $$x(t) = x(t_{0}) + \int_{t_{0}}^{t}v(t)~dt  $$
Considerando la velodità media si sta ovviamente considerando una velocità costante, 
graficamente, se v = $\frac{Spazio}{Tempo}$, allora v rappresenta il coefficiente
angolare della nostra retta.
\\ \\
Se per ipotesi avessimo l'accelerazione costante invece avremmo non più 
$\frac{\Delta x}{\Delta t}$ MA $\frac{\Delta v}{\Delta t}$, che rappresenta 
l'accelerazione media mantenendo per ovvio che
\begin{itemize}
	\item $\Delta t = t_{1} - t_{0}$
	\item $\Delta v = \Delta t \cdot a_{Media}$ 
	\item $\Delta v = v_{1} - v_{0}$
\end{itemize}
Si riesce a ricavare lo spostamento che a questo punto diventa 
$$\Delta x = v_{0}\Delta t +(\Delta v) \cdot \frac{\Delta t}{2} =
\Delta t \cdot \frac{v_{1}+v_{0}}{2} = \Delta t(v_{media})$$
A questo punto proviamo a ricavare la posizione $x_{1}$:
$$x_{1} = x_{0} + \frac{v_{1}}{v_{0}}{2}\cdot \Delta t$$
Dove in pratica $v_{1} = v_{0} + a_{Media} \cdot \Delta t$, sostituendo vien fuori
$$x_{1} = x_{0} \frac{v_{0}+a_{Media}\cdot\Delta t + v_{0}}{2} \cdot \Delta t$$
Se generalizziamo:
$$x(t) = x_{0} + v_{0}(t-t_{0})+\frac{a_{Media}}{2}(t-t_{0})^{2}$$
Successivamente, per finire, noteremo che:
$$x(t) = x_{0}+\int_{0}^{t}v_{0} dt + \int_{0}^{t} a_{media}t\cdot dt =$$
$$= x_{0} + v_{0}t + \frac{1}{2}at^{2}$$
\chapter{Moto rettilineo uniformemente accelerato}
Dato qualsiasi piano inclinato, se un corpo parte dall'alto da un punto $x_{0}$
se esso rotola (con attrito volvente che è trascurabile), noteremo che con il
quadrato del tempo ($t^{2}$) la sua velocità si incrementerà
$x = x_{0} + v_{0}t + \frac{1}{2}at^{2}$, in cui possiamo giocare di nuovo a 
modificare le cose belle, tipo la t vi $v_{0}\cdot t$ può diventare 
$\frac{v-v_{0}}{a}$, da cui consegue che $a = \frac{v-v_{0}}{t}$ e quindi 
$x = x_{0} + \frac{1}{2}(v_{0}+v)\cdot t$.
\section{Accelerazione di gravità}
Escludendo le forze di attrito con l'aria quando un corpo cade in caduta libera
si ottiene che esso (sul nostro pianeta) cada accelerando di 9,8 $\frac{m}{s^{2}}$.
Questi corpi prendono il nome di Gravi, e non importa che massa abbiano SE 
consideriamo l'assenza di aria che faccia attrito.
\\ \\
Poniamo caso di lanciare un sasso verso l'alto, o anche un mattone, MacBook, il
proprio gatto, quel che volete, noterete che questo salirà, e poi dopo un breve
periodo inizierà a tornare giù, questo è per via del fatto che l'accelerazione
gravitazionale è costante, ma all'oggetto che lanciamo cosa accade?
\\ \\
Noi sappiamo che $v = v_{0} + a\cdot t$, il grafico sarà una parabola che va
verso l'alto e poi ad un certo punto smetterà di crescere e inizierà a 
decrescere, e nel momento in cui il corpo scende l'accelerazione noteremo che
sarà $\leq$ 0 (Vettorialmente).
\paragraph{ATTENZIONE} L'accelerazione non si annulla MA sul punto di massimo
locale dove cambia il verso dell'accelerazione si avrà una velocità = 0
\chapter{Sistema cartesiano}
Il grafico di un qualsiasi sistema cartesiano è composto da due assi, uno x ed
uno y, perpendicolari tra essi, ed ogni punto sul grafico è composto da due
coordinate (x, y) tali che $P = (x_{P}, y_{P})$
\section{Coordinate polari} 
Ogni punto ha una distanza dall'origine, quella distanze si chiama R, e l'angolo
che si forma tra l'asse x e R si chiama $\phi$, di conseguenza ricaviamo che:
\begin{itemize}
	\item $\frac{y}{R} = \sin(\phi)$
	\item $\frac{x}{R} = \cos(\phi)$
	\item $y = R \cos(\phi)$
	\item $x = R \cos(\phi)$
	\item $\frac{y}{x} = \tan(\phi)$
	\item $\phi = \arctan(\frac{y}{x})$
	\item $R = \sqrt{x^{2}+y^{2}}$
\end{itemize}
Tutto ciò rimanendo comunque nel secondo quadrante, nel primo chiaramente c'è
da dire che rimane tutto cambiato di segno e viene indicato con le stesse lettere
del primo ma aggiungendoci un \' dopo.
\section{Somma tra vettori}
Dati due vettori è possibile individuarne la somma tramote la regola del 
parallelogramma che consiste nel tracciare due semirette parallele ai due 
vettori che abbiamo, e vanno aggiunte al termine del segmento del vettore
già presente, l'incrocio tra queste due semirette sarà il punto da cui puoi
calcolarti la somma (Non si capisce, ok, senza grafici è impossibile ma 
come già detto, li aggiungerò.) \\ \\
Se ragioniamo nelle due dimensioni la questione del nostro moto si complica, 
perchè abbiamo:
$$\overrightarrow{r} = \widehat{i}\cdot x + \widehat{j} y$$ e quindi
$$\frac{d \overrightarrow{v} (t)}{dt} = \widehat{i} \cdot \frac{dx}{dt}+ 
\widehat{j} \cdot \frac{dy}{dt}$$
Scritto più easy: $\widehat{i} v_{x} + \widehat{y} = \overrightarrow{v}$

$$\overrightarrow{v_{Media}}= \Delta \frac{\overrightarrow{x}}{\Delta t} = 
\frac{\overrightarrow{r}}{\Delta t} $$ dove $\Delta r = r_{1} - r_{0}$ e sarebbe
la distanza vettoriale.  \\ 
$\overrightarrow{v}(t) = \frac{\Delta \overrightarrow{r}}{\Delta t}$ con 
$\Delta t \to 0$
\\ \\ 
L'obbiettivo è quello di passare dal livello 1D al 2D, e la trigonometria entra
in gioco per via del fatto che vengono a formarsi dei triangoli


\end{document}
