\documentclass[12pt, a4paper, openany, twoside]{book}
\usepackage[italian]{babel}
\usepackage[T1]{fontenc}
\usepackage[utf8]{inputenc}
\usepackage{amsmath} 
\usepackage{xcolor}
%usepackage[latin1]{inputenc}
\begin{document}
\pagestyle{plain}
\author{DaveRhapsody}
\title{Fisica}
\date{30 Settembre 2019}
\maketitle
\tableofcontents
\chapter{Introduzione al corso}
Non è presente materiale didattico, le lezioni sono architettate in modo che
si segua dalla lavagna, è consigliato dal prof stesso di usare gli appunti od i
libri che (per coloro che han fatto fisica) si usavano alle superiori.
\newline \newline
Il programma è \textbf{tutta la fisica} in generale, ma affrontata in modo 
semplice, quasi banale, l'ultimo argomento dovrebbe essere il magnetismo, 
immaginatevi
quanto (non) si farà di quell'argomento. Ci sono 5 appelli in un anno, il primo
sarà a gennaio, poi febbraio, giugno, luglio e settembre, MA Gennaio e 
Febbraio dell'anno dopo sono inclusi
\newline \newline
Il che significa che io posso fare i due parziali e poi fare l'orale anche a
Febbraio. Noi possiamo iscriverci solo allo scritto, e verremo spostati 
all'orale SE siamo già sufficienti. 
\end{document}