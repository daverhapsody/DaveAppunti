\documentclass[12pt, a4paper, openany, twoside]{book}
\usepackage[italian]{babel}
\usepackage[T1]{fontenc}
\usepackage[utf8]{inputenc}
\usepackage{amsmath} 
\usepackage{xcolor}
\usepackage[margin=1in]{geometry}
\usepackage{hyperref}
\hypersetup{
colorlinks=true,
linkcolor=blue,
filecolor=magenta,      
urlcolor=cyan,
}
%usepackage[latin1]{inputenc}
\begin{document}
\pagestyle{headings}
\author{DaveRhapsody}
\title{Reti e Sistemi Operativi}
\date{3 Ottobre 2019}
\maketitle
\tableofcontents
\chapter*{La rete}
La rete è un insieme di nodi, che sono tendenzialmente pc, server, terminali,
e dispositivi radio che consentono ai dispositivi senza fili di collegarsi
(per intenderci, gli access point wi-fi), ma attenzione! \\ \\
Qui si parla di wi-fi, non della rete a cella telefonica, non si parla di 4G etc.
\section{Il modello TCP/IP}
Al centro è presente un Router, dispositivo in grado di INDIRIZZARE quelli che
sono i pacchetti IP (Internet Protocol). Cos'è un pacchetto?
\paragraph{Pacchetto}: E' un "Record/Struct" composto da un campo dati 
detto payload), ed un
campo header, che serve al protocollo di riferimento come etichetta che dice (ad 
un protocollo)  "Questo pacchetto lo puoi leggere".
Cosa contiene l'Header? 
$$header=\begin{cases}
indirizzi \\ livelli~di~priorità \\ contatori 
\end{cases}$$
A che servono i contatori? 	\\
Immagina di avere un problema nella rete in cui appare un routing loop (pensate
alla lista circolare)\\ \\
I pacchetti è auspicabile che siano di dimensioni piccine, immaginate un film 
di dimensioni tendenti al Gigabyte, un pacchetto da un Giga è pressochè (ad oggi)
insostenibile, pertanto, si riduce in pezzettini. Soprattutto c'è una probabilità
di errore più bassa! 
(Pensate di mangiarvi una Fiorentina in un solo boccone o di tagliarla prima un
pochino, stesso concetto.) \\ \\
Un altro motivo per cui vengono usati i pacchetti è legato al fatto che inviare
un qualsiasi dato occupa un canale. Intendiamoci, se da A a B (terminali) abbiamo un solo
cavo, non c'è problema, no?	\\ \\ Ok, ma se A e B sono degli switch aventi 10 pc 
connessi assieme? Se uno deve caricare un giga di roba la gente deve aspettare 
che questo finisca MA utilizzando un sistema a pacchetti, otteniamo che ci sia
una sorta di "turno" per ogni pacchetto. Di conseguenza non si intasa la luce.
\\ \\
Il sistema sopracitato si chiama multiplazione o multiplexing statistico, perchè
quando io metto tanti flussi di info che viaggiano sullo stesso collegamento, 
potrebbero accadere congestioni di rete, oppure comunque uno deve aspettare. Ecco
i pacchetti dei pacchetti che attendono si chiama buffer. Più hai host di rete,
più ti conviene aumentare la dimensione del buffer.
\\
\paragraph{Lo switch} Sono dei dispositivi che consentono di costruire la rete
con più livelli, proprio perchè ad essi vengono collegati altri dispositivi. 
Uno switch utilizza connessioni Ethernet, ed una serie di nodi collegati ad uno
switch forma una LAN. ATTENZIONE, quando si menziona l'access point, la rete 
che si forma si chiama WLAN, ovvero (Wireless LAN (Local area network)).
\paragraph{Terminale}
Il terminale (o host) sarà tendenzialmente un qualsiasi dispositivo che hosta 
applicazioni che lavorano in rete. 
\subparagraph{ATTENZIONE} la rete è identificata dal Router.
\\
Quando noi diciamo Internet, di fatto, stiamo parlando di un insieme di reti,
appunto inter - net, più reti, ergo, più router. C'è un agreement totale che 
consente ad un terminale in una zona, per poter arrivare ad un'altra.
\\ \\
Vi ricordate il cammino dei nodi? Ecco, stesso concetto.
\\
Questi nodi che si attraversano per passare da una rete A ad una rete B si 
chiamano autonomous systems, ed il cammino tra mittente e destinatario di un 
messaggio è identificato dagli autonomous per cui il pacchetto passo.
\\ \\
Più precisamente questi Autonomous sono dei Gateway, gestiti con un protocollo
(BGP, Border Gateway Protocol) in grado di regolare le comunicazioni tra questi AS.
\\ Ogni sistema autonomo, a sua volta all'interno avrà un certo insieme di altri 
router e reti, e quindi sarà presente un altro protocollo (IGP, Internal gateway 
protocol).
\paragraph{Sicurezza di un AS}
Ogni sistema autonomo è gestito da un "master" in grado di poter decidere cosa 
possa passare da quella rete, specifica le famose policy di Routing. La domanda
che sorge è.. Chi gestisce gli AS in toto? I gestori della rete, quelli telefonici.
\section{Cosa influenza le prestazioni di una rete?}
Analizzeremo due quantità
\paragraph{Latenza/ritardo}
La latenza (o ritardo) non è altro che il tempo che intercorre tra l'invio e la 
ricezione di un pacchetto. 
Esistono diversi tempi:
\begin{itemize}
	\item Tempo di processing: E' una quantità infinitesimale che indica il 
	tempo impiegato per elaboare un pacchetto
	\item Tempo di coda: Per misurarlo devo capire oggettivamente quanto viene
	usata quella coda (esistono interi studi sulle code, ma a noi ci importa
	una bellissima  ;) (Almeno, per questo corso si intende)) \\ \\
	Per un sistema stabile è auspicabile che il rapporto tra i dati contenibili
	nel buffer ed il rate di dati trasmessi al secondo sia $\leq$ 1, detto meglio:
	$$ \frac{nL}{R}\leq 1$$ 
\end{itemize}
\paragraph{Throughput}
Il through put è il quantitativo di dati che riescono ad essere trasferiti in 
un determinato quantitativo di tempo (Avete presente Mbps, Gbps, Kbps, quella
roba lì)
\section{Stratificazione}
Il concetto di stratificazione consiste nel includere le funzioni di una rete
non in un solo protocollo ma in una serie di protocollo che abbiano uno
specifico compito, e sia indipendente dagli altri 
\paragraph{La filosofia KISS}
Keep it simple stupid, è la filosofia che regola anche il mondo Unix e Gnu/Linux,
ovvero, ogni componente fa il suo mestiere, fa la sua funzione, evitare di 
mettere in mano ad un componente 10 compiti diversi, per intenderci. 
\\ \\ 
Inoltre dividendo in più componenti io ho possibilità di testare il singolo
componente, il singolo protocollo, il singolo problema SENZA, S E N Z A andare
a danneggiare o danneggiare gli altri protocolli/componenti!
\paragraph{Stack ISO/OSI}
Questo è lo stack più preciso ingrandito dell'TCP/IP, visto dal livello più 
basso al livello più alto
\begin{itemize}
	\item physical
	\item data link
	\item network
	\item transport
	\item session
	\item presentation
	\item application
\end{itemize}
\section{livello Physical}
Nel livello fisico viene convertito quello che è il segnale elettrico in una
sequenza di 0 ed 1
\section{Livello Data Link}
Si implementa MAC (Medium Access Control), converte quelli che erano
degli 0 ed 1 e li fa diventare dei pacchetti (yes, c'è il primo header, cioè 
quello del physical).
\\ \\ 
L'indirizzo MAC identifica univocamente un dispositivo in rete, QUALSIASI esso sia.
E non solo, mondialmente, N O N esistono due dispositivi DIVERSI aventi lo STESSO
Mac address. A che serve oltre questo? Serve al protocollo ip per capire CHI
è il mittente o destinatario, e sì, è incluso nell'header
\section{Rete} E' il livello di internet, l'IP protocol, con indirizzo ipv4 v6 etc.
\section{Trasporto}
E' il livello in grado di trasportare tramite internet quelli che sono i pacchetti
applicazione, trasporta i datagram ip, trasporta i pacchetti appunto tramite 
internet, sono presenti due protocolli (TCP e UDP) che si occupano di fare questo
\section{Applicazioni} Il nome parla da sè, non lo studieremo in questo corso.
\\ \\
Qui noi si studierà dal data link al transport MA in ordine invertito, ovvero
partiamo da "Che cosa vuole l'applicazione?" e poi andiamo a scalare.
\paragraph{Precisazione:}
Dato un pacchetto in un qualsiasi livello (escluso lv application), 
quando viene aggiunto l'header, il nuovo pacchetto potrà esser letto SOLO dal
protocollo del livello superiore! \\ \\
Su questo stesso piano, se ho un pacchetto con già l'header aggiunto, questo
potrà essere spacchettato SOLO da un protocollo del livello inferiore.
\\
{\color{black} \rule{\linewidth}{0.2mm} }
\paragraph{La magica teoria dell'incapsulamento di Dave} 
Per spiegare meglio cosa accade, immaginatevi una fetta di prosciutto, bene, 
questa fetta di prosciutto rappresenta il \color{red}flusso elettrico\color{black} che vien gestito 
dal livello fisico. \\
A questo punto, immaginatevi due fette di pane che si chiudono sulla fetta di 
prosciutto. Benissimo, il Data Link ha \textbf{INCAPSULATO} il prosciutto ed ha
creato un panino (\color{red}Pacchetto\color{black})! \\ \\
Il data link passerà questo panino al \color{red}Network\color{black}, e lì ci 
saranno altre due fette di pane che includeranno il nostro panino. 
Ed in questo modo abbiamo ottenuto un panino dentro ad un altro panino! Ossia,
 un pacchetto incapsulato in un pacchetto che semplicemente aggiunge un header 
 (o intestazione) ;) Ma torniamo seri adesso. \\
{\color{black} \rule{\linewidth}{0.2mm} }
\\


\end{document}