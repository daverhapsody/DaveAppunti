\documentclass[12pt, a4paper, openany, twoside]{book}
\usepackage[italian]{babel}
\usepackage[T1]{fontenc}
\usepackage[utf8]{inputenc}
\usepackage{amsmath} 
\usepackage{xcolor}
\usepackage[margin=1in]{geometry}
\usepackage{hyperref}
\hypersetup{
colorlinks=true,
linkcolor=blue,
filecolor=magenta,      
urlcolor=cyan,
}
%usepackage[latin1]{inputenc}
\begin{document}
\pagestyle{headings}
\author{DaveRhapsody}
\title{Reti e Sistemi Operativi}
\date{3 Ottobre 2019}
\maketitle
\tableofcontents
\chapter{La rete}
La rete è un insieme di nodi, che sono tendenzialmente pc, server, terminali,
e dispositivi radio che consentono ai dispositivi senza fili di collegarsi
(per intenderci, gli access point wi-fi), ma attenzione! \\ \\
Qui si parla di wi-fi, non della rete a cella telefonica, non si parla di 4G etc.
\section{Il modello TCP/IP}
Al centro è presente un Router, dispositivo in grado di INDIRIZZARE quelli che
sono i pacchetti IP (Internet Protocol). Cos'è un pacchetto?
\paragraph{Pacchetto}: E' un "Record/Struct" composto da un campo dati 
detto payload), ed un
campo header, che serve al protocollo di riferimento come etichetta che dice (ad 
un protocollo)  "Questo pacchetto lo puoi leggere".
Cosa contiene l'Header? 
$$header=\begin{cases}
indirizzi \\ livelli~di~priorità \\ contatori 
\end{cases}$$
A che servono i contatori? 	\\
Immagina di avere un problema nella rete in cui appare un routing loop (pensate
alla lista circolare)\\ \\
I pacchetti è auspicabile che siano di dimensioni piccine, immaginate un film 
di dimensioni tendenti al Gigabyte, un pacchetto da un Giga è pressochè (ad oggi)
insostenibile, pertanto, si riduce in pezzettini. Soprattutto c'è una probabilità
di errore più bassa! 
(Pensate di mangiarvi una Fiorentina in un solo boccone o di tagliarla prima un
pochino, stesso concetto.) \\ \\
Un altro motivo per cui vengono usati i pacchetti è legato al fatto che inviare
un qualsiasi dato occupa un canale. Intendiamoci, se da A a B (terminali) abbiamo un solo
cavo, non c'è problema, no?	\\ \\ Ok, ma se A e B sono degli switch aventi 10 pc 
connessi assieme? Se uno deve caricare un giga di roba la gente deve aspettare 
che questo finisca MA utilizzando un sistema a pacchetti, otteniamo che ci sia
una sorta di "turno" per ogni pacchetto. Di conseguenza non si intasa la luce.
\\ \\
Il sistema sopracitato si chiama multiplazione o multiplexing statistico, perchè
quando io metto tanti flussi di info che viaggiano sullo stesso collegamento, 
potrebbero accadere congestioni di rete, oppure comunque uno deve aspettare. Ecco
i pacchetti dei pacchetti che attendono si chiama buffer. Più hai host di rete,
più ti conviene aumentare la dimensione del buffer.
\\
\paragraph{Lo switch} Sono dei dispositivi che consentono di costruire la rete
con più livelli, proprio perchè ad essi vengono collegati altri dispositivi. 
Uno switch utilizza connessioni Ethernet, ed una serie di nodi collegati ad uno
switch forma una LAN. ATTENZIONE, quando si menziona l'access point, la rete 
che si forma si chiama WLAN, ovvero (Wireless LAN (Local area network)).
\paragraph{Terminale}
Il terminale (o host) sarà tendenzialmente un qualsiasi dispositivo che hosta 
applicazioni che lavorano in rete. 
\subparagraph{ATTENZIONE} la rete è identificata dal Router.
\\
Quando noi diciamo Internet, di fatto, stiamo parlando di un insieme di reti,
appunto inter - net, più reti, ergo, più router. C'è un agreement totale che 
consente ad un terminale in una zona, per poter arrivare ad un'altra.
\\ \\
Vi ricordate il cammino dei nodi? Ecco, stesso concetto.
\\
Questi nodi che si attraversano per passare da una rete A ad una rete B si 
chiamano autonomous systems, ed il cammino tra mittente e destinatario di un 
messaggio è identificato dagli autonomous per cui il pacchetto passo.
\\ \\
Più precisamente questi Autonomous sono dei Gateway, gestiti con un protocollo
(BGP, Border Gateway Protocol) in grado di regolare le comunicazioni tra questi AS.
\\ Ogni sistema autonomo, a sua volta all'interno avrà un certo insieme di altri 
router e reti, e quindi sarà presente un altro protocollo (IGP, Internal gateway 
protocol).
\paragraph{Sicurezza di un AS}
Ogni sistema autonomo è gestito da un "master" in grado di poter decidere cosa 
possa passare da quella rete, specifica le famose policy di Routing. La domanda
che sorge è.. Chi gestisce gli AS in toto? I gestori della rete, quelli telefonici.
\section{Cosa influenza le prestazioni di una rete?}
Analizzeremo due quantità
\paragraph{Latenza/ritardo}
La latenza (o ritardo) non è altro che il tempo che intercorre tra l'invio e la 
ricezione di un pacchetto. 
Esistono diversi tempi:
\begin{itemize}
	\item Tempo di processing: E' una quantità infinitesimale che indica il 
	tempo impiegato per elaboare un pacchetto
	\item Tempo di coda: Per misurarlo devo capire oggettivamente quanto viene
	usata quella coda (esistono interi studi sulle code, ma a noi ci importa
	una bellissima  ;) (Almeno, per questo corso si intende)) \\ \\
	Per un sistema stabile è auspicabile che il rapporto tra i dati contenibili
	nel buffer ed il rate di dati trasmessi al secondo sia $\leq$ 1, detto meglio:
	$$ \frac{nL}{R}\leq 1$$ 
\end{itemize}
\paragraph{Throughput}
Il through put è il quantitativo di dati che riescono ad essere trasferiti in 
un determinato quantitativo di tempo (Avete presente Mbps, Gbps, Kbps, quella
roba lì)
\section{Stratificazione}
Il concetto di stratificazione consiste nel includere le funzioni di una rete
non in un solo protocollo ma in una serie di protocollo che abbiano uno
specifico compito, e sia indipendente dagli altri 
\paragraph{La filosofia KISS}
Keep it simple stupid, è la filosofia che regola anche il mondo Unix e Gnu/Linux,
ovvero, ogni componente fa il suo mestiere, fa la sua funzione, evitare di 
mettere in mano ad un componente 10 compiti diversi, per intenderci. 
\\ \\ 
Inoltre dividendo in più componenti io ho possibilità di testare il singolo
componente, il singolo protocollo, il singolo problema SENZA, S E N Z A andare
a danneggiare o danneggiare gli altri protocolli/componenti!
\paragraph{Stack ISO/OSI}
Questo è lo stack più preciso ingrandito dell'TCP/IP, visto dal livello più 
basso al livello più alto
\begin{itemize}
	\item physical
	\item data link
	\item network
	\item transport
	\item session
	\item presentation
	\item application
\end{itemize}
\section{livello Physical}
Nel livello fisico viene convertito quello che è il segnale elettrico in una
sequenza di 0 ed 1
\section{Livello Data Link}
Si implementa MAC (Medium Access Control), converte quelli che erano
degli 0 ed 1 e li fa diventare dei pacchetti (yes, c'è il primo header, cioè 
quello del physical).
\\ \\ 
L'indirizzo MAC identifica univocamente un dispositivo in rete, QUALSIASI esso sia.
E non solo, mondialmente, N O N esistono due dispositivi DIVERSI aventi lo STESSO
Mac address. A che serve oltre questo? Serve al protocollo ip per capire CHI
è il mittente o destinatario, e sì, è incluso nell'header
\section{Rete} E' il livello di internet, l'IP protocol, con indirizzo ipv4 v6 etc.
\section{Trasporto}
E' il livello in grado di trasportare tramite internet quelli che sono i pacchetti
applicazione, trasporta i datagram ip, trasporta i pacchetti appunto tramite 
internet, sono presenti due protocolli (TCP e UDP) che si occupano di fare questo
\section{Applicazioni} Il nome parla da sè, non lo studieremo in questo corso.
\\ \\
Qui noi si studierà dal data link al transport MA in ordine invertito, ovvero
partiamo da "Che cosa vuole l'applicazione?" e poi andiamo a scalare.
\paragraph{Precisazione:}
Dato un pacchetto in un qualsiasi livello (escluso lv application), 
quando viene aggiunto l'header, il nuovo pacchetto potrà esser letto SOLO dal
protocollo del livello superiore! \\ \\
Su questo stesso piano, se ho un pacchetto con già l'header aggiunto, questo
potrà essere spacchettato SOLO da un protocollo del livello inferiore.
\\
{\color{black} \rule{\linewidth}{0.2mm} }
\paragraph{La magica teoria dell'incapsulamento di Dave} 
Per spiegare meglio cosa accade, immaginatevi una fetta di prosciutto, bene, 
questa fetta di prosciutto rappresenta il \color{red}flusso elettrico\color{black} che vien gestito 
dal livello fisico. \\
A questo punto, immaginatevi due fette di pane che si chiudono sulla fetta di 
prosciutto. Benissimo, il Data Link ha \textbf{INCAPSULATO} il prosciutto ed ha
creato un panino (\color{red}Pacchetto\color{black})! \\ \\
Il data link passerà questo panino al \color{red}Network\color{black}, e lì ci 
saranno altre due fette di pane che includeranno il nostro panino. 
Ed in questo modo abbiamo ottenuto un panino dentro ad un altro panino! Ossia,
un pacchetto incapsulato in un pacchetto che semplicemente aggiunge un header 
(o intestazione) ;) Ma torniamo seri adesso. \\
{\color{black} \rule{\linewidth}{0.2mm} }
\\
\chapter{Livello Trasporto}
Il livello di trasporto è il livello che riceve messaggi dal livello application
ed ha come compito quello di mettere in comunicazione end-to-end due nodi.
Nel caso del livello di trasporto non si parla di pacchetti ma di Segmenti (
Dall'esempio sopra riportato sì, è stato incapsulato il pacchetto del lv network).
\\ \\
Quando viene analizzato il segmento si effettua la DEMULTIPLAZIONE 
(Demultiplexing), ovvero si analizza l'header del segmento per vedere a quale
applicazione sarà destinataria di un determinato messaggio.
\paragraph{Cosa contiene un segmento di trasporto?}
\begin{itemize}
	\item Numero di porta: presente in ogni segmento, rappresenta una applicazione
	e dal punto di vista dell'applicazione è come se fosse un punto di 
	accesso, detto anche SOCKET. 
	\paragraph{NON CONFONDIAMOCI} Il livello transport vede una porta, mentre
	il livello applicazione vede un socket. All'atto pratico c'è una syscall
	che letteralmente può attivare qualsiasi socket.
	\item Protocollo Transport TCP o UDP.
\end{itemize}
\paragraph{Che differenza c'è tra TCP e UDP?}
UDP è quello che usate negli streaming di Rojadirecta
, in cui pur di vedere la vostra squadra 
(che ovviamente è la Fiorentina, vero? <3) non vi preoccupate di perdere qualche
dato (tipo immagine un po' sgranata ogni tanto e simili).
\subparagraph{Infatti} UDP se ne frega se un pacchetto è arrivato, è usato
per comunicazioni anche per esempio le voice chat, roba di questo genere, poichè
\textbf{NON} effettua alcun controllo sulla correttezza dei messaggi, diciamo
che lui ti invia roba, se non ti arriva non gli importa, va avanti, perchè qui
è importante che si sia il più aggiornati possibile.
\\ 
Al contrario TCP effettua questi controlli, perchè con TCP si garantisce la 
correttezza (sia dei dati stessi che dell'ordine in cui arrivano), pertanto 
potrebbe essere un problema.
\\ \\
Il segmento può avere sia TCP che UDP, e nel caso di UDP si ha un IP +
Porta di destinazione, o meglio
$$Pacchetto UDP~ =~ \begin{cases}
SourcePort: ~Porta~ sorgente \\
DestinationPort: ~Porta~ di ~destinazione \\
Lunghezza: Indica~ la~ lunghezza ~totale ~del ~segmento~ (+payload~ \\dei ~livelli
~inferiori)\\
Checksum:~ Verifica ~la ~correttezza ~di ~trasmissione \\
\end{cases}$$
\\ 
Supponiamo di avere un web server, (quello che consente di vedere una pagina web)
, la porta d'accesso è la stessa ma hai più utenti con la stessa pagina
\\ \\
Nel caso di TCP invece un segmento è composto non solo dall'IP + la porta di 
destinazione, ma è composto in questo modo:
$$\begin{cases}
IP_{A} \\
IP_{B} \\
PORT_{A} \\
PORT_{B}
\end{cases}$$
A e B in questo caso sono tendenzialmente sorgente e destinataria, ed il calcolo
del checksum precedente è effettuato direttamente dal livello Transport stesso.
\\ \\
Attenzione, nell'introduzione abbiamo anche detto che già nel datalink abbiamo 
dei controlli, pertanto ogni livello può svolgere la correzione dei pacchetti,
pertanto non si attiva quasi mai, perchè Transport dà per scontato che Network
e Datalink gli abbian passato qualcosa di corretto.
\\ \\
Il codice di implementazione del controllo è tipo di migliaia di righe di codice
per correggere o prevenire errori, ed è giustissimo così. Contiamo che ormai TCP
e UDP non si possono rompere, sono più che sicuri.
\section{TCP}
Il concetto è che si vuole trasferire dati in modo affidabile, o meglio, si 
vuole fare in modo che i dati arrivino a destianazione 
e siano corretti in generale.
\\ \\ 
Il trasferimento dati affidabile dipende da una serie di fattori, e si verifica
quando ho una rete con le seguenti caratteristiche:
\begin{enumerate}
	\item Nessuna perdita
	\item Nessun errore
	\item I dati son corretti e presenti tutti quanti
\end{enumerate} 
\paragraph{Nel caso avvengano errori} si può ricorrere a diverse metodiche di
soluzione, come ad esempio la tecnica di ritrasmissione. Ossia, non ti è 
arrivato il mio pacchetto? Ok, te lo rimando. Nell'atto pratico è una cosa del tipo
$$\begin{cases}
if(ricevoAcknoledgment)~then~mandoSucessivo(); \\
else~if(ricevoNotAcknoledgment)~rimandoPacchetto();
\end{cases}$$
Quello che chiamo not acknoledgment (che da ora chiameremo solo ACK) viene 
inviato SE il pacchetto è non corretto, per esempio per il checksum.
\\ \\
E se il pacchetto poi è lo stesso? Ne ricevo due? No. I pacchetti sono numerati
hanno un ordine, pertanto
$$\begin{cases}
if(ricevoACK) m+1;\\
else m
\end{cases}$$
con m che è il "pezzetto" successivo o precedente, è importante che si capisca 
questo.
\\ \\ Ora cerchiamo di integrare, è necessario che il sistema mi mandi indietro
sì se ha ricevuto ma ANCHE la posizione a cui è arrivato! (Sistema PAR, positive
acknoledgmente with retransmition).
\\ \\
Quindi mi mandi un messaggio e io ti rispondo dicendo ok messaggio, tu mi mandi
il due, io rispondo ok due, tu mi mandi tre io rispondo ok due, mi rimandi tre.
\paragraph{Perdite}
Sì, ok, ma se uno non mi riesce a rispondere? Si imposta un semplice timer
la cui scadenza implica che ti rimando il pacchetto. Un po' come quando dici
una cosa con la musica alta a uno e lui ti guarda tipo fisso... Passa tempo, e
poi ripeti la cosa. 
\\ \\ 
E se io continuo a rispondere? Lui che fa continua come un pazzo a ripetere la
stessa cosa come Ciuchino che chiedeva a Shrek ogni 2 millisecondi
"Siamo arrivati?"? NO! C'è un contatore che si aziona ad ogni mancata risposta,
una volta che arriva ad un tot, chiudo la comunicazione.
$$\begin{cases}
if(timerGoesOff) ~ trasmetti(messaggio_{i}) ~ AND ~ setTimer(0); \\
else ~if(ACK_{i}) ~trasmetti(messaggio_{i+1} ~ AND~  setTimer(0)); \\
else~ trasmetti(messaggio_{i}) ~ AND ~ setTimer(0);
\end{cases}$$
\paragraph{ATTENZIONE} Supponendo un timer troppo corto, accadrebbe che il 
destinatario non fa in tempo a rispondere, e quindi il mittente rifà la domanda
prima di ricevere la risposta, hai già due domande con una risposta. \\ \\ 
All'atto pratico questo problema esiste davvero, è un problema reale, pertanto si deve
riuscire a capire il modo in cui il timer vada adattato alle condizioni di rete,
che a seconda del traffico può essere più lenta o anche più veloce.\\ \\
\subsection{I tempi di una connessione}
Riprendiamo il concetto di latenza (tempo che intercorre tra invio e ricezione),
bene, quello è concettualmente il tempo di trasmissione, da qui possiamo calcolare che
$$utilizzoRete = \frac{\frac{L}{R}}{latenza+\frac{L}{R}}$$ 
in cui $\frac{L}{R}$ Ha R che è la velocità di trasferimento
dei pacchetti, mentre invece L sarebbe la lunghezza
, ossia: $\frac{L}{R} = \frac{Lunghezza}{Velocita'~trasmissione}$
E la latenza... E' la latenza. 
\subsection{Protocolli a Finestra Scorrevole (Sliding Windows)}
Io trasferisco un insieme di w pacchetti, i quali impiegheranno $w*\frac{L}{R} 
\geq \frac{L}{R}+RTT$ tempo ad esser trasposti, pertanto ne esce che
$$w \geq \frac{RTT\cdot R}{L} +1$$
Questo ovviamente in mancanza di errori e perdite.
\\ \\
\subparagraph{Osservazione:} Stiamo ragionando su sistemi concettualmente 
indistruttibili, quindi diciamo che non si è quasi menzionata la possibilità che
un host possa disconnettersi.
\subsection{Go Back N} 
Poniamo caso che io trasmetto 5 messaggi: $m_{1, ..., 5}$ e mi arriva l'ok di
ricezione dall'1 al 3, malgrado li abbia trasmessi tutti e 5, pertanto ciò che
accade è che rinvierò dal messaggio 3 in poi.
\subsection{Selective Repeat}
Traccio la vita di ogni pacchetto individualmente, ossia ritrasmetto solo il
pacchetto mancante, il problemino è legato al fatto di avere tanti timer, ma
oggi c'è da considere che abbiamo processori nell'ordine dei Ghz, no problem.
\\ \\ 
Quando all'epoca il massimo processore esistente era di massimo 10 Mhz, insomma
chiaro che ora essa sia la migliore delle soluzione. Ma cosa richiede più precisamente
l'SR?
\begin{itemize}
	\item Ack individuali per ogni pacchetto
	\item Un timer per ogni singolo pacchetto
	\item Buffer sul ricevitore (E' un array di booleani all'atto pratico)
\end{itemize}
Ci sono dei casi in cui ci sono interazioni tra numero di pacchetto e dimensione
finestra, ma sono problematiche di cui non approfondiamo perchè non ci interessa
siccome non può praticamente mai capitare
\end{document}