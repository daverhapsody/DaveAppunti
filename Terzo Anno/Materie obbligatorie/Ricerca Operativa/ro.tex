\documentclass[12pt, a4paper, openany, twoside]{book}
\usepackage[italian]{babel}
\usepackage[T1]{fontenc}
\usepackage[utf8]{inputenc}
\usepackage{amsmath} 
\usepackage{xcolor}
\usepackage[margin=1in]{geometry}
\usepackage{hyperref}
\usepackage{graphicx}
\graphicspath{{./img/}}
\usepackage{tikz}
\hypersetup{
    colorlinks=true,
    linkcolor=blue,
    filecolor=magenta,      
    urlcolor=cyan,
}
\begin{document}
\fontfamily{cmss}\selectfont
\pagestyle{plain}
\author{DaveRhapsody}
\title {Ricerca operativa}
\date {28 Settembre 2020}
\maketitle
\tableofcontents
\chapter{Introduzione}
\section{Organizzazione del corso}
Gli argomenti principali del corso saranno:
\begin{enumerate}
	\item Modelli matematici per la Ricerca Operativa
	\item Programmazione lineare
	\item Durata e analisi di sensitività
	\item Programmazione lineare a numeri interi
	\item Ottimizzazione non-lineare
	\item Meta-euristiche
\end{enumerate}
Anche qua ci sono i compitini e l'esame totale, come al solito insomma. Per quanto
riguarda i compitini c'è il recupero di uno dei due dopo il secondo parziale, 
però solo uno dei due, come per LC. 
\section{Assignments}
Ci saranno degli assignments, dei compiti a casa insomma, da consegnare entro 
una certa data, oltre la quale non accettano consegna. Il voto per ognuno di 
questi è da 0 a 1. Non sono troppo pesanti in fase di correzione, l'importante è
farli.
\section{Prova orale}
Non obbligatoria ma c'è, son 3 domandine aperte o esercizi su tutto il programma,
ci puoi accedere solo se hai un voto di almeno 18. Ci puoi fare da -3 a +3 punti,
quindi non è detto che convenga farlo.
\section{Lezioni}
Le lezioni sono asincrone, quindi con i video (c'è uno schedule), mentre le
esercitazioni invece sono sincrone, ma c'è lo stesso la registrazione è resa
disponibile dopo un giorno, domande e risposte non vengono registrate.
Per le domande c'è un forum online dove sottoporre tutte le varie perplessità. 
\end{document}